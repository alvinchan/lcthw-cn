\chapter{Exercise 27: Creative And Defensive Programming}

You have now learned most of the basics of C programming and are ready to start becoming a serious
programmer.  This is where you go from beginner to expert, both with C and hopefully with
core computer science concepts.  I will be teaching you a few of the core data structures and
algorithms that every programmer should know, and then a few very interesting ones I've used
in real software for years.

Before I can do that I have to teach you some basic skills and ideas that will help you
make better software.  Exercises 27 through 31 will teach you advanced concepts and feature
more talking than code, but after those you'll apply what you learn to making a core library
of useful data structures.

The first step in getting better at writing C code (and really any language) is
to learn a new mindset called "defensive programming".  Defensive programming
assumes that you are going to make many mistakes and then attempts to prevent
them at every possible step.   In this exercise I'm going to teach you how 
to think about programming defensively.

\section{The Creative Programmer Mindset}

It's not possible to tell you how to be creative in a short exercise like this,
but I will tell you that creativity involves taking risks and being open minded.
Fear will quickly kill creativity, so the mindset I adopt, and many programmers
adopt on, accident is designed to make me unafraid of taking chances and 
looking like an idiot:

\begin{enumerate}
\item I can't make a mistake.
\item It doesn't matter what people think.
\item Whatever my brain comes up with is going to be a great idea.
\end{enumerate}

I only adopt this mindset temporarily, and even have little tricks to turn it on.
By doing this I can come up with ideas, find creative solutions, open my thoughts
to odd connections, and just generally invent weirdness without fear.  In this
mindset I will typically write a horrible first version of something just to get
the idea out.

However, when I've finished my creative prototype I will throw it out and 
get serious about making it solid.  Where other people make a mistake is
carrying the creative mindset into their implementation phase.  This
then leads to a very different destructive mindset that is the dark side
of the creative mindset:

\begin{enumerate}
\item It is possible to write perfect software.
\item My brain tells me the truth, and it can't find any errors, therefore I have written perfect software.
\item My code is who I am and people who criticize its perfection are criticizing me.
\end{enumerate}

These are lies.  You will frequently run into programmers who feel intense
pride about what they've created, which is natural, but this pride gets in the
way of their ability to objectively improve their craft.  Because of pride and
attachment to what they've written, they can continue to believe that what they
write is perfect.  As long as they ignore other people's criticism of their
code they can protect their fragile ego and never improve.

The trick to being creative \emph{and} making solid software is to 
also be able to adopt a defensive programming mindset.


\section{The Defensive Programmer Mindset}

After you have a working creative prototype and you're feeling good about the
idea, it's time to switch to being a defensive programmer.  The defensive
programmer basically hates your code and believes these things:

\begin{enumerate}
\item Software has errors.
\item You are not your software, yet you are are responsible for the errors.
\item You can never remove the errors, only reduce their probability.
\end{enumerate}

This mindset lets you be honest about your work and critically analyze it 
for improvements.  Notice that it doesn't say \emph{you} are full of errors?
It says your \emph{code} is full of errors.  This is a significant thing to
understand because it gives you the power of objectivity for the next
implementation.

Just like the creative mindset, the defensive programming mindset has a
dark side as well.  The defensive programmer is a paranoid who is afraid
of everything, and this fear prevents them from possibly being wrong or
making mistakes.  That's great when you are trying to be ruthlessly
consistent and correct, but it is murder on creative energy and 
concentration.


\section{The Eight Defensive Programmer Strategies}

Once you've adopted this mindset, you can then rewrite your prototype and follow
a set of eight strategies I use to make my code as solid as I can.  While I work
on the "real" version I ruthlessly follow these strategies and try to remove
as many errors as I can, thinking like someone who wants to break the software.


\begin{description}
\item[Never Trust Input] Never trust the data you are given and always validate it.
\item[Prevent Errors] If an error is possible, no matter how probable, try to prevent it.
\item[Fail Early And Openly] Fail early, cleanly, and openly, stating what happened, where and how to fix it.
\item[Document Assumptions] Clearly state the pre-conditions, post-conditions, and invariants.
\item[Prevention Over Documentation] Do not do with documentation, that which can be done with code or avoided completely.
\item[Automate Everything] Automate everything, especially testing.
\item[Simplify And Clarify] Always simplify the code to the smallest, cleanest form that works without sacrificing safety.
\item[Question Authority] Do not blindly follow or reject rules.
\end{description}

These aren't the only ones, but they're the core things I feel programmers have
to focus on when trying to make good solid code.  Notice that I don't really
say exactly how to do these.  I'll go into each of these in more detail, and
some of the exercises actually cover them extensively.


\section{Applying The Eight Strategies}

I'll now go through each of the eight strategies and give some basic advice and examples on how to use them in real code.
This will help you understand these better since they may be vague or misinterpreted.


\subsection{Never Trust Input}

Explain untrusted inputs. Use a few examples from the real world, maybe the recent Rails and Github attack.
Show a simple C examples using C strings taken from a socket.

\subsection{Prevent Errors}

Discuss the difference between a possible error and a probable error, then how humans are very bad
at determining probability, therefore you should try to block all the possible errors you can.


\subsection{Fail Early And Openly}

Show how my awesome macros help you do this and discuss good error messages.  Make sure they understand
that you should try to explain how to maybe fix it, or report the defect.

\subsection{Document Assumptions}

Explain design by contract and how you can use it to create pre-conditions, post-conditions, and 
invariants.  Show a simple example of this with a function that has it all.

\subsection{Prevention Over Documentation}

Talk about how programmers think that a documented flaw means there is no flaw and that they
should just remove the flaw.

\subsection{Automate Everything}

Discuss the advantages of automated testing and how I'll teach that in a later exercise.

\subsection{Simplify And Clarify}

Talk about how simpler code wins over more complex code because it reduces the probability of
an error by reducing the number of branches and couplings in the code.

\subsection{Question Authority}

The final strategy is the most important because it breaks you out of the
defensive programming mindset and lets you transition into the creative
mindset.  Defensive programming is authoritarian and it can be cruel.  The
job of this mindset is to make you follow rules because without them you'll
miss something or get distracted.

This authoritarian attitude has the disadvantage of disabling independent 
creative thought.  Rules are necessary for getting things done, but being
a slave to them will kill your creativity.

This final strategy means you should question the rules you follow periodically
and assume that they could be wrong, just like the software you are reviewing.
What I will typically do is, after a session of defensive programming, I'll go
take a non-programming break and let the rules go.  Then I'll be ready to do
some creative work or do more defensive coding if need to.

\section{Order Is Not Important}

The final thing I'll say on this philosophy is that I'm not telling you to do this in
a strict oder of "CREATE! DEFEND! CREATE! DEFEND!"  At first you may want to do that,
but I will actually do either in varying amounts depend on what I want to do, and I may
even meld them together with no defined boundary.

I also don't think one mindset is better than another, or that there are strict separation
between them.  You need both creativity and strictness to do programming well, so work
on both if you want to improve.


\section{Extra Credit}

\begin{enumerate}
\item The code in the book up to this point (and for the rest of it) potentially violates these rules.  Go back through and apply what you've learned to one 
    exercise to see if you can improve it or find bugs.
\item Find an open source project and give some of the files a similar code review.  Submit a patch that fixes a bug if you find it.
\end{enumerate}

