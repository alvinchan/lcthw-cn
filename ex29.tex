\chapter{Exercise 29: Libraries And Linking}

A central part of any C program is the ability to link it to libraries that
your operating system provides.  Linking is how you get additional features
for your program that someone else created and packaged on the system.  You've
been using some standard libraries that are automatically included, but I'm
going to explain the different types of libraries and what they do.

First off, libraries are poorly designed in every programming language.  I have 
no idea why, but it seems language designers think of linking as something they
just slap on later.  They are usually confusing, hard to deal with, can't do
versioning right, and end up being linked differently everywhere.

C is no different, but the way linking and libraries are done in C is an artifact
of how the Unix operating system and executable formats were designed years ago.
Learning how C links things helps you understand how your OS works and how it
runs your programs.

To start off there are two basic types of libraries:

\begin{description}
\item[static] You've made one of these when you used \program{ar} and \program{ranlib}
    to create the \file{libYOUR\_LIBRARY.a} in the last exercise.  This kind of library
    is nothing more than a container for a set of \file{.o} object files and their
    functions, and you can treat it like one big \file{.o} file when building 
    your programs.
\item[dynamic] These typically end in \file{.so}, \file{.dll} or about 1 million other
    endings on OSX depending on the version and who happened to be working that day.
    Seriously though, OSX adds \file{.dylib}, \file{.bundle}, and \file{.framework} with not much distinction between the three. These files are built and then placed
    in a common location.  When you run your program the OS dynamically loads these
    files and links them to your program on the fly.
\end{description}

I tend to like static libraries for small to medium sized projects because they are
easier to deal with and work on more operating systems.  I also like to put all of the
code I can into a static library so that I can then link it to unit tests and to the
file programs as needed.

Dynamic libraries are good for larger systems, when space is tight, or if you have
a large number of programs that use common functionality.  In this case you don't
want to statically link all of the code for the common features to every program,
so you put it in a dynamic library so that it is loaded only once for all of them.

In the previous exercise I laid out how to make a static library (a \file{.a} file),
and that's what I'll use in the rest of the book.  In this exercise I'm going to 
show you how to make a simple .so library, and how to dynamically load it with the
Unix \program{dlopen} system.  I'll have you do this manually so that you understand
everything that's actually happening, then the Extra Credit will be to use the
\file{c-skeleton} skeleton to create it.

\subsection{Dynamically Loading A Shared Library}

To do this I will create two source files.  One will be used to make a 
\file{libex29.so} library, the other will be a program called \program{ex29}
that can load this library and run functions from it.

\begin{code}{libex29.c}
<< d['code/c-skeleton/src/libex29.c|pyg|l'] >>
\end{code}

There's nothing fancy in there, although there's some bugs I'm leaving in
on purpose to see if you've been paying attention.  You'll fix those later.

What we want to do is use the functions \ident{dlopen}, \ident{dlsym} and
\ident{dlclose} to work with the above functions.

\begin{code}{ex29.c}
<< d['code/ex29.c|pyg|l'] >>
\end{code}

I'll now break this down so you can see what's going on in this small bit
of useful code:

\begin{description}
\item[ex29.c:5] I'll use this function pointer definition later to call functions
    in the library.  This is nothing new, but make sure you understand what
    it's doing.
\item[ex29.c:17] After the usual setup for a small program, I use the \ident{dlopen}
    function to load up the library indicated by \ident{lib\_file}.  This function
    returns a handle that we use later and works a lot like opening a file.
\item[ex29.c:18] If there's an error, I do the usual check and exit, but notice at
    then end that I'm using \ident{dlerror} to find out what the library related
    error was.
\item[ex29.c:20] I use \ident{dlsym} to get a function out of the \ident{lib} 
    by it's \emph{string} name in \ident{func\_to\_run}.  This is the powerful
    part, since I'm dynamically getting a pointer to a function based on a
    string I got from the command line \ident{argv}.
\item[ex29.c:23] I then call the \ident{func} function that was returned, and 
    check its return value.
\item[ex29.c:26] Finally, I close the library up just like I would a file.  Usually
    you keep these open the whole time the program is running, so closing at
    the end isn't as useful, but I'm demonstrating it here.
\end{description}

\section{What You Should See}

Now that you know what this file does, here's a shell session of me building
the \file{libex29.so}, \program{ex29} and then working with it.  Follow
along so you learn how these things are built manually.

\begin{code}{Building And Using libex29.so}
<< d['code/ex29.sh-session|pyg|l'] >>
\end{code}

One thing that you may run into is that every OS, every version of every
OS, and every compiler on every version of every OS, seems to want to change
the way you build a shared library every other month that some new programmer
thinks it's wrong.  If the line I use to make the \file{libex29.so} file is
wrong, then let me know and I'll add some comments for other platforms.

\section{How To Break It}

Open \file{libex29.so} and edit it with an editor that can handle
binary files.  Change a couple bytes, then close it.  Try to see
if you can get the \ident{dlopen} function to load it even though
you've corrupted it.

\section{Extra Credit}

\begin{enumerate}
\item Were you paying attention to the bad code I have in the \file{libex29.c} functions?
    See how, even though I use a for-loop they still check for \verb|'\0'|
    endings?  Fix this so the functions always take a length for the
    string to work with inside the function.
\item Take the \file{c-skeleton} skeleton, and create a new project
    for this exercise.  Put the \file{libex29.c} file in the \file{src/}
    directory.  Change the Makefile so that it builds this as \file{build/libex29.so}.
\item Take the \file{ex29.c} file and put it in \file{tests/ex29\_tests.c} so
    that it runs as a unit test.  Make this all work, which means you have to
    change it so that it loads the \file{build/libex29.so} file and runs
    tests similar to what I did manually above.
\item Read the \program{man dlopen} documentation and read about all the
    related functions.  Try some of the other options to \ident{dlopen}
    beside \ident{RTLD\_NOW}.
\end{enumerate}

