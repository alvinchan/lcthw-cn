\chapter{Exercise 8: Sizes And Arrays}

In the last exercise you did math, but with a \verb|'\0'| (nul) character.
This may be odd coming from other languages, since they try to treat
"strings" and "byte arrays" as different beasts.  C however treats strings
as just arrays of bytes, and it's only the different printing functions
that know there's a difference.

Before I can really explain the significance of this, I have to
introduce a few more concepts: \ident{sizeof} and arrays.  Here's
the code we'll be talking about:

\begin{code}{ex8.c}
<< d['code/ex8.c|pyg|l'] >>
\end{code}

In this code we create a few arrays with different data types in them.  Because
arrays of data are so central to how C works, there's a huge number of ways to
create them.  For now, just use the syntax \verb|type name[] = {initializer};|
and we'll explore more.  What this syntax means is, "I want an array of
type that is initialized to {..}."  When C sees this it does the following:

\begin{enumerate}
\item Look at the type, in this first case it's \ident{int}.
\item Look at the \verb|[]| and see that there's no length given.
\item Look at the initializer, \verb|{10, 12, 13, 14, 20}| and 
    figure out that you want those 5 ints in your array.
\item Create a piece of memory in the computer, that can hold 
    5 integers one after another.
\item Take the name you want, \ident{areas} and assign it this
    location.
\end{enumerate}

In the case of \ident{areas} it's creating an array of 5 ints
that contain those numbers.  When it gets to \verb|char name[] = "Zed";|
it's doing the same thing, except it's creating an array of 3 chars and
assigning that to \ident{name}.  The final array we make is \ident{full\_name},
but we use the annoying syntax of spelling it out, one character at a time.
To C, \ident{name} and \ident{full\_name} are identical methods of creating 
a char array.

The rest of the file, we're using a keyword called \ident{sizeof} to
ask C how big things are in \emph{bytes}.  C is all about the size
and location of pieces of memory and what you do with them.  To help
you keep that straight, it gives you \ident{sizeof} so you can ask
how big something is before you work with it.

This is where stuff gets tricky, so first let's run this and then
explain further.

\section{What You Should See}

\begin{code}{ex8 output}
\begin{lstlisting}
<< d['code/ex8.out'] >>
\end{lstlisting}
\end{code}

Now you see the output of these different \ident{printf} calls
and start to get a glimpse of what C is doing.  Your output could
actually be totally different from mine, since your computer might
have different size integers.  I'll go through my output:

\begin{description}
\item [5] My computer thinks an \ident{int} is 4 bytes in size. Your computer
    might use a different size if it's a 32-bit vs. 64-bit.
\item [6] The \ident{areas} array has 5 integers in it, so it makes
    sense that my computer requires 20 bytes to store it.
\item [7] If we divide the size of \ident{areas} by size of an \ident{int}
    then we get 5 elements.  Looking at the code, this matches what 
    we put in the initializer.
\item [8] We then did an array access to get \verb|areas[0]| and 
    \verb|areas[1]| which means C is "zero indexed" like Python
    and Ruby.
\item [9-11] We repeat this for the \ident{name} array, but notice
    something odd about the size of the array?  It says it's 
    \emph{4} bytes long, but we only typed "Zed" for 3 characters.
    Where's the 4th one coming from?
\item [12-13] We do the same thing with \ident{full\_name} and 
    notice it gets this correct.
\item [13] Finally we just print out the \ident{name} and \ident{full\_name}
    to prove that they actually are "strings" according to printf.
\end{description}

Make sure you can go through and see how these output lines match
what was created.  We'll be building on this and exploring more
about arrays and storage next.

\section{How To Break It}

Breaking this program is fairly easy.  Try some of these:

\begin{enumerate}
\item Get rid of the \verb|'\0'| at the end of \ident{full\_name}
    and re-run it.  Run it under Valgrind too.  Now, move the definition
    of \ident{full\_name} to the top of \ident{main} before \ident{areas}.
    Try running it under Valgrind a few times and see if you get some 
    new errors.  In some cases, you might still get lucky and not catch
    any errors.
\item Change it so that instead of \verb|areas[0]| you try to
    print \verb|areas[10]| and see what Valgrind thinks of that.
\item Try other versions of these, doing it to \ident{name} and
    \ident{full\_name} too.
\end{enumerate}

\section{Extra Credit}

\begin{enumerate}
\item Try assigning to elements in the \ident{areas} array with \verb|areas[0] = 100;| and similar.
\item Try assigning to elements of \ident{name} and \ident{full\_name}.
\item Try setting one element of \ident{areas} to a character from \ident{name}.
\item Go search online for the different sizes used for integers on different
    CPUs.
\end{enumerate}

