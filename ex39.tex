\chapter{Exercise 39: String Algorithms}

In this exercise I'm going to show you one of the supposedly faster string
search algorithms, and compare it to the one that exists in \file{bstrlib.c}
call \ident{binstr}.  The documentation for \ident{binstr} says that it uses a
simple "brute force" string search to find the first instance.  The one I'll
implement will use the Boyer-Moore-Horspool (BMH) algorithm, which is supposed
to be faster if you analyze the theoretical time.  You'll see that, assuming my
implementation isn't flawed, that the practical time for BMH is much worse than
the simple brute force of \ident{binstr}.

The point of this exercise isn't really to explain the algorithm because it's
simple enough for you to go to
\href{http://en.wikipedia.org/wiki/Boyer\%E2\%80\%93Moore\%E2\%80\%93Horspool\_algorithm}{the
Boyer-Moore-Horspool Wikipedia page} and read it.  The gist of this algorithm
is that it calculates a "skip characters list" as a first operation, then it
uses this list to quickly scan through the string.  It is supposed to be faster
than brute force, so let's get the code into the right files and see.

First, I have the header:

\begin{code}{src/lcthw/string\_algos.h}
<< d['code/liblcthw/src/lcthw/string_algos.h|pyg|l'] >>
\end{code}

In order to see the effects of this "skip characters list" I'm going to make
two versions of the BMH algorithm:

\begin{description}
\item[String\_find] Simply find the first instance of one string in another,
    doing the entire algorithm in one shot.
\item[StringScanner\_scan] Uses a \ident{StringScanner} state structure to
    separate the skip list build from the actual find.  This will let me
    see what impact that has on performance.  This model also has the advantage
    that I can incrementally scan for one string in another and find all 
    instances quickly.
\end{description}

Once you have that, here's the implementation:

\begin{code}{src/lcthw/string\_algos.c}
<< d['code/liblcthw/src/lcthw/string_algos.c|pyg|l'] >>
\end{code}

The entire algorithm is in two \ident{static inline} functions called
\ident{String\_setup\_skip\_chars} and \ident{String\_base\_search}.  These are
then used in the other functions to actually implement the searching styles I
want.  Study these first two functions and compare them to the Wikipedia
description so you know what's going on.

The \ident{String\_find} then just uses these two functions to do a find and
return the position found.  It's very simple and I'll use it to see how this
"build skip chars" phase impacts real practical performance.  Keep in mind that
you could maybe make this faster, but I'm teaching you how to confirm
theoretical speed after you implement an algorithm.

The \ident{StringScanner\_scan} function is then following the common pattern I
use of "create, scan, destroy" and is used to incrementally scan a string for
another string.  You'll see how this is used when I show you the unit test that
will test this out.

Finally, I have the unit test that first confirms this is all working, then
runs simple performance tests for all three finding algorithms.

\begin{code}{tests/string\_algos\_tests.c}
<< d['code/liblcthw/tests/string_algos_tests.c|pyg|l'] >>
\end{code}

There's nothing amazing in this unit test, it just runs each of the different
functions in loops that last long enough to get a few seconds of sampling.  The
first test (\ident{test\_find\_and\_scan}) just confirms that what I've written
works, because there's no point in testing the speed of something that doesn't
work.  Then the next three functions run a large number of searches using each
of the three functions.

The trick to notice is that I grab the starting time in \ident{start}, and then
I loop until at least \ident{TEST\_TIME} seconds have passed.  This makes sure
that I get enough samples to work with in comparing the three.  I'll then run
this test with different \ident{TEST\_TIME} settings and analyze the results.

\section{What You Should See}

When I run this test on my laptop, I get number that look like this:

\begin{code}{2 Second Test Run}
<< d['code/ex39.1.sh-session|pyg|l'] >>
\end{code}

I look at this and I sort of want to do more than 2 seconds of each run, and I
want to run this many times then use R to check it out like I did before.
Here's what I get for 10 samples of about 10 seconds each:

\begin{code}{10 Runs At 10 Seconds, Operations / Second}
\begin{Verbatim}
<< d['code/ex39.time.txt'] >>
\end{Verbatim}
\end{code}

The way I got this is with a little bit of shell help and then editing the
output:

\begin{code}{Getting Timing Logs}
<< d['code/ex39.2.sh-session|pyg|l'] >>
\end{code}

Right away you can see that the scanning system beats the pants off both of the
others, but I'll open this in R and confirm the results:

\begin{code}{R Summary Of Operations/Second}
<< d['code/ex39.3.sh-session|pyg|l'] >>
\end{code}

To understand why I'm getting the summary statistics I have to explain some
statistics for you.  What I'm looking for in these numbers can be said simply
to be, "Are these three functions (scan, find, bsinter) actually different?"  I
know that each time I run my tester function I get slightly different numbers,
and that those numbers can cover a certain range.  You see here that the
1st and 3rd quarters do that for each sample.

What I look at first is the mean and I want to see if each sample's mean is
different from the others.  I can see that, and clearly the \ident{scan} beats
\ident{binstr} which also beats \ident{find}.  However, I have a problem, if I
use just the mean, there's a \emph{chance} that the \ident{ranges} of each
sample might overlap.

What if I have means that are different, but the 1st and 3rd quarters overlap?
In that case I could say that there's a chance that if I ran the samples again
the means might not be different.  The more overlap I have in the ranges the
higher probability that my two samples (and my two functions) are \emph{not}
actually different.  Any difference I'm seeing in the two (in this case three)
is just random chance.

Statistics has many tools to solve this problem, but in our case I can just
look at the 1st and 3rd quarters as well as the mean for all three samples.  If
the means are different and the quarters are way off never possibly
overlapping, then it's alright to say they are different.

In my three samples I can say that \ident{scan}, \ident{find} and
\ident{binstr} are different, don't overlap in range, and that I can trust the
sample (for the most part).


\section{Analyzing The Results}

Looking at the results I can see that \ident{String\_find} is much slower than
the other two.  In fact, so slow I'd think there's something wrong with how I
implemented it.  However when I compare it with \ident{StringScanner\_scan} I
can see that it's the part that builds the skip list that is most likely
costing the time.  Not only is \ident{find} slower, it's also doing \emph{less}
than \ident{scan} because it's just finding the first string while \ident{scan}
finds all of them.

I can also see that \ident{scan} beats \ident{binstr} as well by quite a large
margin.  Again I can say that not only does \ident{scan} do more than both of
these, but it's also much faster.

There's a few caveats with this analysis:

\begin{enumerate}
\item I may have messed up this implementation or the test. At this
    point I would go research all the possible ways to do a BMH algorithm
    and try to improve it.  I would also confirm that I'm doing the test
    right.
\item If you alter the time the test runs, you get different results.
    There is a "warm up" period I'm not investigating.
\item The \ident{test\_scan\_performance} unit test isn't quite the
    same as the others, but it is doing more than the other tests so
    it's probably alright.
\item I'm only doing the test by searching for one string in another.
    I could randomize the strings to find to remove their position
    and length as a confounding factor.
\item Maybe \ident{binstr} is implemented better than "simple" brute force.
\item I could be running these in an unfortunate order and maybe randomizing
    which test runs first will give better results.
\end{enumerate}

One thing to gather from this is you need to confirm real performance even if
you implement an algorithm "correctly".  In this case the claim is that the BMH
algorithm should have beaten the \ident{binstr} algorithm, but a simple test
proved it didn't.  Had I not done this I would have been using an inferior
algorithm implementation without knowing it.  With these metrics I can start to
tune my implementation, or simply scrap it and find another one.

\section{Extra Credit}

\begin{enumerate}
\item See if you can make the \ident{Scan\_find} faster.  Why is my implementation
    here slow?
\item Try some different scan times and see if you get different numbers.
    What impact does the length of time that you run the test have on
    the \ident{scan} times?  What can you say about that result?
\item Alter the unit test so that it runs each function for a short burst
    in the beginning to clear out any "warm up" period, then start the
    timing portion.  Does that change the dependence on the length of time
    the test runs and how many operations / second are possible?
\item Make the unit test randomize the strings to find and then measure
    the performance you get.  One way to do this is use the \ident{bsplit}
    function from \file{bstrlib.h} to split the \ident{IN\_STR} on 
    spaces.  Then use the \ident{bstrList} struct you get to access each
    string it returns.  This will also teach you how to use \ident{bstrList}
    operations for string processing.
\item Try some runs with the tests in different orders and see if you get different
    results.
\end{enumerate}

