\chapter{Exercise 46: Ternary Search Tree}

The final data structure I'll show you is call the \emph{TSTree} and it's
similar to the \ident{BSTree} except it has three branches \ident{low},
\ident{equal}, and \ident{high}.  It's primarily used to just like
\ident{BSTree} and \ident{Hashmap} to store key/value data, but it is
keyed off of the individual characters in the keys.  This gives the
\ident{TSTree} some abilities that neither \ident{BSTree} or \ident{Hashmap}
have.

How a \ident{TSTree} works is every key is a string, and it's inserted by
walking and building a tree based on the equality of the characters in
the string.  Start at the root, look at the character for that node, and
if lower, equal to, or higher than that then go in that direction.  You can
see this in the header file:

\begin{code}{src/lcthw/tstree.h}
<< d['code/liblcthw/src/lcthw/tstree.h|pyg|l'] >>
\end{code}

The \ident{TSTree} has the following elements:

\begin{description}
\item[splitchar] The character at this point in the tree.
\item[low] The branch that is lower than \ident{splitchar}.
\item[equal] The branch that is equal to \ident{splitchar}.
\item[high] The branch that is higher than \ident{splitchar}.
\item[value] The value set for a string at that point with that \ident{splitchar}.
\end{description}

You can see this implementation has the following operations:

\begin{description}
\item[search] Typical "find a value for this \ident{key}" operation.
\item[search\_prefix] Finds the first value that has this as a prefix of its key. 
    This is the an operation that you can't easily do in 
    a \ident{BSTree} or \ident{Hashmap}.
\item[insert] Breaks the \ident{key} down by each character and inserts it 
    into the tree.
\item[traverse] Walks the tree allowing you to collect or analyze all the
    keys and values it contains.
\end{description} 

The only thing missing is a \ident{TSTree\_delete}, and that's because 
it is a horribly expensive operation, even more than \ident{BSTree\_delete}
was.  When I use \ident{TSTree} structures I treat them as constant data
that I plan on traversing many times and not removing anything from them.
They are very fast for this, but are not good if you need to insert and
delete quickly.  For that I use \ident{Hashmap} since it beats both
\ident{BSTree} and \ident{TSTree}.

The implementation for the \ident{TSTree} is actually simple, but it might
be hard to follow at first.  I'll break it down after you enter it in:

\begin{code}{src/lcthw/tstree.c}
<< d['code/liblcthw/src/lcthw/tstree.c|pyg|l'] >>
\end{code}

For \ident{TSTree\_insert} I'm using the same pattern for recursive structures
where I have a small function that calls the real recursive function.  I'm not
doing any additional check here but you should add the usual defensive programming
to it.  One thing to keep in mind is it is using a slightly different
design where you don't have a separate \ident{TSTree\_create} function, and
instead if you pass it a \ident{NULL} for the \ident{node} then it will create
it, and returns the final value.

That means I need to break down \ident{TSTree\_insert\_base} for you
to understand the insert operation:

\begin{description}
\item[tstree.c:10-18] As I mentioned, if I'm given a \ident{NULL} then I need to make
    this node and assign the \ident{*key} (current char) to it. This is used
    to build the tree as we insert keys.
\item[tstree.c:20-21] If the \ident{*key} < this then recurse, but go to the \ident{low}
    branch.
\item[tstree.c:22] This \ident{splitchar} is equal, so I want to go to deal with equality.
    This will happen if we just created this node, so we'll be building the
    tree at this point.
\item[tstree.c:23-24] There's still characters to handle, so recurse down the \ident{equal}
    branch, but go to the next \ident{*key} char.
\item[tstree.c:26-27] This is the last char, so I set the value and that's it.  I have
    an \ident{assert} here in case of a duplicate.
\item[tstree.c:29-30] The last condition is that this \ident{*key} is greater than 
    \ident{splitchar} so I need to recurse down the \ident{high} branch.
\end{description}

The key to some of the properties of this data structure is the fact that I'm
only incrementing the character I analyze when a \ident{splitchar} is equal.
The other two conditions I just walk the tree until I hit an equal character to
recurse into next.  What this does is it makes it very fast to \emph{not}
find a key.  I can get a bad key, and simply walk a few \ident{high} and \ident{low}
nodes until I hit a dead end to know that this key doesn't exist.  I don't need
to process every character of the key, or every node of the tree.

Once you understand that then move onto analyzing how \ident{TSTree\_search}
works:

\begin{description}
\item[tstree.c:46] I don't need to process the tree recursively in the \ident{TSTree},
    I can just use a while loop and a \ident{node} for where I am currently.
\item[tstree.c:47-48] If the current char is less than the node \ident{splitchar}, then go low.
\item[tstree.c:49-51] If it's equal, then increment \ident{i} and go equal as long as it's
    not the last character.  That's why the \verb|if(i < len)| is there, so that
    I don't go too far past the final \ident{value}.
\item[tstree.c:52-53] Otherwise I go \ident{high} since the char is greater.
\item[tstree.c:57-61] If after the loop I have a node, then return its \ident{value},
    otherwise return \ident{NULL}.
\end{description}

This isn't too difficult to understand, and you can then see that it's almost
exacty the same algorithm for the \ident{TSTree\_search\_prefix} function. 
The only difference is I'm trying to not find an exact match, but the longest
prefix I can.  To do that I keep track of the \ident{last} node that was equal,
and then after the search loop, walk that node until I find a \ident{value}.

Looking at \ident{TSTree\_search\_prefix} you can start to see the second advantage
a \ident{TSTree} has over the \ident{BSTree} and \ident{Hashmap} for finding
strings.  Given any key of X length, you can find any key in X moves.  You can
also find the first prefix in X moves, plus N more depending on how big the matching
key is.  If the biggest key in the tree is 10 characters long, then you can find
any prefix in that key in 10 moves.  More importantly, you can do all of this
by only comparing each character of the key \emph{once}.

In comparison, to do the same with a \ident{BSTree} you would have to check the
prefixes of each character in every possibly matching node in the \ident{BSTree}
against the characters in the prefix.  It's the same for finding keys, or seeing
if a key doesn't exist.  You have to compare each character against most of the
characters in the \ident{BSTree} to find or not find a match.

A \ident{Hashamp} is even worse for finding prefixes since you can't hash just
the prefix.  You basically can't do this efficiently in a \ident{Hashmap}
unless the data is something you can parse like a URL.  Even then that usually
requires whole trees of \ident{Hashmaps}.

The last two functions should be easy for you to analyze as they are the typical
traversing and destroying operations you've seen already for other data structures.

Finally, I have a simple unit test for the whole thing to make sure it works
right:

\begin{code}{tests/tstree\_tests.c}
<< d['code/liblcthw/tests/tstree_tests.c|pyg|l'] >>
\end{code}

\section{Advantages And Disadvantages}

There's other interesting practical things you can do with a \ident{TSTree}:

\begin{enumerate}
\item In addition to finding prefixes, you can reverse all the keys you insert,
    and then find by \emph{suffix}.  I use this to lookup host names, since
    I want to find \verb|*.learncodethehardway.com| so if I go backwards
    I can match them quickly.
\item You can do "approximate" matching, where you gather all the nodes that
    have most of the same characters as the key, or using other algorithms
    for what's a close match.
\item You can find all the keys that have a part in the middle.
\end{enumerate}

I've already talked about some of the things \ident{TSTrees} can do, but they
aren't the best data structure all the time.  The disadvantages of the \ident{TSTree}
are:

\begin{enumerate}
\item As I mentioned, deleting from them is murder.  They are better for 
    data that needs to be looked up fast and you rarely remove from.  If you
    need to delete then simply disable the \ident{value} and then periodically
    rebuild the tree when it gets too big.
\item It uses a ton of memory compared to \ident{BSTree} and \ident{Hashmaps}
    for the same key space.  Think about it, it's using a full node for
    each character in every key.  It might do better for smaller keys, but if you
    put a lot in a \ident{TSTree} it will get huge.
\item They also do not work well with large keys, but "large" is subjective
    so as usual test first.  If you're trying to store 10k character sized keys then use a \ident{Hashmap}.
\end{enumerate}



\section{How To Improve It}

As usual, go through and improve this by adding the defensive preconditions,
asserts, and checks to each function.  There's some other possible 
improvements, but you don't necessarily have to implement all of these:

\begin{enumerate}
\item You could allow duplicates by using a \ident{DArray} instead of the 
    \ident{value}.
\item As I mentioned deleting is hard, but you could simulate it by setting
    the values to \ident{NULL} so they are effectively gone.
\item There are no ways to collect all the possible matching values.  I'll have
    you implement that in an extra credit.
\item There are other algorithms that are more complex but have slightly
    better properties.  Take a look at Suffix Array, Suffix Tree, and 
    Radix Tree structures.
\end{enumerate}

\section{Extra Credit}

\begin{enumerate}
\item Implement a \ident{TSTree\_collect} that returns a \ident{DArray} containing
    all the keys that match the given prefix.
\item Implement \ident{TSTree\_search\_suffix} and a \ident{TSTree\_insert\_suffix}
    so you can do suffix searches and inserts.
\item Use \program{valgrind} to see how much memory this structure uses to store
    data compared to the \ident{BSTree} and \ident{Hashmap}.
\end{enumerate}

