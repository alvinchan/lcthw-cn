\chapter{Exercise 43: A Simple Statistics Engine}

This is a simple algorithm I use for collecting summary statistics "online", 
or without storing all of the samples.  I use this in any software that needs
to keep some statistics such as mean, standard deviation, and sum, but where
I can't store all the samples needed.  Instead I can just store the rolling
results of the calculations which is only 5 numbers.

\section{Rolling Standard Deviation And Mean}

The first thing you need is a sequence of samples.  This can be anything
from time to complete a task, numbers of times someone accesses something,
or even star ratings on a website.  Doesn't really matter what, just so long
as you have a stream of numbers and you want to know the following summary
statistics about them:

\begin{description}
\item[sum] This is the total of all the numbers added together.
\item[sum squared (sumsq)] This is the sum of the square of each number.
\item[count (n)] This is the number samples you've taken.
\item[min] This is the smallest sample you've seen.
\item[max] This is the largest sample you've seen.
\item[mean] This is the most likely middle number.  It's not actually the middle,
    since that's the median, but it's an accepted approximation for it.
\item[stddev] Calculated using $sqrt(sumsq - (sum * mean)) / (n - 1) ))$ where \ident{sqrt} is the square root function in the \file{math.h} header.
\end{description}

I will confirm this calculation works using R since I know R gets 
these right:

\begin{code}{Standard Deviation in R}
<< d['code/ex43.1.sh-session|pyg|l'] >>
\end{code}

You don't need to know R, just follow along while I explain how I'm breaking this
down to check my math:

\begin{description}
\item[lines 1-4] I use the function \ident{runif} to get a "random uniform" distribution
    of numbers, then print them out.  I'll use these in the unit test later.
\item[lines 5-7] Here's the summary, so you can see the values that R calculates for these.
\item[lines 8-9] This is the \ident{stddev} using the \ident{sd} function.
\item[lines 10-11] Now I begin to build this calculation manually, first by getting the
    \ident{sum}.
\item[lines 12-13] Next piece of the \ident{stdev} formula is the \ident{sumsq}, which I
    can get with \verb|sum(s * s)| which tells R to multiple the whole \ident{s}
    list by itself and then \ident{sum} those.\footnote{The power of R is being able to 
    do math on entire data structures like this.}
\item[lines 14-15] Looking at the formula, I then need the \ident{sum} multiplied by \ident{mean}, so I do \verb|sum(s) * mean(s)|.
\item[lines 16-17] I then combine the \ident{sumsq} with this to get \verb|sum(s * s) - sum(s) * mean(s)|.
\item[lines 18-19] That needs to be divided by $n-1$, so I do \verb|(sum(s * s) - sum(s) * mean(s)) / (length(s) - 1)|.
\item[lines 20-21] Finally, I \ident{sqrt} that and I get 3.547868 which matches
    the number R gave me for \ident{sd} above.
\end{description}

\section{Implemention}

That's how you calculate the \ident{stddev}, so now I can make some simple code
to implement this calculation.

\begin{code}{src/lcthw/stats.h}
<< d['code/liblcthw/src/lcthw/stats.h|pyg|l'] >>
\end{code}

Here you can see I've put the calculations I need to store in a \ident{struct}
and then I have functions for sampling and getting the numbers.  Implementing
this is then just an exercise in converting the math:

\begin{code}{src/lcthw/stats.c}
<< d['code/liblcthw/src/lcthw/stats.c|pyg|l'] >>
\end{code}

Here's what each function in \file{stats.c} does:

\begin{description}
\item[Stats\_recreate] I'll want to load these numbers from some kind of 
    database, and this function let's me recreate a \ident{Stats} struct.
\item[Stats\_create] Simply called \ident{Stats\_recreate} with all 0 values.
\item[Stats\_mean] Using the \ident{sum} and \ident{n} it gives the mean.
\item[Stats\_stddev] Implements the formula I worked out, with the only
    difference being that I calculate the mean with \verb|st->sum / st->n|
    in this formula instead of calling \ident{Stats\_mean}.
\item[Stats\_sample] This does the work of maintaining the numbers in the \ident{Stats} struct.  When you give it the first value it sees that \ident{n} is 0 and
    sets \ident{min} and \ident{max} accordingly.  Every call after that keeps
    increasing \ident{sum}, \ident{sumsq}, and \ident{n}.  It then figures out
    if this new sample is a new \ident{min} or \ident{max}.
\item[Stats\_dump] Simple debug function that dumps the stats so you can 
    view them.
\end{description}

The last thing I need to do is confirm that this math is correct.  I'm going
to use my numbers and calculations from my R session to create a unit test
that confirms I'm getting the right results.

\begin{code}{tests/stats\_tests.c}
<< d['code/liblcthw/tests/stats_tests.c|pyg|l'] >>
\end{code}

There's nothing new in this unit test, except maybe the \ident{EQ} macro.
I felt lazy and didn't want to look up the standard way to tell if two
\ident{double} values are close, so I made this macro.  The problem with
\ident{double} is that equality assumes totally equal, but I'm using two
different systems with slightly different rounding errors.  The solution 
is to say I want the numbers to be "equal to X decimal places".

I do this with \ident{EQ} by raising the number to a power of 10, then using
the \ident{round} function to get an integer.  This is a simple way to round
to N decimal places and compare the results as an integer.  I'm sure there's
a billion other ways to do the same thing, but this works for now.

The expected results are then in a \ident{Stats} \ident{struct} and then
I simply make sure that the number I get is close to the number R gave me.

\section{How To Use It}

You can use the standard deviation and mean to determine if a new sample
is "interesting", or you can use this to collect statistics on statistics.  The 
first one is easy for people to understand so I'll explain that quickly
using an example for login times.

Imagine you're tracking how long users spend on a server and you're using
stats to analyze it.  Every time someone logs in, you keep track of 
how long they are there, then you call \ident{Stats\_sample}.  I'm looking
for people are a on "too long" and also people who seem to be on "too quickly".

Instead of setting specific levels, what I'd do is compare how long someone
is on with the \ident{mean (plus or minus) 2 * stddev} range.  I get the
\ident{mean} and \ident{2 * stddev}, and consider login times to be "interesting"
if they are outside these two ranges.  Since I'm keeping these statistics
using a rolling algorithm this is a very fast calculation and I can then have
the software flag the users who are outside of this range.

This doesn't necessarily point out people who are behaving badly, but instead
it flags potential problems that you can review to see what's going on.  It's
also doing it based on the behavior of all the users, which avoids the problem
where you pick some arbitrary number that's not based on what's really happening.

The general rule you can get from this is that the \ident{mean (plus or minus) 2 * stddev} is an estimate of where 90\% of the values are expected to fall, and that
anything outside those ranges is interesting.

The second way to use these statistics is to go meta and calculate them for
other \ident{Stats} calculations.  You basically do your \ident{Stats\_sample}
like normal, but then you run \ident{Stats\_sample} on the \ident{min}, 
\ident{max}, \ident{n}, \ident{mean}, and \ident{stddev} on that sample.
This gives a two-level measurement, and let's you compare samples of samples.

Confusing right?  I'll continue my example above and add that you have 
100 servers that each hold a different application.  You are already
tracking user's login times for each application server, but you want to
compare all 100 applications and flag any users that are logging in "too much"
on all of them.  Easiest way to do that is each time someone logs in, calculate
the new login stats, then add \emph{that} \ident{Stats structs} elements to
a second \ident{Stat}.

What you end up with is a series of stats that can be named like this:

\begin{description}
\item[mean of means]  This is a full \ident{Stats struct} that gives you \ident{mean} and \ident{stddev} of the means of all the servers.  Any server or user who is outside of this is work looking at on a global level.
\item[mean of stddevs] Another \ident{Stats struct} that produces the statistics
    of how \emph{all} of the servers range.  You can then analyze each server and
    see if any of them have unusually wide ranging numbers by comparing their
    \ident{stddev} to this \ident{mean of stddevs} statistic.
\end{description}

You could do them all, but these are the most useful.  If you wanted to then
monitor servers for erratic login times you'd do this:

\begin{enumerate}
\item User John logs into and out of server A.  Grab server A's stats, update them.
\item Grab the \ident{mean of means} stats, and take A's mean and add it as a sample.
    I'll call this \ident{m\_of\_m}.
\item Grab the \ident{mean of stddevs} stats, and add A's stddev to it as a sample.
    I'll call this \ident{m\_of\_s}.
\item If A's \ident{mean} is outside of \ident{m\_of\_m.mean + 2 * m\_of\_m.stddev}
    then flag it as possibly having a problem.
\item If A's \ident{stddev} is outside of \ident{m\_of\_s.mean + 2 * m\_of\_s.stddev}
    then flag it as possible behaving too erratically.
\item Finally, if John's login time is outside of A's range, or A's \ident{m\_of\_m}
    range, then flag it as interesting.
\end{enumerate}

Using this "mean of means" and "mean of stddevs" calculation you can do efficient
tracking of many metrics with a minimal amount of processing and storage.


\section{Extra Credit}

\begin{enumerate}
\item Convert the \ident{Stats\_stddev} and \ident{Stats\_mean} to \ident{static inline} functions in the \file{stats.h} file instead of in the \file{stats.c} file.
\item Use this code to write a performance test of the \file{string\_algos\_test.c}.
    Make it optional and have it run the base test as a series of samples then report
    the results.
\item Write a version of this in another programming language you know.  Confirm that this
    version is correct based on what I have here.
\item Write a little program that can take a file full of numbers and spit these statistics 
    out for them.
\item Make the program accept a table of data that has headers on one line, then all
    the other numbers on lines after it separated by any number of spaces.  Your program
        should then print out these stats for each column by the header name.
\end{enumerate}


