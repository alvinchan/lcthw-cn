\chapter{Exercise 34: Dynamic Array}

This is an array that grows on its own and has most of the same features as a
linked list.  It will usually take up less space, run faster, and has other
beneficial properties.  This exercise will cover a few of the disadvantages
like very slow removal from the front, with a solution (just do it at the end).

A dynamic array is simply an array of \ident{void **} pointers that is pre-allocated
in one shot and that point at the data.  In the linked list you had a full struct
that stored the \ident{void *value} pointer, but in a dynamic array there's just
a single array with all of them.  This means you don't need any other pointers
for next and previous records since you can just index into it directly.

To start, I'll give you the header file you should type up for the implementation:

\begin{code}{src/lcthw/darray.h}
<< d['code/liblcthw/src/lcthw/darray.h|pyg|l'] >>
\end{code}

This header file is showing you a new technique where I put \ident{static inline}
functions right in the header.  These function definitions will work similar to
the \ident{\#define} macros you've been making, but they're cleaner and easier to
write.  If you need to create a block of code for a macro and you don't need
code generation, then use a \ident{static inline} function.

Compare this technique to the \ident{LIST\_FOREACH} that \emph{generates} a
proper for-loop for a list. This would be impossible to do with a \ident{static inline} function because it actually has to generate the inner block of code for
the loop.  The only way to do that is with a callback function, but that's not
as fast and is harder to use.

I'll then change things up and have you create the unit test for \ident{DArray}:

\begin{code}{tests/darray\_tests.c}
<< d['code/liblcthw/tests/darray_tests.c|pyg|l'] >>
\end{code}

This shows you how all of the operations are used, which then makes implementing
the \ident{DArray} much easier:

\begin{code}{src/lcthw/darray.c}
<< d['code/liblcthw/src/lcthw/darray.c|pyg|l'] >>
\end{code}

This shows you another way to tackle complex code.  Instead of diving right
into the \file{.c} implementation, look at the header file, then read the
unit test.  This gives you an "abstract to concrete" understanding how the
pieces work together and making it easier to remember.

\section{Advantages And Disadvantages}

A \ident{DArray} is better when you need to optimize these operations:

\begin{enumerate}
\item Iteration.  You can just use a basic for-loop and \ident{DArray\_count}
    with \ident{DArray\_get} and you're done.  No special macros needed, and
    it's faster because you aren't walking pointers.
\item Indexing.  You can use \ident{DArray\_get} and \ident{DArray\_set} to
    access any element at random, but with a \ident{List} you have to go
    through N elements to get to N+1.
\item Destroying.  You just free the struct and the \ident{contents} in 
    two operations.  A \ident{List} requires a series of \ident{free} calls
    and also walking every element.
\item Cloning. You can also clone it in just two operations (plus whatever
        it's storing) by copying the struct and \ident{contents}.  A list
        again requires walking the whole thing and copying every \ident{ListNode}
        plus its value.
\item Sorting. As you saw, \ident{List} is horrible if you need to keep the
    data sorted.  A \ident{DArray} opens up a whole class of great sorting
    algorithms because now you can access elements randomly.
\item Large Data. If you need to keep around a lot of data, then a \ident{DArray}
    wins since it's base \ident{contents} takes up less memory than the same
    number of \ident{ListNode} structs.
\end{enumerate}

The \ident{List} however wins on these operations:

\begin{enumerate}
\item Insert and remove on the front (what I called shift).  A \ident{DArray} 
    needs special treatment to be able to do this efficiently, and usually
    has to do some copying.
\item Splitting or joining.  A \ident{List} can just copy some pointers and
    it's done, but with a \ident{DArray} you have to do copying of the
    arrays involved.
\item Small Data. If you only need to store a few elements, then typically the
    storage will be less in a \ident{List} than a generic \ident{DArray} because
    the \ident{DArray} needs to expand the backing store to accommodate future
    inserts, but a \ident{List} only makes what it needs.
\end{enumerate}

With this, I prefer to use a \ident{DArray} for most of the things you see
other people use a \ident{List}.  I reserve using \ident{List} for any
data structure that requires small number of nodes that are inserted and
removed from either end.  I'll show you two similar data structures 
called a \ident{Stack} and \ident{Queue} where this is important.

\section{How To Improve It}

As usual, go through each function and operation and add the defensive programming
checks, pre-conditions, invariants, and anything else you can find to make the
implementation more bulletproof.

\section{Extra Credit}

\begin{enumerate}
\item Improve the unit tests to cover more of the operations and test that
    using a for-loop to iterate works.
\item Research what it would take to implement bubble sort and merge sort
    for DArray, but don't do it yet.  I'll be implementing DArray algorithms
    next and you'll do this then.
\item Write some performance tests for common operations and compare them
    to the same operations in \ident{List}.  You did some of this, but this
    time, write a unit test that repeatedly does the operation in question,
    then in the main runner do the timing.
\end{enumerate}


