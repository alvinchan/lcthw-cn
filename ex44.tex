\chapter{Exercise 44: Ring Buffer}

Ring buffers are incredibly useful when processing asynchronous IO.  They allow
one side to receive data in random intervals of random sizes, but feed cohesive
chunks to another side in set sizes or intervals.  They are a variant on the 
\ident{Queue} data structure but it focuses on blocks of bytes instead of
a list of pointers.  In this exercise I'm going to show you the 
\ident{RingBuffer} code, and then you have to make a full unit test
for it.

\begin{code}{src/lcthw/ringbuffer.h}
<< d['code/liblcthw/src/lcthw/ringbuffer.h|pyg|l'] >>
\end{code}

Looking at the data structure you see I have a \ident{buffer}, 
\ident{start} and \ident{end}.  A \ident{RingBuffer} does nothing
more than move the \ident{start} and \ident{end} around the buffer
so that it "loops" whenever it reaches the buffer's end.  Doing
this gives the illusion of an infinite read device in a small space.

Here's the implementation which is a much better explanation of how
this works:

\begin{code}{src/lcthw/ringbuffer.c}
<< d['code/liblcthw/src/lcthw/ringbuffer.c|pyg|l'] >>
\end{code}

This is all there is to a basic \ident{RingBuffer} implementation.  You
can read and write blocks of data to it.  You can ask how much is in it
and how much space it has.  There are some fancier ring buffers that use
tricks in the OS to create an imaginary infinite store, but those aren't
portable.


\section{What You Should See}

Here's my \file{ringbuffer\_tests} run:

\begin{code}{ringbuffer\_tests}
<< d['code/ex44.1.sh-session|pyg|l'] >>
\end{code}

You should have at least three tests that confirm all the basic operations,
and then see how much more you can test beyond what I've done.

\section{How To Improve It}

As usual you should go back and add the defensive programming checks
to this exercise.  Hopefully you've been doing this because the base
code in most of \file{liblcthw} doesn't check for common defensive
programming that I'm teaching you.  I leave this to you so that you get
used to improving code with these extra checks.

For example, in this ring buffer there's not a lot of checking that an
access will actually be inside the buffer.

If you read the \href{http://en.wikipedia.org/wiki/Ring_buffer}{Ring Buffer
Wikipedia page} you'll see the "Optimized POSIX implementation" that uses POSIX
specific calls to create an infinite space.  Study that as I'll have you 
try it in the extra credit.


\section{Extra Credit}

\begin{enumerate}
\item Create an alternative implementation of \ident{RingBuffer} that uses
    the POSIX trick but still works with the same (or nearly the same) unit test.
\item Add a performance comparison test to this unit test that compares the
    two versions by fuzzing them with random data and random read/write operations.
    Make sure that you setup this fuzzing so that the same operations are done
    to each so you can compare them between runs.
\item Use \program{callgrind} and \program{cachegrind} to compare the 
    performance of these two.
\end{enumerate}

