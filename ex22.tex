\chapter{Exercise 22: The Stack, Scope, And Globals}

The concept of "scope" seems to confuse quite a few people when they first
start programming.  Originally it came from the use of the system stack
(which we lightly covered earlier) and how it was used to store temporary
variables.  In this exercise, we'll learn about scope by learning about
how a stack data structure works, and then feeding that concept back in
to how modern C does scoping.

The real purpose of this exercise though is to learn where the hell things
live in C.  When someone doesn't grasp the concept of scope, it's almost
always a failure in understanding where variables are created, exist, and
die.  Once you know where things are, the concept of scope becomes easier.

This exercise will require three files:

\begin{description}
\item[ex22.h] A header file that sets up some external variables and some functions.
\item[ex22.c] Not your main like normal, but instead a source file that will become
    a object file \file{ex22.o} which will have some functions and variables in it
    defined from \file{ex22.h}.
\item[ex22\_main.c] The actual \func{main} that will include the other two and
    demonstrate what they contain as well as other scope concepts.
\end{description}

\subsection{ex22.h and ex22.c}

Your first step is to create your own header file named \file{ex22.h} which
defines the functions and "extern" variables you need:

\begin{code}{ex22.h}
<< d['code/ex22.h|pyg|l'] >>
\end{code}

The important thing to see is the use of \verb|extern int THE_SIZE|, which I'll
explain after you also create the matching \file{ex22.c}:

\begin{code}{ex22.c}
<< d['code/ex22.c|pyg|l'] >>
\end{code}

These two files introduce some new kinds of storage for variables:

\begin{description}
\item[extern] This keyword is a way to tell the compiler "the variable exists,
    but it's in another 'external' location".  Typically this means that one
    .c file is going to use a variable that's been defined in another .c file.
    In this case, we're saying \file{ex22.c} has a variable \ident{THE\_SIZE}
    that will be accessed from \file{ex22\_main.c}.
\item[static (file)] This keyword is kind of the inverse of \ident{extern} and says
    that the variable is only used in this .c file, and should not be available
    to other parts of the program.  Keep in mind that \ident{static} at the
    file level (as with \ident{THE\_AGE} here) is different than in other places.
\item[static (function)] If you declare a variable in a function \ident{static}, then
    that variable acts like a \ident{static} defined in the file, but it's only
    accessible from that function. It's a way of creating constant state for a
    function, but in reality it's \emph{rarely} used in modern C programming
    because they are hard to use with threads.
\end{description}

In these two files then, you have the following variables and functions
that you should understand:

\begin{description}
\item[THE\_SIZE] This is the variable you declared \ident{extern} that you'll
    play with from \file{ex22\_main.c}.
\item[get\_age and set\_age] These are taking the static variable \ident{THE\_AGE},
    but exposing it to other parts of the program through functions.  You couldn't
    access \ident{THE\_AGE} directly, but these functions can.
\item[update\_ratio] This takes a new \ident{ratio} value, and returns the old
    one.  It uses a function level static variable \ident{ratio} to keep track
    of what the ratio currently is.
\item[print\_size] Prints out what \file{ex22.c} thinks \ident{THE\_SIZE} is
    currently.
\end{description}

\subsection{ex22\_main.c}

Once you have that file written, you can then make the main function which
uses all of these and demonstrates some more scope conventions:

\begin{code}{ex22\_main.c}
<< d['code/ex22_main.c|pyg|l'] >>
\end{code}

I'll break this file down line-by-line, and as I do you should find each
variable I mention and where it lives.

\begin{description}
\item[ex22\_main.c:4] Making a \ident{const} which stands for constant and is an 
    alternative to using a \ident{define} to create a constant variable.
\item[ex22\_main.c:6] A simple function that demonstrates more scope issues in a function.
\item[ex22\_main.c:8] Prints out the value of \ident{count} as it is at the top of the function.
\item[ex22\_main.c:10] An \ident{if-statement} that starts a new \emph{scope block}, and then
    has another \ident{count} variable in it.  This version of \ident{count}
    is actually a whole new variable. It's kind of like the \ident{if-statement}
    started a new "mini function".
\item[ex22\_main.c:11] The \ident{count} that is local to this block is actually different
    from the one in the function's parameter list.  What what happens as we 
    continue.
\item[ex22\_main.c:13] Prints it out so you can see it's actually 100 here, not what was
    passed to \func{scope\_demo}.
\item[ex22\_main.c:16] Now for the freaky part.  You have \ident{count} in two places: the
    parameters to this function, and in the \ident{if-statement}.  The
    \ident{if-statement} created a new block, so the \ident{count} on line
    11 \emph{does not impact the parameter with the same name}.  This line
    prints it out and you'll see that it prints the value of the parameter, 
    not 100.
\item[ex22\_main.c:18-20] Then I set the parameter \ident{count} to 3000 and print that
    out, which will demonstrate that you can change function parameters
    and they don't impact the caller's version of the variable.
\end{description}

Make sure you trace through this function, but don't think that you understand
scope quite yet.  Just start to realize that if you make a variable inside a
block (as in \ident{if-statements} or \ident{while-loops}), then those variables
are \emph{new} variables that exist only in that block.  This is crucial to 
understand, and is also a \emph{source of many bugs}.  We'll address why
you shouldn't do this shortly.

The rest of the \file{ex22\_main.c} then demonstrates all of these by
manipulating and printing them out:

\begin{description}
\item[ex22\_main.c:26] Prints out the current values of \ident{MY\_NAME} and gets
    \ident{THE\_AGE} from \file{ex22.c} using the accessor function
    \func{get\_age}.
\item[ex22\_main.c:27-30] Uses \func{set\_age} in \file{ex22.c} to change \ident{THE\_AGE}
    and then print it out.
\item[ex22\_main.c:33-39] Then I do the same thing to \ident{THE\_SIZE} from \file{ex22.c},
    but this time I'm accessing it directly, and also demonstrating that it's
    actually changing in that file by printing it here and with \func{print\_size}.
\item[ex22\_main.c:42-44] Show how the static variable \ident{ratio} inside \func{update\_ratio}
    is maintained between function calls.
\item[ex22\_main.c:46-51] Finally running \func{scope\_demo} a few times so you can see
    the scope in action.  Big thing to notice is that the local \ident{count}
    variable remains unchanged.  You \emph{must} get that passing in a variable
    like this will not let you change it in the function.  To do that you need
    our old friend the pointer.  If you were to pass a pointer to this \ident{count},
    then the called function has the address of it and can change it.
\end{description}

That explains what's going on in all of these files, but you should trace
through them and make sure you know where everything is as you study it.

\section{What You Should See}

This time, instead of using your \file{Makefile} I want you to build these
two files manually so you can see how they are actually put together by
the compiler.  Here's what you should do and what you should see for output.

\begin{code}{ex22 output}
\begin{lstlisting}
<< d['code/ex22.out'] >>
\end{lstlisting}
\end{code}

Make sure you trace how each variable is changing and match it to the line
that gets output.  I'm using \func{log\_info} from the \file{dbg.h} macros
so you can get the exact line number where each variable is printed and
find it in the files for tracing.

\section{Scope, Stack, And Bugs}

If you've done this right you should now see many of the different ways
you can place variables in your C code.  You can use \ident{extern} or
access functions like \func{get\_age} to create globals.  You can make
new variables inside any blocks, and they'll retain their own values until
that block exits, leaving the outer variables alone.  You also can pass
a value to a function, and change the parameter but not change the caller's
version of it.

The most important thing to realize though is that all of this causes
bugs.  C's ability to place things in many places in your machine and then
let you access it in those places means you get confused easily about
where something lives.  If you don't where it lives then there's a chance
you'll not manage it properly.

With that in mind, here's some rules to follow when writing C code
so you avoid bugs related to the stack:

\begin{enumerate}
\item Do not "shadow" a variable like I've done here with \ident{count}
    in \func{scope\_demo}.  It leaves you open to subtle and hidden bugs
    where you \emph{think} you're changing a value and you actually aren't.
\item Avoid too many globals, especially if across multiple files.  If you have
    to then use accessor functions like I've done with \ident{get\_age}.  This
    doesn't apply to constants, since those are read-only.  I'm talking about
    variables like \ident{THE\_SIZE}.  If you want people to modify or set this,
    then make accessor functions.
\item When in doubt, put it on the heap.  Don't rely on the semantics of the
    stack or specialized locations and instead just create things with
    \ident{malloc}.
\item Don't use function static variables like I did in \func{update\_ratio}.
    They're rarely useful and end up being a huge pain when you need to make
    your code concurrent in threads.  They are also hard as hell to find compared
    to a well done global variable.
\item Avoid reusing function parameters as it's confusing whether you're 
    just reusing it or if you think you're changing the \emph{caller's}
    version of it.
\end{enumerate}

As with all things, these rules can be broken when it's practical.  In fact,
I guarantee you'll run into code that breaks all of these rules and is perfectly
fine.  The constraints of different platforms makes it necessary sometimes.

\section{How To Break It}

For this exercise, breaking the program involves trying to access or change
things you can't:

\begin{enumerate}
\item Try to directly access variables in \file{ex22.c} from \file{ex22\_main.c}
    that you think you can't.  For example, you can't get at \ident{ratio}
    inside \func{update\_ratio}? What if you had a pointer to it?
\item Ditch the \ident{extern} declaration in \file{ex22.h} to see what you
    get for errors or warnings.
\item Add \ident{static} or \ident{const} specifiers to different variables
    and then try to change them.
\end{enumerate}


\section{Extra Credit}

\begin{enumerate}
\item Research the concept of "pass by value" vs. "pass by reference".  Write an
    example of both.
\item Use pointers to gain access to things you shouldn't have access to.
\item Use valgrind to see what this kind of access looks like when you 
    do it wrong.
\item Write a recursive function that causes a stack overflow.  Don't know
    what a recursive function is?  Try calling \func{scope\_demo} at the
    bottom of \func{scope\_demo} itself so that it loops.
\item Rewrite the \file{Makefile} so that it can build this.
\end{enumerate}

