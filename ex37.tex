\chapter{Exercise 37: Hashmaps}

Hash Maps (Hashmaps, Hashes, or sometimes Dictionaries) are used frequently in
many dynamic programming for storing key/value data.  A Hashmap works by performing
a "hashing" calculation on the keys to produce an integer, then uses that
integer to find a bucket to get or set the value.  It is a very fast practical
data structure since it works on nearly any data and they are easy to implement.

Here's an example of using a Hashmap (aka dict) in Python:

\begin{code}{ex37.py}
<< d['code/ex37.py|pyg|l'] >>
\end{code}

Almost every modern language has something like this, so many people
end up writing code and never understand how this actually works.
By creating the \ident{Hashmap} data structure in C I'll show you how
this works.  I'll start with the header file so I can talk about the
data structure.

\begin{code}{src/lcthw/hashmap.h}
<< d['code/liblcthw/src/lcthw/hashmap.h|pyg|l'] >>
\end{code}

The structure consists of a \ident{Hashmap} that contains
any number of \ident{HashmapNode} structs.  Looking at \ident{Hashmap}
you can see that it is structured like this:

\begin{description}
\item[DArray *buckets] A dynamic array that will be set to a fixed size of
    100 buckets.  Each bucket will in turn contain a \ident{DArray} that will
    actually hold \ident{HashmapNode} pairs.
\item[Hashmap\_compare compare] This is a comparison function that the
    \ident{Hashmap} uses to actually find elements by their key.  It should
    work like all of the other compare functions, and defaults to using
    \ident{bstrcmp} so that keys are just bstrings.
\item[Hashmap\_hash hash] This is the hashing function and it's responsible
    for taking a key, processing its contents, and producing a single
    \ident{uint32\_t} index number.  You'll see the default one soon.
\end{description}

This almost tells you how the data is stored, but the \ident{buckets}
\ident{DArray} isn't created yet.  Just remember that it's kind of a
two level mapping:

\begin{enumerate}
\item There are 100 buckets that make up the first level, and things are 
    in these buckets based on their hash.
\item Each bucket is a \ident{DArray} that then contains \ident{HashmapNode}
    structs simply appended to the end as they're added.
\end{enumerate}

The \ident{HashmapNode} is then composed of these three elements:

\begin{description}
\item[void *key] The key for this key=value pair.
\item[void *value] The value.
\item[uint32\_t hash] The calculated hash, which makes finding this
    node quicker since we can just check the hash and skip any that
    don't match, only checking they key if it's equal.
\end{description}

The rest of the header file is nothing new, so now I can show you the
implementation \file{hashmap.c} file:

\begin{code}{src/lcthw/hashmap.c}
<< d['code/liblcthw/src/lcthw/hashmap.c|pyg|l'] >>
\end{code}

There's nothing very complicated in the implementation, but the
\ident{default\_hash} and \ident{Hashmap\_find\_bucket} functions will need
some explanation.  When you use \ident{Hashmap\_create} you can pass in any
compare and hash functions you want, but if you don't it uses the
\ident{default\_compare} and \ident{default\_hash} functions.

The first thing to look at is how \ident{default\_hash} does its thing.
This is a simple hash function called a "Jenkins hash" after Bob Jenkins.
I got if from the \href{http://en.wikipedia.org/wiki/Jenkins\_hash\_function}{Wikipedia page} for the algorithm.  It simply goes through each byte of the
key to hash (a bstring) and works the bits so that the end result is
a single \ident{uint32\_t}.  It does this with some adding and xor operations.

There are many different hash functions, all with different properties,
but once you have one you need a way to use it to find the right buckets.
The \ident{Hashmap\_find\_bucket} does it like this:

\begin{enumerate}
\item First it calls \verb|map->hash(key)| to get the hash for the key.
\item It then finds the bucket using \verb|hash % DEFAULT_NUMBER_OF_BUCKETS|, that
    way every hash will always find some bucket no matter how big it is.
\item It then gets the bucket, which is also a \ident{DArray}, and if it's 
    not there it will create it.  That depends on if the \ident{create}
    variable says too.
\item Once it has found the \ident{DArray} bucket for the right hash,
    it returns it, and also the \ident{hash\_out} variable is used to
    give the caller the hash that was found.
\end{enumerate}

All of the other functions then use \ident{Hashmap\_find\_bucket} to do 
their work:

\begin{enumerate}
\item Setting a key/value involves finding the bucket, then
making a \ident{HashmapNode}, and then adding it to the bucket.  
\item Getting a key involves finding the bucket, then finding the HashmapNode
that matches the \ident{hash} and \ident{key} you want.
\item Deleting an item again finds the bucket, finds where the requested node
is, and then removes it by swapping the last node into its place.
\end{enumerate}

The only other function that you should study is the \ident{Hashmap\_travers}.
This simply walks every bucket, and for any bucket that has possible
values, it calls the \ident{traverse\_cb} on each value.  This is how you
scan a whole \ident{Hashmap} for its values.

\subsection{The Unit Test}

Finally you have the unit test that is testing all of these operations:

\begin{code}{tests/hashmap\_tests.c}
<< d['code/liblcthw/tests/hashmap_tests.c|pyg|l'] >>
\end{code}

The only thing to learn about this unit test is that at the top I use
a feature of \ident{bstring} to create static strings to work with 
in the tests.  I use the \ident{tagbstring} and \ident{bsStatic} 
to create them on lines 7-13.

\section{How To Improve It}

This is a very simple implementation of \ident{Hashmap} as are most of
the other data structures in this book.  My goal isn't to give you
insanely great hyper speed well tuned data structures.  Usually those
are much too complicated to discuss and only distract you from the real
basic data structure at work.  My goal is to give you an understandable
starting point to then improve it or understand how they are implemented.

In this case, there's some things you can do with this implementation:

\begin{enumerate}
\item You can use a sort on each bucket so that they are always sorted.
    This increases your insert time, but decreases your find time because
    you can then use a binary search to find each node.  Right now
    it's looping through all of the nodes in a bucket just to find one.
\item You can dynamically size the number of buckets, or let the caller
    specify the number for each \ident{Hashmap} created.
\item You can use a better \ident{default\_hash}.  There are tons of them.
\item This (and nearly every \ident{Hashmap} is vulnerable to someone picking
    keys that will fill only one bucket, and then tricking your program
    into processing them.  This then makes your program run slower because
    it changes from processing a \ident{Hashmap} to effectively processing
    a single \ident{DArray}.  If you sort the nodes in the bucket this
    helps, but you can also use better hashing functions, and for 
    the really paranoid add a random salt so that keys can't be predicted.
\item You could have it delete buckets that are empty of nodes to save space,
    or put empty buckets into a cache so you save on creating and destroying
    them.
\item Right now it just adds elements even if they already exist.  Write an
    alternative set method that only adds it if it isn't set already.
\end{enumerate}

As usual you should go through each function and make it bullet proof.  The 
\ident{Hashmap} could also use a debug setting for doing an invariant check.


\section{Extra Credit}

\begin{enumerate}
\item Research the \ident{Hashmap} implementation of your favorite programming language to see what features they have.
\item Find out what the major disadvantages of a \ident{Hashmap} are and how to avoid them.  For example, they
    do not preserve order without special changes and they don't work when you need to find things based on parts
    of keys.
\item Write a unit test that demonstrates the defect of filling a \ident{Hashmap} with keys that land
    in the same bucket, then test how this impact performance.  A good way to do this is to just reduce
    the number of buckets to something stupid like 5.
\end{enumerate}

