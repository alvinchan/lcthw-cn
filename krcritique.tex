\chapter{Deconstructing \krc}

When I was a kid I read this awesome book called "The C Programming Language"
by the language's creators, Brian Kernighan and Dennis Ritchie.  This book 
taught me and many people of my generation, and a generation before, how to
write C code.  You talk to anyone, whether they know C or not, and they'll
say, "You can't beat \krc. It's the best C book."  It is an established
piece of programmer lore that is not soon to die.

I myself believed that until I started writing this book. You see, \krc is
actually riddled with bugs and bad style.  Its age is no excuse. These were
bugs when they wrote the first printing, and the 42nd printing.  I hadn't
actually realized just how bad most of the code was in this book and recommended
it to many people.  After reading through it for just an hour I decided that it
needs to be taken down from its pedestal and relegated to history rather than
vaunted as state of the art.

I believe it is time to lay this book to rest, but I want to use it as an
exercise for you in finding hacks, attacks, defects, and bugs by going through
\krc to break all the code.  That's right, you are going to destroy this
sacred cow for me, and you're going to have no problem doing it.  When you are
done doing this, you will have a finely honed eye for defect.  You will also
have an informed opinion of the book's actual quality, and will be able to make
your own decisions on how to use the knowledge it contains.

In this chapter we will use all the knowledge you've gained from
this book, and spend it reviewing the code in \krc.  What we will do is
take many pieces of code from the book, find all the bugs in it, and write
a unit test that \emph{exercises} the bugs.  We'll then run this test under
Valgrind to get statistics and data, and then we'll fix the bugs with a redesign.

This will obviously be a long chapter so I'm going to only do a handful of
these and then I'm going have you do the rest.  I'll provide a guide that is
each page, with the code on it, and hints to the bugs that it has.  Your job is
to then tear that piece of code apart and try to think like an attacker trying
to break the code.

\begin{aside}{Warning For The Fanboys}
As you read this, if you feel that I am being disrespectful to the
authors, then that's not my intent.  I respect the authors more than
anything you know and owe them a debt of gratitude for writing their book.
My criticisms here are both for educational purposes of teaching people
\emph{modern} C code, and to destroy the belief in their work as a 
item of worship that cannot be questioned.

However, if when you read this you have feelings of me insulting \emph{you}
then just stop reading.  You will gain nothing from this chapter but personal
grief because you've attached your identity to \krc and my criticisms will
only be taken personally.
\end{aside}

\section{An Overall Critique Of Correctness}

The primary problem \krc has is its view of "correctness" comes from
the first system it was used on: \emph{Unix}.  In the world of Unix software
programs have a particular set of properties:

\begin{enumerate}
\item Programs are started and then exit, making resource allocation easier.
\item Most functions are only called by other parts of the same program in set ways.
\item The inputs to the program are limited to "expert" restricted users.
\end{enumerate}

In the context of this 1970's computing style, \krc is actually correct.  As
long as only trusted people run complete cohesive programs that exit and clean
up all their resources then their code is fine.

Where \krc runs into problems is when the functions or code snippets are taken
\emph{out of the book} and used in other programs.  Once you take many of these
code snippets and try use them in some other program they fall apart.  They then
have blatant buffer overflows, bugs, and problems that a beginner will trip over.

Another problem is that software these days doesn't exit right away, but instead
it stays running for long periods of time because they're servers, desktop applications
and mobile applications.  The old style of "leaving the cleanup to the OS" doesn't
work in the modern world the way it did back in the day.

The final problem though is that no software lives in a vacuum anymore. Software
is now frequently attacked by people over network connections in an attempt to
gain special privilege or simple street cred.  The idea that "nobody will ever do
that" is dead, and actually that's probably the first thing somebody will do.

The best way to summarize the problem of \krc "correctness" is with an example
from English.  Imagine if you have the pair of sentences, "Jack and Jill went up the hill.
He fell down."  Well, from context clues you know that "He" means Jack.  However,
if you have that sentence on its own it's not clear who "He" is.  Now, if you put
that sentence at the end of another sentence you can get an unclear
pronoun reference: "Jack and Frank went up the hill. He fell down."  Which "He"
are we talking about in that sentence?

This is how the code in \krc works.  As long as that code is not used in other
programs without serious analysis of the entire software then it works.  The second
you take many of the functions out and try to use them in other systems they
fall apart.  And, what's the point of a book full of code you can't actually
use in your own programs?


\subsection{A First Demonstration Defect}

The following \ident{copy} function is found in the very first chapter and is
an example of copying two strings.  Here's a new source file to demonstrate the
defects in this function.

\begin{code}{exercise-1.9-1.c}
<< d['code/krc/1.9-1.c|pyg|l'] >>
\end{code}

In the above example, I'm doing something that is fairly common: switching from
using stack allocation to heap allocation with \ident{malloc}.  What happens
is, typically \ident{malloc} returns memory from the heap, and so the bytes
after it are not initialized.  Then you see me use a loop to accidentally
initialize it wrong.  This is a common defect, and one of the reasons we
avoided classic style C strings in this book.  You could also have this bug in
programs that read from files, sockets, or other external resources.  It is a
\emph{very} common bug, probably the most common in the world.

Before the switch to heap memory, this program probably ran just fine
because the stack allocated memory will probably have a \verb|'\0'| character
at the end on accident.  In fact, it would appear to run fine almost always
since it just runs and exits quickly.

What's the effect of running this new program with \ident{copy} used wrong?

\begin{code}{exercise-1.9-1.c Valgrind Failures}
\begin{lstlisting}
<< d['code/krc/1.9-1.out|dexy'] >>
\end{lstlisting}
\end{code}

As you've already learned, Valgrind will show you all of your sins in full
color.  In this case, a perfectly harmless seeming program has a ton of
"Invalid read of size 1".  If you kept running it you'd find other errors pop
up at random.

Now, in the context of the entire program in the original \krc example, this function
will work correctly.  However, the second this function is called with
\ident{longest} and \ident{line} uninitialized, initialized wrong, 
without a trailing \verb|'\0'| character, then you'll hit difficult to debug
errors.

This is the failing of the book.  While the code works in the book, it does
\emph{not} work in many other situations leading to difficult to spot defects,
and those are the worst kind of defects for a beginner (or expert).
Instead of code that only works in this delicate balance, we will strive to
create code that has a higher probability of working in any situation.

\subsection{Why copy() Fails}

Many people have looked at this copy function and thought that it is not
defective.  They claim that, as long as it's used correctly, it is correct.
One person even went so far as to say, "It's not defective, it's just unsafe."
Odd, since I'm sure this person wouldn't get into a car if the manufacturer
said, "Our car is not defective, it's just unsafe."

However, there is a way to formally prove that this function is defective by
enumerating the possible inputs and then seeing if any of them cause the
while loop to never terminate.  If we ignore the concept of buffer overflows
and assume that the while loop will run forever, then we can actually walk
through all possible inputs and see what happens.

What we'll do is have two strings, A and B, and figure out what copy() does
with them:

\begin{enumerate}
\item \verb|A = {'a','b','\0'}; B = {'a', 'b', '\0'};  copy(A,B);|
\item \verb|A = {'a','b'}; B = {'a', 'b', '\0'};  copy(A,B);|
\item \verb|A = {'a','b','\0'}; B = {'a', 'b'};  copy(A,B);|
\item \verb|A = {'a','b'}; B = {'a', 'b'};  copy(A,B);|
\end{enumerate}

This is all the basic permutations of strings that can be passed to the function based
on whether they are terminated with a \verb|'\0'| or not.  We can then go through each case
and determine if the while loop in copy() terminates:

\begin{enumerate}
\item while-loop finds first \verb|'\0'| in A, terminates.
\item while-loop does not find \verb|'\0'| in A, but finds second \verb|'\0'| in B, terminates.
\item while-loop finds first \verb|'\0'| in A, terminates.
\item while-loop does not find \verb|'\0'| in A, and does not find \verb|'\0'| B, \emph{runs forever}.
\end{enumerate}

This provides a formal proof that the function is defective because there is a possible
input that causes the while-loop to run forever.

Next, if we put back the practical limitations of memory and that we can overflow if there
is a missing terminator in one of the strings, then we find that permutations 2-4 have
an additional error.  In conditions 2-4 there is a probability of having a buffer overflow
that is either exploitable or can cause the system to crash.  This depends on the lengths
of the two strings, but this makes the function vulnerable in 3/4th of the possible 
inputs.

Let's now compare this to a copy function that knows the lengths of all the inputs:

\begin{enumerate}
\item \verb|A = {'a','b','\0'}; B = {'a', 'b', '\0'};  safercopy(2, A, 2, B);|
\item \verb|A = {'a','b'}; B = {'a', 'b', '\0'};  safercopy(2, A, 2, B);|
\item \verb|A = {'a','b','\0'}; B = {'a', 'b'};  safercopy(2, A, 2, B);|
\item \verb|A = {'a','b'}; B = {'a', 'b'};  safercopy(2, A, 2, B);|
\end{enumerate}

Also assume that the safercopy() function uses a for-loop that does not test
for a \verb|'\0'| only, but instead uses the given lengths to determine the amount 
to copy.  With that we can then do the same analysis:

\begin{enumerate}
\item for-loop processes 2 characters of A, terminates.
\item for-loop processes 2 characters of A, terminates.
\item for-loop processes 2 characters of A, terminates.
\item for-loop processes 2 characters of A, terminates.
\end{enumerate}

In every case the for-loop variant with string length given as arguments will terminate
no matter what.  To really test the for-loop variant we'd need to add some permutations
for differing lengths of strings A and B, but in every case we come to the for-loop
will always stop because it will only go through a fixed previously known finite number
of characters.

That means the for-loop will never loop forever, and as long as it handles all the
possible differing lengths of A and B, never overflow either side.  The only way to
break safercopy() is to lie about the lengths of the strings, but even then it will
\emph{still always terminate}.  The worst possible scenario for the safercopy() function
is that you are given an erroneous length for one of the strings and that string does
not have a \verb|'\0'| properly, so the function buffer overflows.

This shows exactly why the copy() function is defective, because it does not terminate
cleanly for most possible inputs, and is only reliable for one of the conditions (A and B both
terminated).  It also shows why a for-loop variant with a given fixed length for each
input is superior.

Finally, the significance of this is that I've effectively done a formal proof (well, mostly
formal) that shows what you should be doing to analyze code.  Each function has to stand on
its own and not have any defects such as while-loops that do not terminate.  In the above
discussion I've shown that the original \krc is defective, and fatally so since there is
no way to fix it given the inputs.  There's no way from just a pointer to ask if a 
string is properly formed since the only way to test that is to scan it, and scanning it
runs into this same problem.


\subsection{Stylistic Issues}

A more minor critique of the book is that the style is not only old, but just
error prone and annoyingly "clever".  Take the code you just saw again and look
at the \ident{while-loop} in \ident{copy}.  There's no reason to write this
loop this way, as the compiler can just as easily work with a \ident{for-loop}
and without the clever triple-equality trick.  The original code also
has a while-loop without braces, but an if-statement with braces, which
leads to even more confusion:

\begin{code}{Braces Are Free, Use Them}
<< d['code/krc/1.9-2.c|pyg|l'] >>
\end{code}

This code is \emph{incredibly} error prone because you can't easily tell
where the pair of if-statements and the while-loop are paired.  A quick
glance makes it seem like this while-loop will loop both if-statements,
but it doesn't.  In modern C code you would instead just use braces all
the time and avoid the confusion completely.

While the book could be forgiven for this because of its age, it has been
republished in this form \emph{42 times}, and it was updated for the ANSI
standard.  At some point in its history you'd think the authors or some
publisher ghostwriter could have been bothered to update the book's style.
However, this is the problem with sacred cows.  Once they become idols of
worship people are reluctant to question them or modify them.

In the rest of this chapter though we will be modernizing the code in \krc
to fit the style you've been learning throughout this book.  It will be
more verbose, but it will be clearer and less error prone because of
this slight increase in verbosity.


\section{Chapter 1 Examples}

Now we begin...
