\chapter{Exercise 24: Input, Output, Files}
\chapter{Exercise 24: 输入, 输出, 文件}

你已经用 \ident{printf} 来打印了, 这已经很棒了。但是你肯定想做的更多一些。 
在这一节,你将会用 \func{fscanf} 和 \func{fgets} 来构建一个包含某人信息的结构体。
在简单介绍过如何获得输入之后,你将会看到一个c语言拥有的处理IO的完整的函数列表。
你应该已经见过而且用过其中的一部分函数,所以这会是另一个唤醒记忆的练习。

\begin{code}{ex24.c}
<< d['code/ex24.c|pyg|l'] >>
\end{code}

这个程序表面看似简单,其实不然。这里面用到了一个函数 \func{fscanf},它是用来做文件扫描 "file scanf"。
 \func{scanf} 相关的一系列函数与 \func{printf} 一系列函数作用正相反。 
\func{printf} 按着指定格式输出数据, \func{scanf} 按着格式读取(或者扫描输入)。

There's nothing original in the beginning of the file, so here's what
the \func{main} is doing:

\begin{description}
\item[ex24.c:24-28] Set up some variables we'll need.
\item[ex24.c:30-32] Get your first name using the \func{fgets} function, which reads a
    string from the input (in this case \ident{stdin}) but makes sure it
    doesn't overflow the given buffer.
\item[ex24.c:34-36] Same thing for \ident{you.last\_name}, again using \func{fgets}.
\item[ex24.c:38-39] Uses \func{fscanf} to read an integer from \func{stdin} and put it
    into \ident{you.age}.  You can see that the same format string is used
    as \ident{printf} to print an integer.  You should also see that you have
    to give the \emph{address} of \ident{you.age} so that \ident{fscanf} has
    a pointer to it and can modify it.  This is a good example of using a
    pointer to a piece of data as an "out parameter".
\item[ex24.c:41-45] Print out all the options available for eye color, with a matching
    number that works with the \ident{EyeColor} enum above.
\item[ex24.c:47-50] Using \func{fscanf} again, get a number for the \ident{you.eyes}, 
    but make sure the input is valid.  This is important because someone can
    enter a value outside the \ident{EYE\_COLOR\_NAMES} array and cause a 
    segfault.
\item[ex24.c:52-53] Get how much you make as a \ident{float} for the \ident{you.income}.
\item[ex24.c:55-61] Print everything out so you can see if you have it right.  Notice
    that \ident{EYE\_COLOR\_NAMES} is used to print out what the \ident{EyeColor}
    enum is actually called.
\end{description}


\section{What You Should See}

When you run this program you should see your inputs being properly converted.
Make sure you try to give it bogus input too so you can see how it protects
against the input.

\begin{code}{ex24 output}
\begin{lstlisting}
<< d['code/ex24.out|dexy'] >>
\end{lstlisting}
\end{code}


\section{How To Break It}

This is all fine and good, but the real important part of this exercise is
how \func{scanf} actually sucks.  It's fine for simple conversion of numbers,
but fails for strings because it's difficult to tell \func{scanf} how big a buffer
is before you read.  There's also a problem with a function like \func{gets}
(not \func{fgets}, the non-f version) which we avoided.  That function has
no idea how big the input buffer is at all and will just trash your program.

To demonstrate the problems with \func{fscanf} and strings, change the lines
that use \func{fgets} so they are \verb|fscanf(stdin, "%50s", you.first_name)|
and then try to use it again.  Notice it seems to read too much and then 
eat your enter key?  This doesn't do what you think it does, and really
rather than deal with weird \func{scanf} issues, just use \func{fgets}.

Next, change the \func{fgets} to use \func{gets}, then bust out your
\program{valgrind} and do this:  \verb|valgrind ./ex24 < /dev/urandom|
to feed random garbage into your program.  This is called "fuzzing"
your program, and it is a good way to find input bugs.  In this case,
you're feeding garbage from the \file{/dev/urandom} file, and then watching
it crash.  On some platforms you may have to do this a few times, or even
adjust the \ident{MAX\_DATA} define so it's small enough.

The \func{gets} function is so bad that some platforms actually warn you
when the \emph{program} runs that you're using \func{gets}.  You should
never use this function, ever.

Finally, take the input for \ident{you.eyes} and remove the check that the
number given is within the right range.  Then feed it bad numbers like -1 or
1000.  Do this under Valgrind too so you can see what happens.

\section{The I/O Functions}

This is a short list of various I/O functions that you should look up and
create index cards that have the function name, what it does, and all the
variants similar to it.

\begin{enumerate}
\item fscanf
\item fgets
\item fopen
\item freopen
\item fdopen
\item fclose
\item fcloseall
\item fgetpos
\item fseek
\item ftell
\item rewind
\item fprintf
\item fwrite
\item fread
\end{enumerate}

Go through these and memorize the different variants and what they do.  For example,
for the card on \func{fscanf} you'll have \func{scanf}, \func{sscanf}, \func{vscanf},
etc. and then what each of those do on the back.

Finally, to get the information you need for these cards, use \program{man} to
read the help for it.  For example, the page for \func{fscanf} comes from 
\verb|man fscanf|.


\section{Extra Credit}

\begin{enumerate}
\item Rewrite this to not use \func{fscanf} at all.  You'll need to use
    functions like \func{atoi} to convert the input strings to numbers.
\item Change this to use plain \func{scanf} instead of \func{fscanf} to
    see what the difference is.
\item Fix it so that the input names get stripped of the trailing newline
    characters and any whitespace.
\item Use \ident{scanf} to write a function that reads a character at a time
    and files in the names but doesn't go past the end.  Make this function
    generic so it can take a size for the string, and make sure you end
    the string with \verb|'\0'| no matter what.
\end{enumerate}

