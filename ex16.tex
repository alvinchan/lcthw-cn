\chapter{Exercise 16: Structs And Pointers To Them}

In this exercise you'll learn how to make a \ident{struct}, point a pointer
at them, and use them to make sense of internal memory structures.  I'll also
apply the knowledge of pointers from the last exercise and get you
constructing these structures from raw memory using \func{malloc}.

As usual, here's the program we'll talk about, so type it in and
make it work:

\begin{code}{ex16.c}
<< d['code/ex16.c|pyg|l'] >>
\end{code}

To describe this program, I'm going to use a different approach
than before.  I'm not going to give you a line-by-line breakdown
of the program, but I'm going to make \emph{you} write it.  I'm
going to give you a guide through the program based on the
parts it contains, and your job is to write out what each line does.

\begin{description}
\item [includes] I include some new header files here to gain
    access to some new functions.  What does each give you?
\item [struct Person] This is where I'm creating a structure that
    has 4 elements to describe a person.  The final result is a
    new compound type that lets me reference these elements all
    as one, or each piece by name.  It's similar to a row of a database
    table or a class in an OOP language.
\item[function Person\_create] I need a way to create these structures
    so I've made a function to do that.  Here's the important 
    things this function is doing:
    \begin{enumerate}
    \item I use \func{malloc} for "memory allocate" to ask the OS
        to give me a piece of raw memory.
    \item I pass to \func{malloc} the \verb|sizeof(struct Person)| 
        which calculates the total size of the struct, given all the 
        fields inside it.
    \item I use \func{assert} to make sure that I have a valid 
        piece of memory back from malloc.  There's a special constant called
    \ident{NULL} that you use to mean "unset or invalid pointer".  This
    \func{assert} is basically checking that malloc didn't return a
        NULL invalid pointer.
    \item I initialize each field of \ident{struct Person} using the
        \verb|x->y| syntax, to say what part of the struct I want to set.
    \item I use the \func{strdup} function to duplicate the string 
        for the name, just to make sure that this structure actually owns
        it.  The \func{strdup} actually is like \func{malloc} and it also
        copies the original string into the memory it creates.
    \end{enumerate}
\item[function Person\_destroy] If I have a create, then I always need
    a destroy function, and this is what destroys \ident{Person} structs.
    I again use \func{assert} to make sure I'm not getting bad input.
    Then I use the function \func{free} to return the memory I got with
    \func{malloc} and \func{strdup}.  If you don't do this you get
    a "memory leak".
\item[function Person\_print] I then need a way to print out people,
    which is all this function does.  It uses the same \verb|x->y|
    syntax to get the field from the struct to print it.
\item[function main] In the main function I use all the previous
    functions and the \ident{struct Person} to do the following:
    \begin{enumerate}
    \item Create two people, \ident{joe} and \ident{frank}.
    \item Print them out, but notice I'm using the \verb|%p| 
        format so you can see \emph{where} the program
        has actually put your struct in memory.
    \item Age both of them by 20 years, with changes to their
        body too.
    \item Print each one after aging them.
    \item Finally destroy the structures so we can clean up 
        correctly.
    \end{enumerate}
\end{description}

Go through this description carefully, and do the following:

\begin{enumerate}
\item Look up every function and header file you don't know about.
    Remember that you can usually do \verb|man 2 function| or
    \verb|man 3 function| and it'll tell you about it.  You can also
    search online for the information.
\item Write a \emph{comment} above each and every single line saying
    what the line does in English.
\item Trace through each function call and variable so you know where
    it comes from in the program.
\item Look up any symbols you don't know as well.
\end{enumerate}


\section{What You Should See}

After you augment the program with your description comments,
make sure it really runs and produces this output:

\begin{code}{ex16 output}
\begin{lstlisting}
<< d['code/ex16.out'] >>
\end{lstlisting}
\end{code}

\section{Explaining Structures}

If you've done the work I asked you then structures should be
making sense, but let me explain them explicitly just to make
sure you've understood it.

A structure in C is a collection of other data types (variables)
that are stored in one block of memory but let you access each
variable independently by name.  They are similar to a record
in a database table, or a very simplistic class in an object
oriented language.  We can break one down this way:

\begin{enumerate}
\item In the above code, you make a \ident{struct} that has the fields 
    you'd expect for a person: name, age, weight, height.
\item Each of those fields has a type, like \ident{int}.
\item C then packs those together so they can all be contained in 
    one single \ident{struct}.
\item The \ident{struct Person} is now a \emph{compound data type}, which
    means you can now refer to \ident{struct Person} in the same kinds
    of expressions you would other data types.
\item This lets you pass the whole cohesive grouping to other
    functions, as you did with \func{Person\_print}.
\item You can then access the individual parts of a
    \ident{struct} by their names using \verb|x->y| if you're
    dealing with a pointer.
\item There's also a way to make a struct that doesn't need
    a pointer, and you use the \verb|x.y| (period) syntax
    to work with it.  You'll do this in the Extra Credit.
\end{enumerate}

If you didn't have \ident{struct} you'd need to figure out
the size, packing, and location of pieces of memory with
contents like this.  In fact, in most early assembler code
(and even some now) this is what you do.  With C you can 
let C handle the memory structuring of these compound data types
and then focus on what you do with them.


\section{How To Break It}

With this program the ways to break it involve how you use
the pointers and the \func{malloc} system:

\begin{enumerate}
\item Try passing \ident{NULL} to \func{Person\_destroy} to see what
    it does.  If it doesn't abort then you must not have the
    \file{-g} option in your Makefile's \ident{CFLAGS}.
\item Forget to call \func{Person\_destroy} at the end, then run
    it under \program{Valgrind} to see it report that you forgot
    to free the memory.  Figure out the options you need to pass
    to \program{Valgrind} to get it to print how you leaked
    this memory.
\item Forget to free \ident{who->name} in \func{Person\_destroy}
    and compare the output.  Again, use the right options to 
    see how \program{Valgrind} tells you exactly where you messed
    up.
\item This time, pass \ident{NULL} to \func{Person\_print} and
    see what \program{Valgrind} thinks of that.
\item You should be figuring out that \ident{NULL} is a quick way
    to crash your program.
\end{enumerate}

\section{Extra Credit}

In this exercise I want you to attempt something difficult for
the extra credit:  Convert this program to \emph{not} use pointers
and \func{malloc}.  This will be hard, so you'll want to research
the following:

\begin{enumerate}
\item How to create a \ident{struct} on the \emph{stack}, which
    means just like you've been making any other variable.
\item How to initialize it using the \verb|x.y| (period) character
    instead of the \verb|x->y| syntax.
\item How to pass a structure to other functions without using
    a pointer.
\end{enumerate}

