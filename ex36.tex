\chapter{Exercise 36: Safer Strings}

I've already introduced you to the \href{http://bstring.sourceforge.net/}{Better String} library in Exercise 26 when we made \program{devpkg}.  This exercise
is designed to get you into using bstring from now on, why C's strings are
an incredibly bad idea, and then have you change the \file{liblcthw} 
code to use bstring.

\section{Why C Strings Were A Horrible Idea}

When people talk about problems with C, it's concept of a "string" is one of the top flaws.
You've been using these extensively, and I've talked about the kinds of flaws they have, but there's
not much that explains exactly why C strings are flawed and always will be.  I'll try to explain
that right now, but part of my explanation will just be that after decades of using C's strings
there's enough evidence that they are just a bad idea.

It is impossible to confirm that any given C string is valid:

\begin{enumerate}
\item A C string is invalid if it does not end in \verb|'\0'|.
\item Any loop that processes an invalid C string will loop infinitely (or, just buffer overflow).
\item C strings do not have a known length, so the only way to check if it's terminated correctly is to loop through it.
\item Therefore, it is not possible to validate a C string without possibly looping infinitely.
\end{enumerate}

This is simple logic.  You can't write a loop that checks if a C string is valid
because invalid C strings cause loops to never terminate.  That's it, and the
only solution is to \emph{include the size}.  Once you know the size you can
avoid the infinite loop problem.  If you look at the two functions I showed
you from Exercise 27 you can see this:

\begin{code}{ex36.c}
<< d['code/ex36.c|pyg|l'] >>
\end{code}

Imagine you want to add a check to the \ident{copy} function to confirm that
the \ident{from} string is valid.  How would you do that?  Why you'd write a
loop that checked that the string ended in \verb|'\0'|.  Oh wait, if the string
doesn't end in \verb|'\0'| then how does the checking loop end?  It doesn't.
Checkmate.

No matter what you do, you can't check that a C string is valid without
knowing the length of the underlying storage, and in this case the 
\ident{safercopy} includes those lengths.  This function doesn't have
the same problem as it's loops will always terminate, even if you lie
to it about the size, you still have to give it a finite size.

What the Better String library does is create a struct that always includes
the length of the string's storage.  Because the length is always available
to a \ident{bstring} then all of its operations can be safer.  The loops
will terminate, the contents can be validated, and it will not have this
major flaw.  The bstring library also comes with a ton of operations
you need with strings, like splitting, formatting, searching, and 
they are most likely done right and safer.

There could be flaws in bstring, but it's been around a long time so
those are probably minimal.  They still find flaws in \file{glibc}
so what's a programmer to do right?

\section{Using bstrlib}

There's quite a few improved string libraries, but I like \file{bstrlib}
because it fits in one file for the basics and has most of the stuff you need
to deal with strings.  You've already used it a bit, so in this exercise you'll
go get the two files \file{bstrlib.c} and \file{bstrlib.h} from the
\href{http://bstring.sourceforge.net/}{Better String} project.

Here's me doing this in the \file{liblcthw} project directory:

\begin{code}{Adding bstrlib.c}
<< d['code/ex36.sh-session|pyg|l'] >>
\end{code}

On line 14 you seem me edit the \file{bstrlib.c} file to move it
to a new location and to fix a bug on OSX.  Here's the diff:

\begin{code}{bstrlib.diff}
<< d['code/ex36.diff|pyg|l'] >>
\end{code}

That is, change the include to be \file{<lcthw/bstrlib.h>}, and then
fix one of the \ident{ifdef} at line 2759.

\section{Learning The Library}

This exercise is short and simply getting you ready for the remaining 
exercises that use the library.  In the next two exercises I'll use
\file{bstrlib.c} to create a \ident{Hashmap} data structure.

You should now get familiar with this library by reading the 
header file, the implementations, and then write a \file{tests/bstr\_tests.c}
that tests out the following functions:

\begin{description}
\item[bfromcstr] Create a bstring from a C style constant.
\item[blk2bstr] Same but give the length of the buffer.
\item[bstrcpy] Copy a bstring.
\item[bassign] Set one bstring to another.
\item[bassigncstr] Set a bstring to a C string's contents.
\item[bassignblk] Set a bstring to a C string but give the length.
\item[bdestroy] Destroy a bstring.
\item[bconcat] Concatenate one bstring onto another.
\item[bstricmp] Compare two bstrings returning the same result as strcmp.
\item[biseq] Tests if two bstrings are equal.
\item[binstr] Tells if one bstring is in another.
\item[bfindreplace] Find one bstring in another then replace it with a third.
\item[bsplit] How to split a bstring into a bstrList.
\item[bformat] Doing a format string, super handy.
\item[blength] Getting the length of a bstring.
\item[bdata] Getting the data from a bstring.
\item[bchar] Getting a char from a bstring.
\end{description}

Your test should try out all of these operations, and a few more that you
find interesting from the header file.  Make sure to run the test under
\program{valgrind} to make sure you use the memory correctly.

