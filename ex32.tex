\chapter{Exercise 32: Double Linked Lists}

The purpose of this book is to teach you how your computer really works, and included in that is
how various data structures and algorithms function.  Computers by themselves don't do a lot of
useful processing.  To make them do useful things you need to structure the data and then
organize processing on these structures.  Other programming languages either include libraries
that implement all of these structures, or they have direct syntax for them.  C makes you
implement all the data structures you need yourself, which makes it the perfect language to
learn how they actually work.

My goal in teaching you these data structures and these algorithms is to help you do three
things:

\begin{enumerate}
\item Understand what is really going on in Python, Ruby, or JavaScript code like: \verb|data = {"name": "Zed}|
\item Get even better at C code by applying what you know to a set of solved problems using the data structures.
\item Learn a core set of data structures and algorithms so that you are better informed about what ones work best in certain situations.
\end{enumerate}


\section{What Are Data Structures}

The name "data structure" is self-explanatory.  It is an organization of data that fits a certain model.  Maybe the model is
designed to allow processing the data in a new way.  Maybe it's just organized to store it on disk efficiently.  In this book
I'll follow a simple pattern for making data structures that works reliably:

\begin{enumerate}
\item Define a struct for the main "outer structure".
\item Define a struct for the contents, usually nodes with links between them.
\item Create functions that operate on these two.
\end{enumerate}

There's other styles of data structures in C, but this pattern works well and is consistent for most data
structures you'll make.


\section{Making The Library}

For the rest of this book you'll be creating a library that you can use when you're done with this book.  This library will
have the following elements:

\begin{enumerate}
\item Header (.h) files for each data structure.
\item Implementation (.c) files for the algorithms.
\item Unit tests that test all of them to make sure they keep working.
\item Documentation we'll autogenerate from the header files.
\end{enumerate}

You already have the \file{c-skeleton} so use it to create a \file{liblcthw} project:

\begin{code}{ex32.sh-session}
<< d['code/ex32.sh-session|pyg|l'] >>
\end{code}

In this session I'm doing the following:

\begin{enumerate}
\item Copy the \file{c-skeleton} over.
\item Edit the Makefile to change \file{libYOUR\_LIBRARY.a} to \file{liblcthw.a}
    as the new \ident{TARGET}.
\item Make the \file{src/lcthw} directory where we'll put our code.
\item Move the \file{src/dbg.h} into this new directory.
\item Edit \file{tests/minunit.h} so that it uses \verb|#include <lcthw/dbg.h>|
    as the include.
\item Get rid of the source and test files we don't need for \file{libex29.*}.
\item Clean up everything that's left over.
\end{enumerate}

With that you're ready to start building the library, and the first data structure
I'll build is the Double Linked List.

\section{Double Linked Lists}

The first data strucutre we'll add to \file{liblcthw} is a double linked list.
This is the simplest data structure you can make, and it has useful properties
for certain operations.  A linked list works by nodes having pointers to their
next or previous element.  A "double linked list" contains pointers to both,
while a "single linked list" only points at the next element.

Because each node has pointers to the next and previous, and because you
keep track of the first and last element of the list, you can do some operations
very quickly.  Anything that involves inserting or deleting an element 
will be very fast.  They are also easy to implement by most people.

The main disadvantage of a linked list is that traversing it involves
processing every single pointer along the way.  This means that searching,
most sorting, or iterating over the elements will be slow.  It also means
that you can't really jump to random parts of the list.  If you had an 
array of elements you could just index right into the middle of the list,
but a linked list uses a stream of pointers.  That means if you want
the 10th element, you have to go through elements 1-9.

\subsection{Definition}

As I said in the introduction to this exercise, the process to follow is
to first write a header file with the right C struct statements in it.

\begin{code}{src/lcthw/list.h}
<< d['code/liblcthw/src/lcthw/list.h|pyg|l'] >>
\end{code}

The first thing I do is create two structs for the \ident{ListNode} and
the \ident{List} that will contain those nodes.  This creates the data
structure I'll use in the functions and macros I define after that.  If
you read through these functions they seem rather simple.  I'll be 
explaining them when I cover the implementation, but hopefully you can
guess what they do.

How the data structure works is each \ident{ListNode} has three components:

\begin{enumerate}
\item A value, which is a pointer to anything and stores the thing we want
    to put in the list.
\item A \ident{ListNode *next} pointer which points at another ListNode 
    that holds the next element in the list.
\item A \ident{ListNode *prev} that holds the previous element.  Complex
    right?  Calling the previous thing "previous".  I could have used
    "anterior" and "posterior" but only a jerk would do that.
\end{enumerate}

The \ident{List} struct is then nothing more than a container for these
\ident{ListNode} structs that have been linked together in a chain.
It keeps track of the \ident{count}, \ident{first} and \ident{last}
element of the list.

Finally, take a look at \file{src/lcthw/list.h:37} where I define
the \ident{LIST\_FOREACH} macro.  This is a common idiom where you
make a macro that generates iteration code so people can't mess
it up.  Getting this kind of processing right can be difficult with
data structures, so writing macros helps people out.  You'll see
how I use this when I talk about the implementation.

\subsection{Implemention}

Once you understand that, you mostly understand how a double linked list
works.  It is nothing more than nodes with two pointers to the next and
previous element of the list.  You can then write the \file{src/lcthw/list.c}
code to see how each operation is implemented.

\begin{code}{src/lcthw/list.c}
<< d['code/liblcthw/src/lcthw/list.c|pyg|l'] >>
\end{code}

I then implement all of the operations on a double linked list that can't
be done with simple macros.  Rather than cover every tiny little line of
this file, I'm going to give high-level overview of every operation
in both the \file{list.h} and \file{list.c} file, then leave you to read
the code.

\begin{description}
\item[list.h:List\_count] Returns the number of elements in the list, which is 
    maintained as elements are added and removed.
\item[list.h:List\_first] Returns the first element of the list, but does not
    remove it.
\item[list.h:List\_last] Returns the last element of the list, but does not
    remove it.
\item[list.h:LIST\_FOREACH] Iterates over the elements in the list.
\item[list.c:List\_create] Simply creates the main \ident{List} struct.
\item[list.c:List\_destroy] Destroys a \ident{List} and any elements it might have.
\item[list.c:List\_clear] Convenience function for freeing the \emph{values} in each
    node, not the nodes.
\item[list.c:List\_clear\_destroy] Clears and destroys a list.  It's not very efficient since it loops through them twice.
\item[list.c:List\_push] The first operation that demonstrates the advantage of a
    linked list.  It adds a new element to the end of the list, and because that's
    just a couple of pointer assignments, does it very fast.
\item[list.c:List\_pop] The inverse of \ident{List\_push}, this takes the last
    element off and returns it.
\item[list.c:List\_shift] The other thing you can easily do to a linked list is
    add elements to the \emph{front} of the list very fast.  In this case I call
    that \ident{List\_shift} for lack of a better term.
\item[list.c:List\_unshift] Just like \ident{List\_pop}, this removes the first
    element and returns it.
\item[list.c:List\_remove] This is actually doing all of the removal when you do
    \ident{List\_pop} or \ident{List\_unshift}. 
    Something that seems to always be difficult in 
    data structures is removing things, and this function is no different. It
    has to handle quite a few conditions depending on if the element
    being removed is at the front; the end; both front and end; or middle.
\end{description}

Most of these functions are nothing special, and you should be able to easily
digest this and understand it from just the code.  You should definitely focus
on how the \ident{LIST\_FOREACH} macro is used in \ident{List\_destroy} so you
can understand how much it simplifies this common operation.

\section{Tests}

After you have those compiling it's time to create the test that makes sure
they operate correctly.

\begin{code}{tests/list\_tests.c}
<< d['code/liblcthw/tests/list_tests.c|pyg|l'] >>
\end{code}

This test simply goes through every operation and makes sure it works.  I use
a simplification in the test where I create just one \ident{List *list} for
the whole program, then have the tests work on it.  This saves the trouble
of building a \ident{List} for every test, but it could mean that some tests
only pass because of how the previous test ran.  In this case I try to make
each test keep the list clear or actually use the previous test's results.

\section{What You Should See}

If you did everything right, then when you do a build and run the unit tests
it should look like this:

\begin{code}{Ex32 Session}
<< d['code/ex32.build.sh-session|pyg|l'] >>
\end{code}

Make sure 6 tests ran, that it builds without warnings or errors, and that it's
making the \file{build/liblcthw.a} and \file{build/liblcthw.so} files.

\section{How To Improve It}

Instead of breaking this, I'm going to tell you how to improve the code:

\begin{enumerate}
\item You can make \ident{List\_clear\_destroy} more efficient by using
    \ident{LIST\_FOREACH} and doing both \ident{free} calls inside one
    loop.
\item You can add asserts for preconditions that it isn't given a \ident{NULL}
    value for the \ident{List *list} parameters.
\item You can add invariants that check the list's contents are always correct,
    such as \ident{count} is never \verb|< 0|, and if \verb|count > 0| then \ident{first} isn't NULL.
\item You can add documentation to the header file in the form of comments before
    each struct, function, and macro that describes what it does.
\end{enumerate}

These amount to going through the defensive programming practices I talked about
and "hardening" this code against flaws or improving usability.  Go ahead and
do these things, then find as many other ways to improve the code.

\section{Extra Credit}

\begin{enumerate}
\item Research double vs. single linked lists and when one is preferred over
    the other.
\item Research the limitations of a double linked list.  For example, while they
    are efficient for inserting and deleting elements, they are very slow for
    iterating over them all.
\item What operations are missing that you can imagine needing?  Some examples
    are copying, joining, splitting.  Implement these operations and write the
    unit tests for them.
\end{enumerate}

