\chapter{习题6: 变量类型}

现在你应该对 C 语言的结构有了大概的了解,那么接下来我们再做些十分简单的事情,即输出不同类型的变量:

\begin{code}{ex6.c}
<< d['code/ex6.c|pyg|l'] >>
\end{code}

在这个程序中我们将:
1. 声明不同类型的变量,并且
2. 使用不同的字符串格式(string format) \ident{printf} 来打印输出它们。


\section{你应该看到的结果}

你的输出应该和我的一样,你也许也会发现这些 C 的字符串格式与 Python 和其他语言的字符串格式是多么的相似。其实它们已经存在很长时间了。

\begin{code}{ex6 output}
\begin{lstlisting}
<< d['code/ex6.out|dexy'] >>
\end{lstlisting}
\end{code}

你将看到一系列的“类型(types)”, 它们将告诉 C 编译器哪些变量应该被提出来,并格式化字符串以匹配不同类型。以下是它们匹配过程的细节:

\begin{description}
\item[整数型 (Integer) ] 使用 \ident{int} 声明整数, 用 \verb|%d| 打印输出。
\item[浮点型 (float)] 根据大小决定使用单精度浮点型 (\ident{float}) 或双精度浮点型 (\ident{double}) 声明浮点数 (双精度型更大些), 用 \verb|%f| 打印输出。
\item[字符型 (char)] 使用 \ident{char} 声明字符型, 写法是用单引号 \verb|'| 包围字符,用 \verb|%c| 打印输出。
\item[字符串 (string)] 使用 \verb|char name[]| 声明字符串型, 写法是用双引号 \verb|"| 包围字符, 用 \verb|%s| 打印输出。 
\end{description}

这里需要注意的是C语言非常注意区分单双引号。单引号是为字符 \ident{char} 所准备的,而双引号则是为一堆字符,既字符串型, \ident{char[]} 所准备的。

\begin{aside}{C数据类型的英文缩写}
当我们谈论到 C 数据类型时,在英语中我将直接使用 char[] 来表示, 而不会使用完整字符名 SOMENAME[]. 但事实上作为 C 代码这是不对的, 缩写仅仅是书中一种表达数据类型的简便方式。
\end{aside}

\section{让程序出错}

你可以通过给 printf 赋错误的值来搅坏这个程序。 用那条打印我的名字的指令来打个比方,如果在参数中,把变量 \ident{initial} 放在变量 \ident{first\_name} 的前面, bug 就来了。 真这样运行的话,编译器会对你发飙,你会得到个“段故障(Segmentation fault)”,就像以下这样:

\begin{code}{ex6 explosion}
\begin{lstlisting}
<< d['code/ex6.bad.out|dexy'] >>
\end{lstlisting}
\end{code}

在 Valgrind 下再次运行,看看关于“Invalid read of size 1”的内容会告诉你些什么。


\section{加分习题}

\begin{enumerate}
\item 除了改变 \ident{printf} 以外,再想想有没有别的办法可以弄坏这个代码,并找出修复它的方法。
\item 上网搜索 "printf formats" ,找出并尝试一些稀有格式。
\item 探索一下有多少种方式可以写出数字。 尝试以下几个,并找到更多: octal, hexadecimal。
\item 尝试打印输出一个空字符串 \verb|""|.
\end{enumerate}

