\chapter{Exercise 25: Variable Argument Functions}

In C you can create your own versions of functions like \func{printf} and
\func{scanf} by creating a "variable argument function".  These functions use
the header \file{stdarg.h} and with them you can create nicer interfaces to
your library.  They are handy for certain types of "builder" functions,
formatting functions, and anything that takes variable arguments.

Understanding "vararg functions" is \emph{not} essential to creating C programs.
I think I've used it maybe a 20 times in my code in the years I've been
programming.  However, knowing how a vararg function works will help you
debug the ones you use and gives you more understanding of the computer.

\begin{code}{ex25.c}
<< d['code/ex25.c|pyg|l'] >>
\end{code}

This program is similar to the previous exercise, except I have written
my own \func{scanf} style function that handles strings the way I want.
The main function should be clear to you, as well as the two functions
\func{read\_string} and \func{read\_int} since they do nothing new.

The varargs function is called \func{read\_scan} and it does the same
thing that \func{scanf} is doing using the \ident{va\_list} data
structure and it's supporting macros and functions.  Here's how it works:

\begin{enumerate}
\item I set as the last parameter of the function the keyword \verb|...|
    which indicates to C that this function will take any number of arguments
    after the \ident{fmt} argument.  I could put many other arguments before
    this, but I can't put anymore after this.
\item After setting up some variables, I create a \ident{va\_list} variable
    and initialize it with \ident{va\_start}.  This configures the gear
    in \file{stdarg.h} that handles variable arguments.
\item I then use a \ident{for-loop} to loop through the format string
    \ident{fmt} and process the same kind of formats that \func{scanf}
    has, but much simpler.  I just have integers, characters, and strings.
\item When I hit a format, I use the \ident{switch-statement} to figure out
    what to do.
\item Now, to \emph{get} a variable from the \ident{va\_list argp} I use the
    macro \ident{va\_arg(argp, TYPE)} where TYPE is the exact type of what
    I will assign this function parameter to.  The downside to this design
    is you're flying blind, so if you don't have enough parameters then
    oh well, you'll most likely crash.
\item The interesting difference from \func{scanf} is I'm assuming that people
    want \func{read\_scan} to create the strings it reads when it hits a
    \ident{'s'} format sequence.  When you give this sequence, the function
    takes two parameters off the \ident{va\_list argp} stack: the max function
    size to read, and the output character string pointer.  Using that information
    it just runs \func{read\_string} to do the real work.
\item This makes \func{read\_scan} more consistent than \func{scanf} since you
    \emph{always} give an address-of \verb|\&| on variables to have them set
    appropriately.
\item Finally, if it encounters a character that's not in the format, it just reads
    one char to skip it.  It doesn't care what that char is, just that it should
    skip it.
\end{enumerate}


\section{What You Should See}

When you run this one it's similar to the last one:

\begin{code}{ex25.out}
\begin{lstlisting}
<< d['code/ex25.out'] >>
\end{lstlisting}
\end{code}

\section{How To Break It}

This program should be more robust against buffer overflows, but it doesn't
handle the formatted input as well as \func{scanf}.  To try breaking this,
change the code that you forget to pass in the initial size for '\%s' formats.
Try also giving it more data than \ident{MAX\_DATA}, and then see how not
using \func{calloc} in \func{read\_string} changes how it works.  Finally,
there's a problem that fgets eats the newlines, so try to fix that using
\func{fgetc} but leave out the '\\0' that ends the string.

\section{Extra Credit}

\begin{enumerate}
\item Make double and triple sure that you know what each of the \ident{out\_}
    variables are doing.  Most important is \ident{out\_string} and how it's
    a pointer to a pointer, so getting when you're setting the pointer vs. the
    contents is important.  Break down each of the 
\item Write a similar function to \func{printf} that uses the varargs system
    and rewrite \func{main} to use it.
\item As usual, read the man page on all of this so you know what it does
    on your platform.  Some platforms will use macros and others use
    functions, and some have these do nothing.  It all depends on the 
    compiler and the platform you use.
\end{enumerate}

