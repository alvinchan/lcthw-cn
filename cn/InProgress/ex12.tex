\chapter{Exercise 12: If, Else-If, Else}

在每种语言中都很常见的就是\ident{ if 语句}, C中同样也有。 下面的代码使用的就是\ident{ if 语句}保证你只输入一个或两个参数(arguments):

\begin{code}{ex12.c}
<< d['code/ex12.c|pyg|l'] >>
\end{code}

\ident{if 语句}的格式是这样的:

\begin{Verbatim}
    if(TEST) {
        CODE;
    } else if(TEST) {
        CODE;
    } else {
        CODE;
    }
\end{Verbatim}

这和其他大多数语言都相同,除了特殊版本的C:

\begin{enumerate}
\item 正如前面所提到的, 如果\ident{TEST}()的部分值为0,结果就为假,其他情况就为真。
\item 你不得不在\ident{TEST}元素的周围放置圆括号,但其他的一些语言让你掠过这一步.
\item 你不需要用\verb|{}|括号结束代码, 但不这么做是一种\emph{非常}不好的格式。 大括号让代码分支的开始和结束更明朗。 如果你不包含它,各种莫名其妙的错误就会出现。
\end{enumerate}

除了那个, 它们和其他的做的一样。 你既不需要\ident{else if}也不需要\ident{else}.

\section{你应该看到的结果}

这是一个值得运行并尝试的例子:

\begin{code}{ex12 output}
\begin{lstlisting}
<< d['code/ex12.out|dexy'] >>
\end{lstlisting}
\end{code}

\section{让程序出错}
这个例子不容易修改,因为它太简单了,但是尝试在\ident{ if 语句}中把它们混合起来测试。

\begin{enumerate}
\item 移除结尾的\ident{else}它就永远不会找到边缘条件(edge case)。
\item 把\verb|&&|换成\verb,||,你就用“或”(or)代替了“与”(and)测试,看看结果是怎样的。
\end{enumerate}

\section{加分习题}

\begin{enumerate}
\item 简单的向你介绍了\verb|&&|, 执行“与”比较, 上网查查和它不用的“布尔操作符”(逻辑与和按位与)。
\item 为这段程序再写点测试用例,看看有什么变化。
\item 回到练习10和11,使用\ident{if语句}使循环提前结束。 完成操作你需要\ident{break}语句。自己查查怎么使。
\item 第一个测试是否真的是对的? 对于你来说"第一个参数"(argument)和用户输入的第一个参数(argument)不同。 完善它。
\end{enumerate}

