\chapter{Exercise 12: If, Else-If, Else}

Something common in every language is the \ident{if-statement}, and
C has one.  Here's code that uses an \ident{if-statement} to make 
sure you enter only 1 or 2 arguments:

\begin{code}{ex12.c}
<< d['code/ex12.c|pyg|l'] >>
\end{code}

The format for the \ident{if-statement} is this:

\begin{Verbatim}
    if(TEST) {
        CODE;
    } else if(TEST) {
        CODE;
    } else {
        CODE;
    }
\end{Verbatim}

This is like most other languages except for some specific C 
differences:

\begin{enumerate}
\item As mentioned before, the \ident{TEST} parts are false if they
    evaluate to 0, and true otherwise.
\item You have to put parenthesis around the \ident{TEST} elements,
    while some other languages let you skip that.
\item You don't need the \verb|{}| braces to enclose the code, but
    it is \emph{very} bad form to not use them.  The braces make it
    clear where one branch of code begins and ends.  If you don't 
    include it then obnoxious errors come up.
\end{enumerate}

Other than that, they work like others do.  You don't need to have
either \ident{else if} or \ident{else} parts.

\section{What You Should See}

This one is pretty simple to run and try out:

\begin{code}{ex12 output}
\begin{lstlisting}
<< d['code/ex12.out|dexy'] >>
\end{lstlisting}
\end{code}

\section{How To Break It}
This one isn't easy to break because it's so simple, but try messing up the
tests in the \ident{if-statement}.

\begin{enumerate}
\item Remove the \ident{else} at the end and it won't catch the edge case.
\item Change the \verb|&&| to a \verb,||, so you get an "or" instead of "and" test
    and see how that works.
\end{enumerate}

\section{Extra Credit}

\begin{enumerate}
\item You were briefly introduced to \verb|&&|, which does an "and" comparison,
    so go research online the different "boolean operators".
\item Write a few more test cases for this program to see what you can come
    up with.
\item Go back to Exercises 10 and 11, and use \ident{if-statements} to make
    the loops exit early.  You'll need the \ident{break} statement to do that.
    Go read about it.
\item Is the first test really saying the right thing?  To you the "first argument"
    isn't the same first argument a user entered.  Fix it.
\end{enumerate}

