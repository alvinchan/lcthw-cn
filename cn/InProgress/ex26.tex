\chapter{习题 26: 第一个真正的程序}

到这里正好是本书的一半,你也该参加一次期中考核了。这次考核我要求你把我专为本书
写的一个叫 \program{devpkg} 的程序重写一遍,然后你要用一些关键方法改进代码,
最重要的方法就是为它写一些单元测试。

\begin{aside}{WARNING: Beta Draft Content}
I wrote this exercise before writing some of the exercises you might
need to complete this.  If you are attempting this one now, please
keep in mind that the software may have bugs, that you might have
problems because of my mistakes, and that you might not know everything
you need to finish it.  If so, tell me at help@learncodethehardway.org
and then wait until I finish the other exercises.
\end{aside}

\section{什么是 \program{devpkg}?}

\program{Devpkg} 是一个简单的 C 程序,它的功能是用来安装别的软件。这个软件是我专为
这本书写的,目的是教你学习真正的软件项目是怎样构架的,以及学习怎样重复使用别人写的 library。
它使用了一个称作 \href{http://apr.apache.org/}{The Apache Portable
Runtime (APR)} 可移植性 library,它里边有很多好用适用于很多平台(包括 Windows)的 C 函数。
除此之外,它所做的就是从网上或者本地抓到代码,然后执行一下我们每个人都会的 
\verb|./configure ; make ; make install| 而已。

本节习题中你的任务是从源代码 build \program{devpkg},完成我给你的每一个\emph{挑战},
然后通过阅读源代码来理解 \program{devpkg} 的功能和原理。

\subsection{What We Want To Make}

我们要做一个工具,它有三条命令:

\begin{enumerate}
\item[devpkg -S] 执行软件的全新安装。
\item[devpkg -I] 通过 URL 安装软件。
\item[devpkg -L] 列出所有安装了的软件。
\item[devpkg -F] 下载源代码以供手动 build。
\item[devpkg -B] 下载源代码,build 并且安装软件,即使在软件已被安装的情况下也会再装一遍。
\end{enumerate}

我们要求 \program{devpkg} 能够识别绝大部分的 URL,识别出这个 URL 对应的是哪个项目,然后
下载并且安装软件,最后记录下来它下载了哪个软件。我们还要求 \program{devpkg} 能够处理一份
简单的需求列表(dependency list),这样它就能同时将当前要安装的软件依赖的软件也安装起来。

\subsection{软件设计}

我们将通过非常简单的设计来达到我们的目的:

\begin{description}
\item[使用外部命令] 你的大部分工作将通过外部命令诸如 \program{curl}、\program{git}、\program{tar}
    来完成。这样可以减少实现 \program{devpkg} 所需的代码量。
\item[简单的文件数据库] 要做复杂也不难,不过起始阶段你只要建立一个单个文件的简单数据库
    \file{/usr/local/.devpkg/db},用来记录所安装的软件即可。
\item[只适用 /usr/local] 这里你也可以做得更高级,不过初始阶段我们就假设所有的东西都装到
    \file{/usr/local} 好了,这也是大部分 Unix 下软件的标准安装路径。
\item[configure, make, make install] 我们假设绝大部分软件可以通过执行 
    \program{configure; make; make install} 来安装,而 \program{configure} 可能
    不是必须的一步。如果你要安装的软件不支持这些基本的安装方式,那你可以通过一个选项来修改
    安装命令,不过更多的东西 \program{devpkg} 就不去理会了。
\item[用户可以是 root] 我们将假设用户可以通过 sudo 命令得到 root 权限,不过在执行完安装
    命令以后他们将回到普通用户级别。
\end{description}

这样我们可以让程序的初始体积比较小,而功能也都能实现,这样我们就能顺利进行后面的学习,而以后
你也能够进一步修改它。

\subsection{Apache Portable Runtime}

接下来你要做的是利用 \href{http://apr.apache.org/}{Apache Portable Runtime (APR)} 
里一成套的可移植函数库以完成本项目。APR 不是必须项,就算不用它你也可以完成本程序,只不过你
需要写更多的代码罢了。我要求你使用 APR 的原因是让你习惯于链接并使用别的库文件。最后要说的
一点是,APR 在 \emph{Windows} 下也能工作,所以你学的 APL 技能可以用在很多别的平台上面。

你需要下载 \library{apr-1.4.5} 和 \library{apr-util-1.3} 这两个函数库,同时阅读一下
\href{http://apr.apache.org/}{APR 主站}提供的文档。

下面是一个用来安装的 shell 脚本,你需要将这个脚本手动誊写一遍,然后运行它,直到它能毫不
出错地安装 APR 为止。

\begin{code}{APR 安装脚本}
<< d['code/ex26.1.sh|pyg|l'] >>
\end{code}

我要求你写这个脚本是因为 \program{devpkg} 实现的功能和这个脚本一样,只不过参数更多,
功能也更完善而已。其实你可以用 shell 完全实现本项目,而且这样做代码量更少,不过在一本 C 语言
的书里教你 shell 程序还是有些不合适吧?

运行这个脚本,如果有错就修改过来,直到它能正常工作为止。这样你就装好了实现剩余项目所需的函数库。

\section{项目结构}

你需要通过建立一些文件来开始一个新项目。以下是我通常建立新项目的方法:

\begin{code}{项目骨架目录}
<< d['code/ex26.2.sh|pyg|l'] >>
\end{code}

\subsection{其他需求}

 你应该已经装了 APR 和 APR-util,所以现在你还需要几个文件作为基本 dependency:

\begin{enumerate}
\item 习题 20 中的 \file{dbg.h}
\item \file{bstrlib.h} 和 \file{bstrlib.c},它们来自 \href{http://bstring.sourceforge.net/}{http://bstring.sourceforge.net/}。下载并解压 .zip 文件,
把这两个文件复制出来即可。
\item 键入命令 \verb|make bstrlib.o|,如果执行失败,请阅读下面的“解决 bstring 的问题”。
\end{enumerate}

\begin{aside}{解决 bstring 的问题}
有些平台上面 bstring.c 会出现如下的错误:

\begin{lstlisting}
bstrlib.c:2762: error: expected declaration specifiers or '...' before numeric constant
\end{lstlisting}

这是因为 bstrlib.c 的作者用了一个有问题的 define 语句,这个语句有时不能正常工作。只要把这个 
\verb|#ifdef| 删掉并且重新编译即可。
\end{aside}

上述步骤完成以后,你应该准备了以下文件:\file{Makefile}、 \file{README}、
\file{dbg.h}、 \file{bstrlib.h}、 \file{bstrlib.c}。现在你可以继续了。

\section{Makefile}

\file{Makefile} 是一个很好的开始着手点,这样你可以计划好要 build 的内容,以及你需要建立的
代码文件。

\begin{code}{Makefile}
\begin{lstlisting}
<< d['code/ex26/Makefile'] >>
\end{lstlisting}
\end{code}

除了这个奇怪的 \verb|?=| 语法以外,这里基本没有你没见过的东西。这个表达式的意思是“如果 PREFIX 
的值还没被设定,那么就将 PREFIX 设为该值”。

\section{代码文件}

From the make file, we see that there's four dependencies for \program{devpkg}
which are:

\begin{description}
\item[bstrlib.o] Comes from \file{bstlib.c} and header file \file{bstlib.h} which
    you already have.
\item[db.o] From \file{db.c} and header file \file{db.h}, and it
    will contain code for our little "database" routines.
\item[shell.o] From \file{shell.c} and header \file{shell.h}, with a couple
    functions that make running other commands like \program{curl} easier.
\item[commands.o] From \file{command.c} and header \file{command.h}, and
    contains all the commands that \program{devpkg} needs to be useful.
\item[devpkg] It's not explicitly mentioned, but instead is the target
    (on the left) in this part of the \file{Makefile}. It comes from
    \file{devpkg.c} which contains the \func{main} function for the whole
    program.
\end{description}

Your job is to now create each of these files and type in their code
and get them correct.

\begin{aside}{Don't Be Fooled By The Magic Show}
You may read this description and think, "Man! How is it that Zed is
so smart he just sat down and typed these files out like this!? I
could never do that."  I didn't magically craft \program{devpkg} 
in this form with my awesome code powers.  Instead, what I did is this:

\begin{enumerate}
\item I wrote a quick little README to get an idea of how I wanted it
    to work.
\item I created a simple bash script (like the one you did) to figure
    out all the pieces that you need.
\item I made one .c file and hacked on it for a few days working through
    the idea and figuring it out.
\item I got it mostly working and debugged, \emph{then} I started
    breaking up the one big file into these four files.
\item After getting these files laid down, I renamed and refined the
    functions and data structures so they'd be more logical and "pretty".
\item Finally, after I had it working the \emph{exact same} but with
    the new structure, I added a few features like the \program{-F} and
    \program{-B} options.
\end{enumerate}

You're reading this in the order I want to teach it to you, but don't think
this is how I always build software.  Sometimes I already know the subject and
I use more planning.  Sometimes I just hack up an idea and see how well it'd
work.  Sometimes I write one, then throw it away and plan out a better one.  It
all depends on what my experience tells me is best, or where my inspiration
takes me.

If you run into an "expert" who tries to tell you there's only one
way to solve a programming problem, then they're lying to you.  Either
they actually use multiple tactics, or they're not very good.
\end{aside}

\subsection{The DB Functions}

There must be a way to record URLs that have been installed, list these
URLs, and check if something has already been installed so we can
skip it.  I'll use a simple flat file database and the \file{bstrlib.h}
library to do it.

First, create the \file{db.h} header file so you know what you'll be
implementing.

\begin{code}{db.h}
<< d['code/ex26/db.h|pyg|l'] >>
\end{code}

Then implement those functions in \file{db.c}, as you build this, use
\program{make} like you've been to get it to compile cleanly.

\begin{code}{db.c}
<< d['code/ex26/db.c|pyg|l'] >>
\end{code}

\subsubsection{Challenge 1: Code Review}

Before continuing, read every line of these files carefully and 
confirm that you have them entered in \emph{exactly}.  Read them
line-by-line backwards to practice that. Also trace each function 
call and make sure you are using \func{check} to validate the
return codes.  Finally, look up \emph{every} function that you
don't recognize either on the APR web site documentation, or
in the \file{bstrlib.h} and \file{bstrlib.c} source.


\subsection{The Shell Functions}

A key design decision for \program{devpkg} is to do most of the work
using external tools like \program{curl}, \program{tar}, and \program{git}.
We could find libraries to do all of this internally, but it's pointless
if we just need the base features of these programs.  There is no shame
in running another command in Unix.

To do this I'm going to use the \file{apr\_thread\_proc.h} functions
to run programs, but I also want to make a simple kind of "template"
system.  I'll use a \ident{struct Shell} that holds all the information
needed to run a program, but has "holes" in the arguments list where I
can replace them with values.

Look at the \file{shell.h} file to see the structure and the commands I'll use.
You can see I'm using \ident{extern} to indicate that other \file{.c} files
can access variables I'm defining in \file{shell.c}.

\begin{code}{shell.h}
<< d['code/ex26/shell.h|pyg|l'] >>
\end{code}

Make sure you've created \file{shell.h} exactly, and that you've got the
same names and number of \ident{extern Shell} variables.  Those are used
by the \func{Shell\_run} and \func{Shell\_exec} functions to run commands.
I define these two functions, and create the real variables in \file{shell.c}.

\begin{code}{shell.c}
<< d['code/ex26/shell.c|pyg|l'] >>
\end{code}

Read the \file{shell.c} from the bottom to the top (which is a common C source
layout) and you see I've created the actual \ident{Shell} variables that you
indicated were \ident{extern} in \file{shell.h}.  They live here, but are 
available to the rest of the program.  This is how you make global variables
that live in one \file{.o} file but are used everywhere.  You shouldn't make
many of these, but they are handy for things like this.

Continuing up the file we get to the \func{Shell\_run} function, which is
a "base" function that just runs a command based on what's in a \ident{Shell}
struct.  It uses many of the functions defined in \file{apr\_thread\_proc.h}
so go look up each one to see how it works.  This seems like a lot of work
compared to just using the \func{system} function call, but this also gives
you more control over the other program's execution.  For example, in our
\ident{Shell} struct we have a \ident{.dir} attribute which forces the program
to be in a specific directory before running.

Finally, I have the \func{Shell\_exec} function, which is a "variable arguments"
function.  You've seen this before, but make sure you grasp the \file{stdarg.h}
functions and how to write one of these.  In the challenge for this section
you are going to analyze this function.

\subsubsection{Challenge 2: Analyze Shell\_exec}

Challenge for these files (in addition to a full code review just like you
did in Challenge 1) is to fully analyze \func{Shell\_exec} and break down
exactly how it works.  You should be able to understand each line, how
the two \ident{for-loops} work, and how arguments are being replaced.

Once you have it analyzed, add a field to \ident{struct Shell} that gives
the number of variable \ident{args} that must be replaced.  Update all the
commands to have the right count of args, and then have an error check that
confirms these args have been replaced and error exit.

\subsection{The Command Functions}

Now you get to make the actual commands that do the work.  These commands
will use functions from APR, \file{db.h} and \file{shell.h} to do the 
real work of downloading and building software you want it to build.
This is the most complex set of files, so do them carefully.  As before, you
start by making the \file{commands.h} file, then implementing its functions
in the \file{commands.c} file.

\begin{code}{commands.h}
<< d['code/ex26/commands.h|pyg|l'] >>
\end{code}

There's not much in \file{commands.h} that you haven't seen already.  You
should see that there's some defines for strings that are used everywhere.
The real interesting code is in \file{commands.c}.

\begin{code}{commands.c}
<< d['code/ex26/commands.c|pyg|l'] >>
\end{code}

After you have this entered in and compiling, you can analyze it.  If you've
don the challenges until now, you should see how the \file{shell.c} functions
are being used to run shells and how the arguments are being replaced.  If
not then go back and make sure you \emph{truly} understand how \func{Shell\_exec}
actually works.

\subsubsection{Challenge 3: Critique My Design}

As before, do a complete review of this code and make sure it's exactly
the same.  Then go through each function and make sure you know how it 
works and what it's doing.  You also should trace how each function calls
the other functions you've written in this file and other files.  Finally,
confirm that you understand all the functions you're calling from APR here.

Once you have the file correct and analyzed, go back through and assume
I'm an idiot.  Then, criticize the design I have to see how you can improve
it if you can.  Don't \emph{actually} change the code, just create a little
\file{notes.txt} file and write down your thoughts and what you might change.


\subsection{The \file{devpkg} Main Function}

The last and most important file, but probably the simplest, is \file{devpkg.c}
where the \func{main} function lives.  There's no \file{.h} file for this, since
this one includes all the others.  Instead this just creates the executable
\program{devpkg} when combined with the other \file{.o} files from our 
\file{Makefile}.  Enter in the code for this file, and make sure it's 
correct.

\begin{code}{devpkg.c}
<< d['code/ex26/devpkg.c|pyg|l'] >>
\end{code}

\subsubsection{Challenge 4: The README And Test Files}

The challenge for this file is to understand how the arguments are
being processed, what the arguments are, and then create the \file{README}
file with instructions on how to use it.  As you write the README, also
write a simple \file{test.sh} that runs \program{./devpkg} to check that
each command is actually working against real live code.  Use the \verb|set -e|
at the top of your script so that it aborts on the first error.

Finally, run the program under valgrind and make sure it's all working
before moving on to the mid-term exam.

\section{The Mid-Term Exam}

Your final challenge is the mid-term exam and it involves three things:

\begin{enumerate}
\item Compare your code to my code available online and starting with 100\%, 
    remove 1\% for each line you got wrong.
\item Take your notes.txt on how you would improve the code and functionality
    of \program{devpkg} and implement your improvements.
\item Write an alternative version of \program{devpkg} using your other 
    favorite language or the one you think can do this the best.  Compare
    the two, then improve your \emph{C} version of \program{devpkg} based on what
    you've learned.
\end{enumerate}

To compare your code with mine, do the following:

\begin{lstlisting}
cd ..  # get one directory above your current one
git clone git://gitorious.org/devpkg/devpkg.git devpkgzed
diff -r devpkg devpkgzed
\end{lstlisting}

This will clone my version of \program{devpkg} into a directory
\program{devpkgzed} and then use the tool \program{diff} to compare
what you've done to what I did.  The files you're working with in
this book come directly from this project, so if you get different
lines then that's an error.

Keep in mind that there's no real pass or fail on this exercise, just
a way for you to challenge yourself to be as exact and meticulous as
possible.

