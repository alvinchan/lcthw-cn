\chapter{习题 4: 介绍 Valgrind}

是时候学习另外一个工具 \program{Valgrind} 了,它将伴随你学习 C 语言的整个过程。我现在介绍 \program{Valgrind} 给你,因为从现在起,在“让程序出错”这个小节里,每个练习你都将用到它。\program{Valgrind} 运行你的程序,然后报告你犯下的所有致命错误。它是一个很棒的自由软件,我经常在我写 C 代码的时候用到它。

还记得在上一个习题里,我让你修改你的代码移除了 \ident{printf} 函数的参数吗?它打印出了一些补寻常的结果,但我没有告诉你,为什么它会打印出这些结果。在这个习题里,我们将用 \program{Valgrind} 来探个究竟。

\begin{aside}{What's With All The Tools}
只经过几章,我们就学习了本书所需要的所有工具,而只学写了一点点代码。这是因为这本书的大部分读者不熟悉编译语言,当然也不知道自动化处理和有用的工具。让你马上接触 \ident{make}  和 \program{Valgrind} ,我就能用他们更快的教会你 C 语言并帮助你找到你在程序中犯的错误。

在这章练习后,一段时间内,我们将不会介绍其他任何工具,大部分将是代码和语法。但我们还将会学习一些工具,用来查看程序如何运行和帮我们理解一些常见错误和问题 。

\end{aside}

\section{安装 Valgrind}

 你能通过操作系统的包管理器来安装 \program{Valgrind},但我要你学习如何从源代码安装程序。它包括以下几个步骤:

\begin{enumerate}
\item 下载源代码包。
\item 解压文件到你的电脑。
\item 运行 \program{./configure} 设置配置。
\item 运行 \program{make} 创建程序, 就像你以前做过的那样。
\item 运行 \program{sudo make install} 把程序安装到你的电脑。
\end{enumerate}

下面是我流程的一个脚本,我要你试着 按它重做一次。
try to replicate:

\begin{code}{ex4.sh}
<< d['code/ex4.sh|pyg|l'] >>
\end{code}

但要根据新版的 Valgrind 更新流程。如果它创建失败,那么查出为什么出错。

\section{ 使用 Valgrind}

使用 \program{Valgrind} 很简单, 你只要运行 \verb|valgrind theprogram| 它就会运行你的程序,然后打印出你程序运行时的错误。在这个习题里,我们将让程序出错,然后我们修正这个程序。

 首先,我们把 \file{ex3.c} 的代码拿来用,换个名字叫 \file{ex4.c} ,当然,代码会故意弄错,为了练习,你需要重新输入一遍。

\begin{code}{ex4.c}
<< d['code/ex4.c|pyg|l'] >>
\end{code}

你可以看到,除了我在上面犯了两个经典错误之外,其余都是一样的。

\begin{enumerate}
\item 我没有初始化 \ident{height} 变量.
\item 我忘记给头一个 \ident{printf} 函数加上 \ident{age} 变量。
\end{enumerate}

\section{你应该看到的结果}

现在我们像平常一样创建它,然后用 \program{Valgrind} 运行它,而不是像以前那样直接运行程序(看 Source: “创建并用 Valgrind 运行 ex4.c”):

\begin{Terminal}{创建并用 Valgrind 运行 ex4.c}
\begin{lstlisting}
<< d['code/ex4.out|dexy'] >>
\end{lstlisting}
\end{Terminal}

输出很长,因为 \program{Valgrind} 会精确告知你,程序每一个错误的所在。你要逐行从头到尾的读一遍(行号在左边,这样你可以对照):

\begin{description}
\item[1] 你像平常一样用 \verb|make ex4| 创建程序。 请确认你的 \ident{cc} 指令编译时用了同样的参数,没有 \verb|-g| 选项,\program{Valgrind} 的输出将不带行号。
\item[2-6] 可以注意到编译器也对你发出了警告,它提醒你 "too few arguments for format". 那是你忘了加 \ident{age} 变量。
\item[7] 用 \verb|valgrind ./ex4| 运行你的程序。
\item[8] 然后 \program{Valgrind} 就:
    \begin{description}
        \item[14-18] On line \verb|main (ex4.c:11)| (read as "in the main function in
            file ex4.c at line 11) you have "Use of uninitialised value of size 8".
            You find this by looking at the error, then you see what's called a "stack trace"
            right under that.  The line to look at first (ex4.c:11) is the bottom one, 
            and if you don't see what's going wrong then you go up, so you'd try
            printf.c:35.  Typically it's the bottom most line that matters (in this case, on line 18).
        \item[20-24] Next error is yet another one on line ex4.c:11 in the main function. \program{Valgrind}
            hates this line.  This error says that some kind of if-statement or while-loop
            happened that was based on an uninitialized variable, in this case height.
        \item[25-35] The remaining errors are more of the same because the variable keeps getting
        used.
    \end{description}
\item[37-46] 最后,程序退出然后 \program{Valgrind} 做了个概要,向你展示你的程序有多糟糕。
\end{description}

That is quite a lot of information to take in, but here's how you deal with it:

\begin{enumerate}
\item Whenever you run your C code and get it working, rerun it under \program{Valgrind}
    to check it.
\item For each error that you get, go to the source:line indicated and
    fix it.  You may have to search online for the error message to figure out
    what it means.
\item Once your program is "Valgrind pure" then it should be good, and you
    have probably learned something about how you write code.
\end{enumerate}

在这个习题里,我没指望你能立马就熟练运用 \program{Valgrind} ,只是让你装好它并学习如何快速上手,这样我们就能在以后的习题里用上它。

\section{加分习题}

\begin{enumerate}
\item 根据 \program{Valgrind} 和编译器的提示,修正错误。
\item 在网上研读 \program{Valgrind} 。
\item 下载其他软件源码,自己创建它。尝试一些你已经用过但还从没自己动手编译创建过的软件。
\item 看看 \program{Valgrind} 的源码,读一下它的 Makefile ,别担心,这些对我也没有意义。
\end{enumerate}

