\chapter{习题 1: 重拾编译器}

这是你用C语言写的第一个简单程序:

\begin{code}{ex1.c}
<< d['code/ex1.c|pyg|l'] >>
\end{code}

把它保存为 \file{ex1.c} 然后输入:

\begin{Terminal}{Building ex1}
\begin{lstlisting}
$ make ex1
cc     ex1.c   -o ex1
\end{lstlisting}
\end{Terminal}

你的系统使用的命令有可能会不大一样,但最终应该都会编译出一个可运行文件叫 \file{ex1}

\section{你应该看到的结果}

你运行这个程序应该会看到以下输出。

\begin{Terminal}{Running ex1}
\begin{lstlisting}
$ ./ex1
Hello world.
\end{lstlisting}
\end{Terminal}

如果输出不同,你需要回到前一步,找出问题所在并修正它。

\section{让程序出错}

在本书中,我为每段程序都准备了这样一个小章节,在这个小章节里,将让你对程序做一些非常规的事情,
用一些怪异的方法运行程序或者改变代码,以此而造成程序出错或编译器报错。

对于这个程序,打开编译器的所有警告项,重新编译它:

\begin{Terminal}{Building ex1 with -Wall}
\begin{lstlisting}
$ rm ex1
$ CFLAGS="-Wall" make ex1
cc -Wall    ex1.c   -o ex1
ex1.c: In function 'main':
ex1.c:3: warning: implicit declaration of function 'puts'
$ ./ex1
Hello world.
$ 
\end{lstlisting}
\end{Terminal}


现在你得到了一个警告说函数“puts”是“implictly declared”(隐式申明)。
C 编译器很智能能够猜出你想要什么,但是你应该尽力消除所有的编译器警告。
对于这个警告,你可以把以下几行加入到 \file{ex1.c} 的顶部,并重新编译:

\begin{lstlisting}
#include <stdio.h>
\end{lstlisting}

现在,按前面的步骤再做一次,你会发现警告消失了。

\section{加分习题}

\begin{enumerate}
\item 用你的文本编辑器打开 \file{ex1} 改变或者删除任意部分。试试运行它,看会发生什么。
\item 打印出超过5行的文字或者比"hello world"复杂一些的句子。
\item 运行 \verb|man 3 puts| 了解这个函数和其它的一些函数。
\end{enumerate}


