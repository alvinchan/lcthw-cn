\chapter{Exercise 13: Switch 语句}

在其他诸如 Ruby 这样的语言里,你可以在 \ident{switch 语句}
中用任何形式的语句,在另外一些像 Python 这样的语言则不提供 \ident{switch 语句},
因为 \ident{if 语句}就能起到相同的作用。对这些语言来说,\ident{switch} 只是将 
\ident{if 语句}换了一种写法而已,而两者的内部工作原理也是一样的。

而在 C 语言里,\ident{switch 语句}是完全不一样的东西,它其实是一个“跳转表”。
你只能把结果是整数的表达式放到 \ident{switch 语句}中,而不是随便的布尔 
表达式都可以,这些整数被用作和 \ident{switch} 语句头部的值比较,如果两者匹配,
代码就会跳到子句的位置继续运行。以下是一些我们将要来分析和理解"跳转表"概念的代码。

\begin{code}{ex13.c}
<< d['code/ex13.c|pyg|l'] >>
\end{code}

在这个程序里我们从命令行参数获得一个参数,然后输出参数里所有的元音字母,这是一个乏味的演示如何使用 \ident{switch} 语句的例子。让我们来看看它是怎么工作的:
\begin{enumerate}

\item 编译器会在程序中 \ident{switch 语句}开始的地方做一个标记,我们记这个位置为 Y。
\item 然后它会评估 \verb|switch(letter)| 表达式并从中得到一个值。
在我们的例子中,这个值是命令行参数 \verb|argv[1]| 中的某个字母对应的原始 ASCII 码。
\item 编译器对于像 \verb|case 'A':| 这样的 \ident{case} 块也会产生一个相对于位置 Y 的偏移值,这个偏移值就是 'A'。所以对于在 \verb|case 'A'| 下的代码来说,它们在程序中的位置是 Y + 'A'。
\item 然后它将会做数学运算来得出 Y+letter 对应于 \ident{switch} 语句中的位置,如果得到的位置太远(即没有找到匹配的值), 它将会把这个位置值置为 Y+default。
\item 一旦程序知道了这个位置,就会跳转到代码中的那个点,然后继续运行。这就是为什么有的代码块有 \ident{break} 语句而有的没有。\footnote{译者注:作者的意思应该是没有 \ident{break} 语句的话程序将会继续执行下面的 \ident{case} 语句,反之则跳出 \ident{switch} 代码块。}
\item 如果输入了 'a',那么程序将会跳转到 \verb|case 'a'|,这里没有 break 语句,所以它将会继续往下执行,进入 \verb|case 'A'| 代码块,那里有一些代码和 break 语句。
\item 最终,它会运行该代码块,碰到 break 语句,然后退出整个 \ident{switch} 语句。
\end{enumerate}

这是对 \ident{switch} 语句如何工作的一个比较深入的研究,不过在实际应用时
你只要记住这些简单的规则:
\begin{enumerate}
\item 每次都包含一个 \ident{defaule:} 分支,这样就能捕捉到任何没有预料到的输入。
\item 除非你真的需要,否则不要允许代码中出现"运行所有分支(fall through)"这样的现象。
如果你使用了,那就添加一个注释 \verb|//fallthrough| ,这样其他人就知道这是你特意而为,
而不是因为疏忽。
\item 在你写具体的分支代码之前,最好始终同时写好 \ident{case} 语句和 \ident{break} 语句(\footnote{译者注:通过这种方式能很大程序上减少遗忘 break 语句而带来的麻烦})。
\item 如果可以的话,就用 \ident{if} 语句来取代使用 \ident{switch} 语句。
\end{enumerate}

\section{你会看到什么}

下面是我如何使用这个程序的例子,同时来演示几种不同的给程序传递参数的方式。
\begin{code}{ex13 output}
\begin{lstlisting}
<< d['code/ex13.out'] >>
\end{lstlisting}
\end{code}

注意在代码的最开始部分,有一个 \verb|return 1;| 语句,它可以保证在你没有
提供足够的参数时退出程序。使用一个不为 0 的返回值可以通知操作系统该程序
发生了错误。任何大于 0 的返回值可以被脚本和其他程序用来检查到底发生了什么。

\section{让程序出错}

要破坏一个 \ident{switch} 语句的正常运行是非常简单的,以下就是一些你可以把事情搞糟的方法:
\begin{enumerate}
\item 忘记 \ident{break} 语句,程序将会运行两个或者更多你本不期望运行的代码块。
\item 忘记 \indent{default} 语句,它就会默默地忽略那些你没考虑到的值。
\item 不小心放一个值为未预期的变量在 \ident{switch} 语句中,比如一个值很怪异的 \ident{int} 值。
\item 在 \ident{switch} 语句中使用未初始化的变量。
\end{enumerate}

你也可以用其他方法来使程序出错,看看你能否破坏它。

\section{加分习题}
\begin{enumerate}
\item 写一个新程序,对字母做一些数学操作,使他们成为小写字母,然后去除 switch 中的那些多余的大写字母的分支。
\item 使用 \verb|','| (逗号)来初始化 \ident{for 循环}中的 \ident{letter}。
\item 使它能处理你使用一个 \ident{for 循环}传递的所有参数。
\item 把这个 \ident{switch} 表达式用 \ident{if} 语句改写,你更喜欢哪一个呢?
\item 在 'Y' 这个分支中,我把 break 放在 \ident{if} 语句外面。如果把它放在
\ident{if} 语句里面会有什么影响?证明一下你的猜想是正确的。
\end{enumerate}

