\chapter{Exercise 13: Switch 语句}

在其他诸如 Ruby 这样的语言里,你可以在 \ident{switch 语句}
中用任何形式的语句,在另外一些像 Python 这样的语言则不提供 \ident{switch 语句},
因为 \ident{if 语句}就能起到相同的作用。对这些语言来说,\ident{switch} 只是将 
\ident{if语句}换了一种写法而已,而两者的内部工作原理也是一样的。
The \ident{switch-statement} is actually entirely different and is really a "jump
table".  Instead of random boolean expressions, you can only put expressions
that result in integers, and these integers are used to calculate jumps from
the top of the \ident{switch} to the part that matches that value.  Here's some
code that we'll break down to understand this concept of "jump tables":
而在 C 语言里,\ident{switch 语句}是完全不一样的东西,它其实是一个“跳转表”。
你只能把结果是整数的表达式放到 \ident{switch 语句}中,而不是随便的布尔 
表达式都可以,这些整数被用作和 \ident{switch} 语句头部的值比较,如果两者匹配,
代码就会跳到子句的位置继续运行。以下是一些我们将要来分析和理解"跳转表"概念的代码。

\begin{code}{ex13.c}
<< d['code/ex13.c|pyg|l'] >>
\end{code}

在这个程序里我们从命令行参数获得一个参数,然后输出参数里所有的元音字母,这是一个乏味的演示如何使用 \ident{switch} 语句的例子。让我们来看看它是怎么工作的:
\begin{enumerate}

\item 编译器会在程序中 \ident{switch 语句}开始的地方做一个标记,我们记这个位置为 Y。
\item 然后它会评估 \verb|switch(letter)| 表达式并从中得到一个值。
在我们的例子中,这个值是命令行参数 \verb|argv[1]| 中的某个字母对应的原始 ASCII 码。
\item 编译器对于像 \verb|case 'A':| 这样的 \ident{case} 块也会产生一个相对于位置 Y 的偏移值,这个偏移值就是 'A'。所以对于在 \verb|case 'A'| 下的代码来说,它们在程序中的位置是 Y + 'A'。
\item It then does the math to figure out where Y+letter is
    located in the \ident{switch-statement}, and if it's too
    far then it adjusts it to Y+default.
\item 然后它将会做数学运算来得出 Y+letter 对应于 \ident{switch} 语句中的位置,如果得到的位置太远(即没有找到匹配的值), 它将会把这个位置值置为 Y+default。
\item Once it knows the location, the program "jumps" to that spot
    in the code, and then continues running.  This is why you have
    \ident{break} on some of the \ident{case} blocks, but not others.
\item 一旦程序知道了这个位置,就会跳转到代码中的那个点,然后继续运行。这就是为什么有的代码块有 \ident{break} 语句而有的没有。\footnote{译者注:作者的意思应该是没有 \ident{break} 语句的话程序将会继续执行下面的 \ident{case} 语句,反之则跳出 \ident{switch} 代码块。}
\item If \verb|'a'| is entered, then it jumps to \verb|case 'a'|, there's
    no break so it "falls through" to the one right under it \verb|case 'A'|
    which has code and a \ident{break}.
\item 如果输入了 'a',那么程序将会跳转到 \verb|case 'a'|,这里没有 break 语句,所以它将会继续往下执行,进入 \verb|case 'A'| 代码块,那里有一些代码和 break 语句。
\item Finally it runs this code, hits the break then exits out of the
    \ident{switch-statement} entirely.
\item 最终,它会运行该代码块,遇到 break 语句,然后退出整个 \ident{switch} 语句。
\end{enumerate}

This is a deep dive into how the \ident{switch-statement} works, but
in practice you just have to remember a few simple rules:
这是对 \ident{switch} 语句如何工作的一个比较深入的研究,在这个练习中你需要记住的一些简单的规则是:
\begin{enumerate}
\item Always include a \ident{default:} branch so that you catch
    any missing inputs.
\item 尽量总是包含一个 \ident{defaule:} 分支,这样就能捕捉到任何没有预料到的输入。
\item Don't allow "fall through" unless you really want it, and
    it's a good idea to add a comment \verb|//fallthrough| so 
    people know it's on purpose.
\item 除非你真的需要,否则不要使用"运行所有分支(fall through)"这样的代码。如果你使用了,那么包含一个注释 \verb|//fallthrough| 来告诉其他人这是你特意而为,不是因为疏忽。
\item Always write the \ident{case} and the \ident{break} before
    you write the code that goes in it.
\item 在你写具体的分支代码之前,最好始终同时写好 \ident{case} 语句和 \ident{break} 语句(\footnote{译者注:通过这种方式能很大程序上减少遗忘 break 语句而带来的麻烦})。
\item Try to just use \ident{if-statements} instead if you can.
\item 如果可以,请用 \ident{if}  语句来取代使用 \ident{switch} 语句。
\end{enumerate}

\section{你会看到什么}
Here's an example of me playing with this, and also demonstrating
various ways to pass the argument in:
下面是我如何使用这个程序的例子,同时来演示几种不同的给程序传递参数的方式。
\begin{code}{ex13 output}
\begin{lstlisting}
<< d['code/ex13.out'] >>
\end{lstlisting}
\end{code}

Remember that there's that \ident{if-statement} at the top that
exits with a \verb|return 1;| when you don't give enough arguments.
Doing a return that's not 0 is how you indicate to the OS that
the program had an error.  Any value that's greater than 0 can be
tested for in scripts and other programs to figure out what
happened.

记住,在代码的最开始部分,有一个 \verb|return 1;| 语句保证在你没有提供足够的参数时退出程序。使用一个不为0的返回值可以通知操作系统,返回该值的程序发生了错误。任何大于0的返回值可以被脚本和其他程序用来检查到底发生了什么。

\section{让程序出错}
\section{}
It is \emph{incredibly} easy to break a \ident{switch-statement}.
Here's just a few of the ways you can mess one of these up:
要破坏一个 \ident{switch} 语句的正常运行是非常简单的,以下就是一些你可以把事情搞糟的方法:
\begin{enumerate}
\item Forget a \ident{break} and it'll run two or more
    blocks of code you don't want it to run.
\item 忘记 \ident{break} 语句将会运行两个或者更多你本不期望运行的代码块。
\item Forget a \ident{default} and have it silently
    ignore values you forgot.
\item 忘记 \indent{default} 语句使它安静地忽略那些你没考虑到的值。
\item Accidentally put in variable into the \ident{switch} that
    evaluates to something unexpected, like an \ident{int}
    that becomes weird values.
\item 错误地放一个值为未预期的变量在 \ident{switch} 语句中,比如一个值很怪异的 \ident{int} 值。
\item Use uninitialized values in the \ident{switch}.
\item 把一个未初始化的变量放到 \ident{switch} 语句中。
\end{enumerate}

You can also break this program in a few other ways.  See if you
can bust it yourself.
你也可以用其他方法来使程序错误地运行,看看你能否破坏它。
\section{Extra Credit}
\section{加分习题}
\begin{enumerate}
\item Write another program that uses math on the letter to
    convert it to lowercase, and then remove all the extraneous
    uppercase letters in the switch.
\item 写一个新程序,对字母做一些数学操作,使他们成为小写字母,然后去除 switch 中的那些多余的大写字母的分支。
\item Use the \verb|','| (comma) to initialize \ident{letter} 
    in the \ident{for-loop}.
\item 使用 \verb|','|(逗号)来初始化 \ident{for 循环}中的 \ident{letter}。
\item Make it handle all of the arguments you pass it with 
    yet another \ident{for-loop}.
\item 使它能处理你使用一个 \ident{for 循环}传递的所有参数。
\item Convert this \ident{switch-statement} to an \ident{if-statement}.
    Which do you like better?
\item 把这个 \ident{switch} 表达式用 \ident{if} 语句改写,你更喜欢哪一个呢?
\item In the case for 'Y' I have the break outside the \ident{if-statement}. What's the impact of this
    and what happens if you move it inside the \ident{if-statement}. Prove to yourself that you're right.
\item 在 'Y' 这个分支中,我把 break 放在 \ident{if} 语句外面。如果把它放在 \ident{if} 语句里面会有什么影响?
解释你的猜想并证明你是对的。
\end{enumerate}

