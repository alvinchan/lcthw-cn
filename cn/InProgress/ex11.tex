\chapter{Exercise 11: While循环和布尔表达式}

你已经体验会过了C语言的循环, 但你可能对布尔表达式\verb|i < argc|不熟悉. 在我们看看 \ident{while循环} 是如何工作之前,先解释一些事儿.

在C语言中, 根本就没有"布尔"(boolean)这种类型, 取而代之的是数字零为"false"(假) 其他数都为"true"(真). 在最后一个练习中表达式 \verb|i < argc|实际上只返回1或0, 而不是像Python中那样返回\ident{True}或\ident{False}. 这是另一个用C语言展示计算机是如何工作的例子, 因为对于计算机来说真值(truth values)只是整型数.

现在你会遇到和最后一个练习一样的问题,并且试着自己解决它,但是是使用\ident{while循环}解决. 接下来会让你比较两者,看看两者之间有什么联系.

\begin{code}{ex11.c}
<< d['code/ex11.c|pyg|l'] >>
\end{code}

\ident{while循环}和下面的格式很像:

\begin{Verbatim}
    while(TEST) {
        CODE;
    }
\end{Verbatim}

这段程序只是在\ident{TEST}为true(1)时运行\ident{CODE}.
这就是说我们要想像\ident{for循环}那样工作
得自己初始化并递增\ident{i}.

\section{What You Should See}

输出和平常一样, 我只是小小的改动一下,你可以看见另一种运行结果.

\begin{code}{ex11 output}
\begin{lstlisting}
<< d['code/ex11.out|dexy'] >>
\end{lstlisting}
\end{code}

\section{让程序出错}

在你的代码中你应该常用\ident{for循环}结构代替\ident{while循环}应为\ident{for循环}更难被中断跳出. 这有几种常用的方法:

\begin{enumerate}
\item 忘了初始化第一个循环的\verb|int i;|值,造成循环错误.
\item 忘了初始化第二个循环的参数\ident{i}所以就会以第一个循环结束时的i值一直循环. 现在你的第二个循环可能运行也可能运行不了.
\item 忘了在循环的结尾执行\verb|i++|自增,造成了"无限循环"(死循环,不会结束的循环), 这是在初学编程时会遇到的那些可怕的问题中的一个.
\end{enumerate}

\section{加分习题}

\begin{enumerate}
\item 使用\verb|i--|让循环从\verb|argc|循环计数递减到0. 你可能需要做一些计算让数组的索引正常工作.
\item 使用循环把\ident{argv}中的值\emph{拷贝}(copy)到\ident{states}中.
\item 让这个拷贝循环永远不会失败,例如,如果\ident{argv}中有太多的元素,它们并不会都拷贝到\ident{states}中.
\item 检验一下你是否真的已经完成了这些字符串的拷贝. 答案可能会使你感到惊讶与迷惑.
\end{enumerate}


