\chapter{习题2: "Make"是你的Python了}

在 \href{http://learnpythonthehardway.org}{Python} 中你只用键入\verb|python|和你希望运行的代码文件就能城启动脚本。Python解释器(interpreter)会直接运行他们,并载入任何其他你可能需要的库(libraries)。 C相对来说是完全不同的,在C中你需要编译 (\emph{compile}) 源文件并手动将它们拼凑成一个可以自己运行的二进制程序。而这个过程就是个悲剧。在上个习题中你是通过直接运行\file{make}来实现他的。

在这个习题中,你将迅速学习GUN make,当然你也会在往后的C语言学习过程中继续学习如何使用它。 Make将在这本书中成为你的"Python解释器"。它将帮助你编写代码,测试代码,准备其他,做所有Python一般会为你做的事。当然区别还是有的,在使用这个即将被展现的智能makefile的过程中,你将不需要总是说明每一件白痴的细节来使程序运行。那么接下来,在本习题以及最开始的一段时间里我们将先使用婴儿级别的make("baby make"),之后,大师级别的make("master make")才会闪亮登场。


\section{使用Make}

使用make的第一阶段就是直接写程序,事实上它已经知道怎样去构建程序了。Make对于在许多文件中构建差异是有着丰富的知识的(Make对于从一些文件来构建各种各样的其他文件有N多知识)。Make has decades of knowledge on building
a wide variety of files from other files.在上一个练习里你已经使用了像这样的命令:

\begin{Terminal}{Building ex1 with -Wall}
\begin{lstlisting}
$ make ex1
# or this one too
$ CFLAGS="-Wall" make ex1
\end{lstlisting}
\end{Terminal}

在第一个命令里你在告诉make,“我希望创建一个叫做ex1的文件”,然后make就按照以下执行:

\begin{enumerate}
\item \file{ex1} 是否已经存在了?
\item 否。好吧,那么是否有另外一个文件以 \file{ex1} 开头呢?
\item 是,它叫 \file{ex1.c}。 那么,我知道如何构建 \file{.c} 的文件吗?
\item 是, 我将运行 \verb|cc ex1.c   -o ex1| 来构建它们。
\item 我将通过使用 \file{cc} 从 \file{ex1.c} 中构建 \file{ex1}.
\end{enumerate}

第二个命令是一种通过“修改器”("modifiers")来实行make命令的一种方式。如果你不太熟悉Unix shell是如何工作的,你可以创建些“环境变量”("environment variables").有时候你可以根据你所使用的shell来运行像这样的命令: \verb|export CFLAGS="-Wall"|。你也同时可以直接将他们放在你想要运行的命令之前,这样的话,环境变量将只在命令运行时才会被设定。

在这个例子中我使用了 \verb|CFLAGS="-Wall" make ex1| , 所以它将添加命令行选项 \verb|-Wall| 到命令 \verb|cc| 中,所以
\ident{make} 一般都会运行.  命令行选项将指挥编译器 (compiler)
\ident{cc} 将报告所有警告 (这有些纠结,因为事实上并不是所有有可能的警告都会出现) to report all warnings (which in a sick twist of fate isn't
actually all the warnings possible).

事实上单单使用 \ident{make} 你就已经可以做很多事情了, 但还是让我们来看一看怎样构建一个 \file{Makefile} ,这样做可以帮助你更好的理解它。那么一开始请创建一个文件并输入以下内容:

\begin{code}{A simple Makefile}
\begin{lstlisting}
<< d['code/ex2.1.Makefile|dexy'] >>
\end{lstlisting}
\end{code}

把它在你目前的路径中保存为 \file{Makefile}. 此时你可以自动假设这里有一个叫做 \file{Makefile} 的文件,并且会直接运行。 同时, \emph{警告: 确保你只是输入了 Tab 字符, 而不是 Tab 和空格键的组合输入。}

这个 \file{Makefile} 将给你展现一些 make 的新玩意儿. 首先,我们需要在文件中设定好 \ident{CFLAGS} 并为 debugging 增加 \verb|-g| flag, 这样以来往后的编写过程中就不需要再设定它们了。 然后,我们将使用小节 \ident{clean} 来让你的小程序看起来美观一些。

先确保它在与文件 \file{ex1.c} 在相同的路径下, 并运行以下命令:

\begin{Terminal}{Running a simple Makefile}
\begin{lstlisting}
$ make clean
$ make ex1
\end{lstlisting}
\end{Terminal}


\section{你应该看到的结果}

如果一切正常你将可以看到:

\begin{Terminal}{通过Makefile完整建立 Full build with Makefile}
\begin{lstlisting}
<< d['code/ex2.out|dexy'] >>
\end{lstlisting}
\end{Terminal}

这里你可以看到我在通过运行 \verb|make clean| 来实现我们的目标 \ident{clean}.  现在再来看看 Makefile, 你会发现我在文件中使用shell命令说明了我希望运行 \ident{make}。 你可以在这里尽可能多的放置命令 (所以说这还是个不错的简化工具)。 

\begin{aside}{Did You Fix ex1.c?}
如果你修复 \file{ex1.c} 来得到 \verb|#include <stdio.h>| 那么你的输出应该不会有关于puts的警告(但事实上应该是会有一个系统错误的)。我得到了警告是因为我并没有修复它。 then your output will not have the warning (which should really be an error) about puts.  I have the error here because I didn't fix it.
\end{aside}

同时还需要注意的是,尽管我们没有在 \file{Makefile} 中提到 \file{ex1} ,这个 \ident{make} 依然知道如何去构建它\emph{并}执行我们的特别设定。


\section{让程序出错}

以上这些应该已经够你起步了,但我们还是将这个make文件用某种方式打乱一下,这样你可以进一步了解到底发生了什么。 请选择行 \verb|rm -f ex1| 并取消缩进 (将整行左移), 你将得到:

\begin{Terminal}{Bad make run}
\begin{lstlisting}
$ make clean
Makefile:4: *** missing separator.  Stop.
\end{lstlisting}
\end{Terminal}

所以说,一定要记得缩进。
如果你得到了类似的错误那么请检查一下你在使用tab键时是否保持了一致性。要知道有些make的variant数据类型是非常挑剔的。


\section{加分习题}

\begin{enumerate}
\item 仅通过命令 \verb|make| 来创建目标 \verb|all: ex1| 和 \file{ex1}。
\item 通过读取 \verb|man make| 来获取更多关于如何运行的信息。
\item 通过读取 \verb|man cc| 来获取更多关于 flag \verb|-Wall| 和 \verb|-g| 是干什么的.
\item 上网搜索一下 Makefiles 并看看你能不能进一步提高对它的使用。
\item 在其他C语言程序中找到一个 \file{Makefile} 并尝试了解它在其中的作用是什么。
\end{enumerate}

