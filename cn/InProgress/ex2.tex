\chapter{习题2: "Make"是你的Python了}

在 \href{http://learnpythonthehardway.org}{Python} 中你只用键入\verb|python|和你希望运行的代码就能城启动脚本。Python编译器(interpreter)会直接运行他们,并载入任何其他你可能需要的库(libraries)。 C相对来说是完全不同的,在C中你需要编译(\emph{compile})你的源文件并手动将它们拼凑起来变成他们可以自己运行的二进制。而这个过程就是个悲剧,在上一个练习中你可以直接运行\file{make}来实现他。

在这个习题中,你将迅速学习GUN make,并且你也会在往后的学习C语言的过程中继续学习如何使用它。 Make将在这本书中成为你的"Python编译器"。它将帮助你编写代码,测试代码,准备其他,做所有Python一般会做的事。而区别是在使用我即将向你展现的这个聪明的makefile的过程中你将不需要总是说明每一件比较白痴的细节来使程序运行。在习题2以及最开始的一段时间里我们将先使用婴儿级别的make("baby make"),之后大师级别的make("master make")才会闪亮登场。


\section{使用Make}

使用make的第一阶段就是直接写程序,它已经知道怎样去构建程序了。Make对在许多文件中构建差异有着丰富的知识(Make对于从一些文件来构建各种各样的其他文件有N多知识)。在上一个练习里你已经使用了像这样的命令:

\begin{Terminal}{Building ex1 with -Wall}
\begin{lstlisting}
$ make ex1
# or this one too
$ CFLAGS="-Wall" make ex1
\end{lstlisting}
\end{Terminal}

在第一个命令里你在告诉make,“我希望创建一个叫做ex1的文件”,然后make就做了下面的:

\begin{enumerate}
\item Does the file \file{ex1} exist already?
\item No. Ok, is there another file that starts with \file{ex1}?
\item Yes, it's called \file{ex1.c}. Do I know how to build \file{.c} files?
\item Yes, I run this command \verb|cc ex1.c   -o ex1| to build them.
\item I shall make you one \file{ex1} by using \file{cc} to build it from \file{ex1.c}.
\end{enumerate}

第二个命令是一种通过“修改器”("modifiers")来实行make命令的一种方式。如果你不太熟悉Unix shell是如何工作的,你可以创建些“环境变量”("environment variables").有时候你可以根据你所使用的shell来运行像这样的命令: \verb|export CFLAGS="-Wall"|。你也同时可以直接将他们放在你想要运行的命令之前,这样环境变量将只在命令运行时才会被设定。

In this example I did \verb|CFLAGS="-Wall" make ex1| so that it would
add the command line option \verb|-Wall| to the \verb|cc| command that
\ident{make} normally runs.  That command line option tells the compiler
\ident{cc} to report all warnings (which in a sick twist of fate isn't
actually all the warnings possible).

You can actually get pretty far with just that way of using \ident{make},
but let's get into making a \file{Makefile} so you can understand
make a little better.  To start off, create a file with just this
in it:

\begin{code}{A simple Makefile}
\begin{lstlisting}
<< d['code/ex2.1.Makefile|dexy'] >>
\end{lstlisting}
\end{code}

Save this file as \file{Makefile} in your current directory.  Make
automatically assumes there's a file called \file{Makefile} and will
just run it.  Also, \emph{WARNING: Make sure you are only entering TAB
characters, not mixtures of TAB and spaces.}

This \file{Makefile} is showing you some new stuff with make.  First we set
\ident{CFLAGS} in the file so we never have to set it again, as well
as adding the \verb|-g| flag to get debugging.  Then we have a 
section named \ident{clean} which tells make how to clean up our
little project.

Make sure it's in the same directory as your \file{ex1.c} file, and then
run these commands:

\begin{Terminal}{Running a simple Makefile}
\begin{lstlisting}
$ make clean
$ make ex1
\end{lstlisting}
\end{Terminal}

\section{What You Should See}

If that worked then you should see this:

\begin{Terminal}{Full build with Makefile}
\begin{lstlisting}
<< d['code/ex2.out|dexy'] >>
\end{lstlisting}
\end{Terminal}

Here you can see that I'm running \verb|make clean| which tells
make to run our \ident{clean} target.  Go look at the Makefile
again and you'll see that under this I indent and then I put
the shell commands I want \ident{make} to run for me.  You could
put as many commands as you wanted in there, so it's a great
automation tool.

\begin{aside}{Did You Fix ex1.c?}
If you fixed \file{ex1.c} to have \verb|#include <stdio.h>| then your
output will not have the warning (which should really be an error) about puts.  I have the error
here because I didn't fix it.
\end{aside}

Notice also that, even though we don't mention \file{ex1} in the
\file{Makefile}, \ident{make} still knows how to build it \emph{plus}
use our special settings.


\section{How To Break It}

That should be enough to get you started, but first let's break this 
make file in a particular way so you can see what happens.  Take
the line \verb|rm -f ex1| and dedent it (move it all the way left)
so you can see what happens.  Rerun \verb|make clean| and you should
get something like this:

\begin{Terminal}{Bad make run}
\begin{lstlisting}
$ make clean
Makefile:4: *** missing separator.  Stop.
\end{lstlisting}
\end{Terminal}

Always remember to indent, and if you get weird errors like this
then double check you're consistently using tab characters since
some make variants are very picky.

\section{Extra Credit}

\begin{enumerate}
\item Create an \verb|all: ex1| target that will build \file{ex1} with
    just the command \verb|make|.
\item Read \verb|man make| to find out more information on how to run it.
\item Read \verb|man cc| to find out more information on what the flags \verb|-Wall| and \verb|-g| do.
\item Research Makefiles online and see if you can improve this one even more.
\item Find a \file{Makefile} in another C project and try to understand
    what it's doing.
\end{enumerate}

