\chapter{习题 0: 准备工作}

本章你将学会架设 C 语言编程的系统环境。如果你使用的是 Linux 或者 Mac OSX,那么有个好消息
可以告诉你,你的系统本来就是\emph{为} C 语言编程设计的。C 语言的发明人也曾是 Unix 操作
系统的创作者,而 Linux 和 OSX 都是基于 Unix 的操作系统。所以,整个架设过程会非常简单。

对于 Windows 的用户,我就只有坏消息了: 在 Windows 下学习 C 语言是一件痛苦的事情。在 
Windows 下写 C 语言代码不是问题,问题是 Windows 下所有的库、函数、以及工具和 Unix 下的
C 语言环境比起来就是有那么一点“跑偏”的感觉。恐怕这也是你不得不接受的事实了。

你也别被这条突如其来的坏消息吓到。我不是说要完全避开 Windows,我的意思是说,如果你想最不
费力地学习C语言,那么你最好还是从 Unix 下手。了解一点 Unix 还有另外一个好处,那就是你可以
学到一些 C语言的惯用技巧,从而扩展你的编程技术。

这也意味着你将会用到\emph{命令行}。没错,我是说命令行。你需要在命令行输入命令。不过别害怕,
我会告诉你该输入什么命令,以及执行命令会有什么样的结果,这样你同时也会学到不少让你大开眼界
的技能。

\section{Linux}

对于大部分 Linux 系统来说,你只需要安装若干软件包就可以了。在基于 Debian 的系统(例如 
Ubuntu)上面,你只要使用下面的命令安装即可:

\begin{code}{在 Ubuntu 上面安装需求软件包}
\begin{lstlisting}
$ sudo aptitude install build-essential
\end{lstlisting}
\end{code}

以上是一个命令行的例子,所以你要先找到系统里的“命令行终端(terminal)”并且把它运行起来,
才能执行上述的命令。你会看到一个类似上面提到的 '\$' 的提示界面,然后键入上述命令。\emph{
'\$' 这个符号是无需键入的,只要把后面的内容键入即可。}

以下是基于 RPM 的 Linux 发行版要做的准备工作,以 Fedora 为例:

\begin{code}{在 Fedora 上面安装需求软件包}
\begin{lstlisting}
$ su -c yum groupinstall development-tools
\end{lstlisting}
\end{code}

执行完上述指令后,你应该可以顺利完成第一个习题了。如果不行,请向作者反馈。


\section{Mac OSX}

Mac OSX 上面的安装就更简单了。首先你需要从 Apple 下载最新版的 \ident{XCode},
或者从你的安装 DVD 中找出来安装也可以。文件很大,可能你一辈子都下不下来,所以
我还是建议你从 DVD 安装好了。另外,你可以上网搜索一下“安装 xcode”,找点说明来
看看。

装完 XCode 可能需要重启电脑。一切完成以后,你可以找到命令行终端(Terminal)程序并把它放到 dock 中。本书里会大量用到命令行终端,所以还是把它放到方便的地方比较好。

\section{Windows}

对于 Windows 用户来说,我就教教你们怎样在虚拟机里安装并运行 Ubuntu Linux 吧。这样你可以有一个做本书习题的环境,但也省得面对各种痛苦的 Linux 安装问题了。

【本节内容尚未完成】... have to figure this one out.


\section{文本编辑器}

编辑器的选择对于程序员来说总是个难题。对于初学者来说我跟他们讲用 \href{http://projects.gnome.org/gedit/}{Gedit} 就可以了,Gedit 功能简单,对代码支持也不错,不过 Gedit 在某些非英语环境下会有问题,而且如果你已经写过一阵子程序的话,你没准已经有自己喜欢的编辑器了。

在此前提下,我要求你试着用用几种支持你的系统平台的代码编辑器,然后选一个你最喜欢的坚持用下去。如果你喜欢 GEdit 就继续用 GEdit,如果你想要换个不一样的,那就简单地试试别的编辑器,然后从中选一个就行了。

最重要的一点是\emph{不要纠结于挑选一个完美的编辑器}。文本编辑器总有各种奇怪的不好用之处。选一个然后用下去就行了。如果你看到别的喜欢的编辑器,就拿来试试。别没玩没了花时间去配置打造那所谓的完美编辑器。

以下是你要去试试的编辑器:

\begin{enumerate}
\item Linux 和 OSX 下的  \href{http://projects.gnome.org/gedit/}{Gedit}。
\item OSX 下的  \href{http://www.barebones.com/products/textwrangler/}{TextWrangler}。
\item \href{http://www.nano-editor.org/}{Nano}, 这是一个命令行终端下的编辑器,基本任何地方都支持。
\item \href{http://www.gnu.org/software/emacs/}{Emacs} 以及 \href{http://emacsformacosx.com/}{Emacs for OSX},不过要做好学习的准备。
\item \href{http://www.vim.org/}{Vim} 和  \href{http://code.google.com/p/macvim/}{MacVim}。
\end{enumerate}

编辑器的种类实在太多了,给大街上的人们每人分一个大概都够,不过上面列出的只是我确定可以用的。试试其它我没提到的编辑器,收费版的也可以试试,直到找到你喜欢的为止。

\subsection{警告: 不要使用IDE}

IDE,也就是“集成开发环境(Integrated Development Environment)”,它会让你变傻。如果你想要成为一个好程序员,那么 IDE 就是最坏的工具。因为 IDE 会为你隐藏真正进行中的事情,而你的任务正是知道真正发生了什么。如果你想要完成某件任务,而该任务的系统平台又是围绕这个 IDE 设计的,这种情况下 IDE 还是有用的。不过对于学习 C 以及很多其他语言来说,IDE 是毫无意义的。

\begin{aside}{IDE和吉他谱}

玩过吉他的人都知道吉他谱是什么东西,不过还是让我给其他没玩过吉他的人解释一下吧。有一种约定俗成的音乐记谱方式叫做“线谱”,这是一种普遍的,古老的,通用的记录如何演奏乐器的方法。线谱很大程度上是为钢琴和作曲家而生,所以如果你弹钢琴的话,线谱是很容易使用的。

然而吉他这种乐器有些古怪,它并不适合这种记谱方式,所以演奏吉他的人使用了一种另类的记谱方式,称作“吉他谱(tablature)”。吉他谱告诉你的不是要演奏的音调,而是你在某一时刻要弹的指位和琴弦。你可以在不了解任何曲调的情况下学会弹奏一首曲子,很多人也是这么去学的。然而如果你想从中读出你弹奏的\emph{曲调},吉他谱就没什么用处了。

传统的记谱方式也许比吉他谱难学,但它可以告诉你如何演奏\emph{音乐},而不仅仅是如何弹吉他。拿着一份线谱,我可以走到一架钢琴前面弹出同样的一首歌曲,我可以用贝司把它弹出来,我还可以把它输入到计算机中重新设计整份乐谱。然而拿着吉他谱,我就只能用它弹弹吉他。

IDE和吉他谱类似。毫无疑问你可以使用IDE快速地写出代码,但你只能在一个固定的平台上使用一种特定的语言。这也是公司企业喜欢兜售这些东西给你的原因。他们知道你是个懒人,而IDE只在他们的平台上面工作,就这样,由于你的懒惰,他们就把你禁锢在他们的平台上了。

打破这个循环的方法也不是没有,你需要卧薪尝胆,最终学会如何不使用IDE进行编程。简单的文本编辑器,或者像Vim和Emacs这样的程序员编辑器,会让代码真正成为你的工作对象。比起使用IDE来这样会更难一些,不过最终的结果就是你可以应对\emph{任何}代码,不管它在什么样的计算机平台上,不管它使用的是什么语言,而且你懂它的深层原理。

\end{aside}
