\chapter{Exercise 23: Meet Duff's Device}
\chapter{习题 23: 接触 Duff 的设备}


This exercise is a brain teaser where I introduce you to one of the
most famous hacks in C called "Duff's Device", named after Tom Duff
the "inventor".  This little slice of awesome (evil?) has nearly everything
you've been learning wrapped in one tiny little package.  Figuring out
how it works is also a good fun puzzle.
我向你介绍这个棘手的习题是一款著名的黑客程序,它叫“Duff的设备”,用 c 语言
写的,后来被 Duff 称为 “邀请者” 。这一小块‘神来之笔’几乎囊括你已学的那一
小部分内容。了解它是如何工作的是一项富有乐趣的解谜。



\begin{aside}{This Is Only An Exercise}
Part of the fun of C is that you can come up with crazy hacks like this,
but this is also what makes C annoying to use.  It's good to learn about
these tricks because it gives you a deeper understanding of
the language and your computer.  But, you should never use this.  Always
strive for easy to read code.
\end{aside}

\begin{aside}{这是唯一的练习}
通过这些有趣的 c 语言代码可以让你领略到黑客的思路,但这不是使用 c 语言的好
方式。 学习这些技巧能让你更深的了解 c 语言和你的计算机。你最好不要去这样使
用c语言, 你应该在阅读代码上下功夫。
\end{aside}

Duff's device was "discovered" (created?) by Tom Duff and is a trick
with the C compiler that actually shouldn't work.  I won't tell you what
it does yet since this is meant to be a puzzle for you to ponder and
try to solve.  You are to get this code running and then try to figure
out what it does, and \emph{why} it does it this way.
Duff 的设备实际上是作者 Tom Duff 跟 c 编译器玩的一个花招。我不能告诉你为什么
会是这样,因为这会让你没有思索就得到了谜底。你可以取得代码并在代码运行中理解
它是如何工作的,\emph{为什么} 是这样工作的。

\begin{code}{ex23.c}
<< d['code/ex23.c|pyg|l'] >>
\end{code}

In this code I have three versions of a copy function:
在这里面我放入了3个版本的拷贝函数:

\begin{description}
\item[normal\_copy] Which is just a plain \ident{for-loop} that copies
    characters from one array to another.

\item[duffs\_device] This is the brain teaser called "Duff's Device", named
    after Tom Duff, the person to blame for this delicious evil.

\item[zeds\_device] A version of "Duff's Device" that just uses a goto so
    you can get a clue about what's happening with the weird \ident{do-while}
    placement in \func{duffs\_device}.
\end{description}

\begin{description}
\item[normal\_copy] 这是一段 \ident{for-loop} 代码,它将一串字符串复制到另一串
	字符串中。

\item[duffs\_device] 这是“ Duff 的设备” 中最棘手的一块代码,后来被命名为 Tom 
	 Duff ,人们就是将它称为 “神来之笔” 。

\item[zeds\_device] 这个版本的 “ Duff 的设备” 是用goto语句实现的,你可以使用
	\ident{do-while} 替换 \func{duffs\_device} 中switch来得到一些关于“为什么
	会这样子?”的线索。
\end{description}

Study these three functions before continuing.  Try to explain what's
going on to yourself before continuing.
在继续往下读之前,先学习这三个函数。并试着去解释为什么会是这个样子。


\section{What You Should See}

There's no output from this program, it just runs and exits.  You should
run it under valgrind and make sure there are no errors.

\section{你能看到}
这是一个没有输出的程序,程序运行完就退出了。并且在 valgrind 下运行也是无
错的。

\section{Solving The Puzzle}
The first thing to understand is that C is rather loose regarding some
of its syntax.  This is why you can put half of a \ident{do-while} in
one part of a \ident{switch-statement}, then the other half somewhere
else and it will still work.  If you look at my version with the \ident{goto again}
it's actually more clear what's going on, but make sure you understand
how that part works.


The second thing is how the default fallthrough semantics of
\ident{switch-statements} means you can jump to a particular case, and
then it will just keep running until the end of the switch.


The final clue is the \verb|count % 8| and the calculation of \ident{n} at
the top.

Now, to solve how these functions work, do the following:

\section{谜底}
首先要明白的是c的语法是相当松散的,所以\ident{do-while}放入到
\ident{switch-statement}中还能正常运行。如果你看我那个版本的\ident{goto again}
部分,能更好的理解其中的细节。但前提条件是你得知道这部分代码干了什么事。

然后需要知道最开始的 \ident{switch-statements} 会跳到对应那个case,并且会一直反复
执行直到 switch 语句结束。

最后的线索就是前面 \verb|count % 8| 计算的结果。

接下来按以下的指示去理解函数是如何工作的。

\begin{enumerate}
\item Print this code out so you can write on some paper.

\item On a piece of paper, write each of the variables in a table as they
    are when they get initialized right before the \ident{switch-statement}.

\item Follow the logic to the switch, then do the jump to the right case.

\item Update the variables, including the \ident{to}, \ident{from}, and the
    arrays they point at.

\item When you get to the \ident{while} part or my \ident{goto} alternative,
    check your variables and then follow the logic either back to the
    top of the \ident{do-while} or to where the \ident{again} label is 
    located.

\item Follow through this manual tracing, updating the variables, until
    you are sure you see how this flows.
\end{enumerate}

\begin{enumerate}
\item 将这些代码打印或者手写到纸上。

\item 在纸上用表列出在 \ident{switch-statement} 之前每一个变量的初始值。

\item 跟着 switch 的逻辑,跳转到对应的 case 中。

\item 更新变量,并包含 \ident{to} 和 \ident{from} 所指向的数组位置。

\item 选择 \ident{while}版本或我的 \ident{goto} 版本进行跟踪。检查变量,
	并跟并来回跟踪 \ident{do-while} 或者 \ident{again} 的定位标签。

\item 重复以上步骤并更新变量的值,直到你觉得你完全理解了这一部分代码是怎样工作的。
\end{enumerate}


\subsection{Why Bother?}

When you've figured out how it actually works, the final question is: Why would
you ever want to do this?  The purpose of this trick is to manually do "loop
unrolling".  Large long loops can be slow, so one way to speed them up is to
find some fixed chunk of the loop, and then just duplicate the code in the loop
out that many times sequentially.  For example, if you know a loop runs a
minimum of 20 times, then you can put the contents of the loop 20 times in the
source code.


Duff's device is basically doing this automatically by chunking up the loop
into 8 iteration chunks.  It's clever and actually works, but these days a good
compiler will do this for you.  You shouldn't need this except in the rare case
where you have \emph{proven} it would improve your speed.

\subsection{有必要这样吗?}
当你完全理解了它是如何工作的后。最后的问题是:“为什么要这样做?”。
它这样做的目的是手动“展开循环”。一次循环只一件事是很慢的,所以让其加速的方法是让
每次循环多做几件事。比如:如果你知道一个循环要运行20次,为了加速,你可以手动写20条
语句用来代替其中的循环。

“Duff 的设备” 是每次自动的让循环迭代8次。它能正常运行并非常巧妙,但是一些优秀现代编
译器已经帮你做了这些事情了。你不需要用这些罕见的玩意去优化速度。


\section{Extra Credit}

\begin{enumerate}
\item Never use this again.

\item Go look at the Wikipedia entry for "Duff's Device" and see if you can
    spot the error.  Compare it to the version I have here and read the article
    carefully to try to understand why the Wikipedia code won't work for you
    but worked for Tom Duff.

\item Create a set of macros that lets you create any length device like this.
    For example, what if you wanted to have 32 case statements and didn't want
    to write out all of them? Can you do a macro that lays down 8 at a time?

\item Change the \func{main} to conduct some speed tests to see which one is
    really the fastest.

\item Read about \func{memcpy}, \func{memmove}, \func{memset}, and also compare
    their speed.

\item Never use this again!
\end{enumerate}

\section{加分练习}
\begin{enumerate}
\item 再也别去使用这些技巧。

\item 去维基百科搜索 “Duff Device” 词条。如果你发现了错误,仔细阅读词条,比较词条上的
	代码和我们的代码。并解释为什么维基百科上的代码不能运行而我们的代码却能?

\item 创建一个宏集,让它能生产任何长度 “duff的设备”。比如,你想要一个有32条case语句
	的 “duff 设备”,你能一次性产生它们吗?你的宏一次能产生8条case语句吗?

\item 在 \func{main} 函数中放一些测速代码,看看那个地方的速度会比较快。

\item 阅读关于函数 \func{memcpy}, \func{memmove}, \func{memset} 的资料,并比较它们的速度。

\item 再次建议你别去使用这些技巧!
\end{enumerate}
