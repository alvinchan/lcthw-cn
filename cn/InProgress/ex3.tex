\chapter{Exercise 3: 格式化输出}

请保留\file{Makefile}文件,因为它可以帮你找到错误,并且在将来我们要自动化构建时需要往里面添加东西.

许多编程语言使用C方式的格式化输出, 让我们来试一下吧:

\begin{code}{ex3.c}
<< d['code/ex3.c|pyg|l'] >>
\end{code}

完成之后运行\shell{make ex3}去生成并运行它. 
确保你\emph{解决了所有警告信息}(warnings).

这些练习只需要很少的代码, 下面就来解决它们吧:

\begin{enumerate}
\item 首先包含另一个叫做\file{stdio.h}的头文件. 这样做就是告诉编译器你要使用"标准输入输出函数"。 其中一个就是\ident{printf}.
\item 接着你将用到一个叫做\ident{age}的变量并且给它赋值为10.
\item 下一步你要把变量\ident{height}赋值为72.
\item 然后你用\ident{printf}函数将打印出这个星球上最高的10孩子的年龄和身高.
\item 在\ident{printf}函数中你会注意到你传进了一个字符串,这个字符串和别的语言中的格式化字符串很像.
\item 在这个格式化字符串之后, 你放置了“代替”\ident{printf}中格式化字符串的变量.
\end{enumerate}

这么做的结果就是你把几个变量交给\ident{printf},然后它构造出一个新的字符串并且在终端中打印这个新的字符串.

\section{你会看到什么}

当你构建代码时,你会看到这些:

\begin{Terminal}{Building and running ex3.c}
\begin{lstlisting}
<< d['code/ex3.out|dexy'] >>
\end{lstlisting}
\end{Terminal}

很快我就不会再告诉你运行\file{make}查看build生成的信息,所以请确保你现在能正常工作.

\section{外部调查}

In the \emph{Extra Credit} section of each exercise I may have you go
find information on your own and figure things out.  This is an important
part of being a self-sufficient programmer.  If you constantly run to
ask someone a question before trying to figure it out first then you
never learn to solve problems independently.  This leads to you never
building confidence in your skills and always needing someone else
around to do your work.

The way you break this habit is to \emph{force} yourself to try to answer
your own questions first, and to confirm that your answer is right.  You
do this by trying to break things, experimenting with your possible answer,
and doing your own research.

For this exercise I want you to go online and find out \emph{all} of the
\ident{printf} escape codes and format sequences.  Escape codes are 
\verb|\n| or \verb|\t| that let you print a newline or tab (respectively).
Format sequences are the \verb|%s| or \verb|%d| that let you print a 
string or a integer.  Find all of the ones available, how you can
modify them, and what kind of "precisions" and widths you can do.

From now on, these kinds of tasks will be in the Extra Credit and you
should do them.

\section{How To Break It}

Try a few of these ways to break this program, which may or may
not cause it to crash on your computer:

\begin{enumerate}
\item Take the \ident{age} variable out of the first \ident{printf} call
    then recompile. You should get a couple of warnings.
\item Run this new program and it will either crash, or print out a really
    crazy age.
\item Put the \ident{printf} back the way it was, and then don't set \ident{age}
    to an initial value by changing that line to \verb|int age;| then
    rebuild and run again.
\end{enumerate}


\begin{Terminal}{Breaking ex3.c}
\begin{lstlisting}
<< d['code/ex3.bad.out|dexy'] >>
\end{lstlisting}
\end{Terminal}

\section{Extra Credit}

\begin{enumerate}
\item Find as many other ways to break \file{ex3.c} as you can.
\item Run \verb|man 3 printf| and read about the other '\%' format
    characters you can use.  These should look familiar if you used
    them in other languages (\ident{printf} is where they come from).
\item Add \file{ex3} to your \file{Makefile}'s \ident{all} list.  Use this
    to \verb|make clean all| and build all your exercises so far.
\item Add \file{ex3} to your \file{Makefile}'s \ident{clean} list as well.
    Now use \verb|make clean| will remove it when you need to.
\end{enumerate}


