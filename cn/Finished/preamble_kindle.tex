\documentclass[8pt]{book}
\usepackage{iwona,palatino}
\usepackage{style}
%\usepackage{palatino}
\usepackage[
    pdftitle={笨办法学 C 语言: 一本清晰直观的现代 C 语言编程介绍},
    pdfauthor={Zed A. Shaw},
    pdfsubject={一本写给学过一门编程语言的人的 C 语言入门书},
    pdfkeywords={C, Programming},
    bookmarks, bookmarksopen,
    pdfstartview=FitH,
    colorlinks,linkcolor=blue,citecolor=blue,
    urlcolor=red,
    CJKbookmarks=true,
]{hyperref}
\setcounter{chapter}{0}
\usepackage{upquote}
\usepackage{float}
\usepackage{fancybox}
%\usepackage{savetrees}
\usepackage{fancyvrb}
\usepackage{color}
\usepackage{parskip}
\usepackage{textcomp}
\usepackage{listings}
\usepackage{attrib}
\usepackage{ctexutf8}
\parskip 7.2pt

% ------------ Kindle Settings --------------
% Ripped from http://tex.stackexchange.com/questions/16735/latex-options-for-kindle

\usepackage{fontspec}      % font selection
%\setmainfont{Cambria}
\usepackage{breqn}         % automatic equation breaking
\usepackage{microtype}     % microtypography, reduces hyphenation
%\usepackage{polyglossia}   % language selection
%\setmainlanguage{english}

\usepackage{graphicx}      % graphics support

\usepackage[font=small,labelformat=simple,]{caption}   % customizing captions

\usepackage{titlesec}      % customizing section titles
\titleformat{\section}{\itshape\large}{}{0em}{}
\titlespacing{\section}{0pt}{8pt}{4pt}
\titleformat{\subsection}{\itshape}{}{0em}{}
\titlespacing{\subsection}{0pt}{4pt}{2pt}
\titleformat{\subsubsection}[runin]{\bf\scshape}{}{0em}{}
\titlespacing{\subsubsection}{0pt}{5pt}{5pt}

\usepackage[papersize={3.6in,4.8in},hmargin=0.1in,vmargin={0.1in,0.1in}]{geometry}  % page geometry

\usepackage{fancyhdr}   % headers and footers
\pagestyle{fancy}
\fancyhead{}            % clear page header
\fancyfoot{}            % clear page footer

\setlength{\abovecaptionskip}{2pt} % space above captions 
\setlength{\belowcaptionskip}{0pt} % space below captions
\setlength{\textfloatsep}{2pt}     % space between last top float or first bottom float and the text
\setlength{\floatsep}{2pt}         % space left between floats
\setlength{\intextsep}{2pt}        % space left on top and bottom of an in-text float

% -------------------------------------------

\lstset{basicstyle=\ttfamily,
upquote=true,
breaklines=true,
postbreak=\raisebox{0ex}[0ex][0ex]{\ensuremath{\hookrightarrow}},
breakatwhitespace=true,
numbers=left
}

% taken from http://mintaka.sdsu.edu/GF/bibliog/latex/floats.html
% Alter some LaTeX defaults for better treatment of figures:
% See p.105 of "TeX Unbound" for suggested values.
% See pp. 199-200 of Lamport's "LaTeX" book for details.
%   General parameters, for ALL pages:
\renewcommand{\topfraction}{0.9}	% max fraction of floats at top
\renewcommand{\bottomfraction}{0.8}	% max fraction of floats at bottom
%   Parameters for TEXT pages (not float pages):
\setcounter{topnumber}{2}
\setcounter{bottomnumber}{2}
\setcounter{totalnumber}{4}     % 2 may work better
\setcounter{dbltopnumber}{2}    % for 2-column pages
\renewcommand{\dbltopfraction}{0.9}	% fit big float above 2-col. text
\renewcommand{\textfraction}{0.07}	% allow minimal text w. figs
%   Parameters for FLOAT pages (not text pages):
\renewcommand{\floatpagefraction}{0.7}	% require fuller float pages
% N.B.: floatpagefraction MUST be less than topfraction !!
\renewcommand{\dblfloatpagefraction}{0.7}	% require fuller float pages

% remember to use [htp] or [htpb] for placement

