\chapter{Exercise 15: 指针啊指针,恐怖的指针}

指针是 C 语言中闻名遐迩而又神秘莫测的东西。我将教给你们一些处理指针所用的词汇
,从而揭开指针的神秘面纱。事实上指针并不复杂,只是常常被以各种方式滥用,导致它
显得十分难用。如果你尽量避免那些愚蠢的使用方法,指针实际上是相当容易的。

为了便于讨论,我写了一个简单的小程序,能够用三种不同的方式打印一组人员年龄:
\footnote{记住,学习本书时你要输入这些程序,也许一时不能理解,
但请尝试在我解释以前自己弄明白。}

\begin{code}{ex15.c}
<< d['code/ex15.c|pyg|l'] >>
\end{code}
在解释指针的工作原理之前,我们先来逐行分析这个程序,了解它的工作过程。
在你阅读这些详细描述的同时,试着在纸上写下答案,留待和我之后的解释相比较,
看看是否相同。

\begin{description}
\item[ex15.c:6-10] 创建两个数组, \ident{ages} 用以存储一些 \ident{int}
    数据,而数组 \ident{names} 则存储一些字符串。
\item[ex15.c:12-13] 稍候 \ident{for 循环}要用到的一些变量。
\item[ex15.c:16-19] 你懂的,循环访问两个数组并打印出每人的年龄。此处使用
     \ident{i} 作为数组的索引。
\item[ex15.c:24] 创建指针,指向 \ident{ages}。
    注意 \verb|int *| 的作用是创建“指向整数类的指针”。这点和
    “指向字符类的指针” \verb|char *| 相似,而字符串就是一个字符的数组。
    发现相似之处了嘛?
\item[ex15.c:25] 创建一个指针,指向\ident{names}。 \verb|char *| 
    已经是一个“指向字符类的指针”,也就是一个字符串。而 \ident{names} 是二维的,
    所以你需要的两级指针,也就是 “(字符指针)的指针” \verb|char **|。
    请推敲研究此处,尝试解释给自己听。
\item[ex15.c:28-31] 依次访问 \ident{ages} 和 \ident{names},但这次使用指针
    加上 \emph{i 的偏移量}.  写为 \verb|*(cur_name+i)|和\verb|name[i]| 是等价的,
    你可以将它读成“(指针 \ident{cur\_name} 加 i)的值”。
\item[ex15.c:35-39] 这显示了 C 语言如何奇妙的将指针和数组当作同一种事物来处理,
    仍然用数组的语法来访问数组,只要换上指针的名称就可以了。C 语言自然能分辨。
\item[ex15.c:44-50] 像前两个一样,又一个让人抓狂的循环,只是它用了不同的指针算法:
    \begin{description}
    \item[ex15.c:44] 通过将 \ident{cur\_name} 和 \ident{cur\_age} 设置到
        \ident{names} 和 \ident{ages} 数组的起点来初始化 \ident{for 循环}。
    \item[ex15.c:45] 然后 \ident{for-loop} 的测试部分比较了指针 \ident{cur\_age}
        到起点 \ident{ages} 的 \emph{距离}。为何这样可行呢?
    \item[ex15.c:46] 接下来 \ident{for-loop} 的增进部分同时增进了
        \ident{cur\_name} 和 \ident{cur\_age},以便二者指向数组
        \ident{name} 和 \ident{age} 的\emph{下一个}元素。
    \item[ex15.c:48-49] 指针 \ident{cur\_name} 和 \ident{cur\_age} 现在数组中
        正在操作的元素,我们只用 \verb|*cur_name| 和 \verb|*cur_age| 就可以将
        他们打印出来,也就是“任何 \ident{cur\_name} 指向位置的值”。
    \end{description}
\end{description}

这段程序看似简单,却包含大量信息,它的目的是使你在看我解释之前,
自己尝试理解指针。\emph{在你写下自己想法之前,请不要继续阅读}

\section{你应该看到的结果}

运行程序之后,请你尝试根据每行打印结果追溯源代码。必要时,更改
\func{printf}以确保你得到正确的行号。

\begin{code}{ex15 输出}
\begin{lstlisting}
<< d['code/ex15.out'] >>
\end{lstlisting}
\end{code}


\section{解释指针}

当你输入诸如 \verb|ages[i]| 的东西时, 其实是在数组 \ident{ages} 中“检索”,
借助 \ident{i} 中存储的数字实现所需功能。当 \ident{i} 被设为0时,相当于输入
\verb|ages[0]|。 因为数字 \ident{i} 表示我们在 \verb|age[0]| 需要访问的位置,
所以一直被称作“索引”。它也可以称作“地址”,比如说“我想要知道数组 \ident{ages}
中地址 \ident{i} 的整数是多少?”

如果 \ident{i} 是一个索引,那么 \ident{ages} 又是什么呢?对于C语言来说 \ident{ages}
是电脑内存中,所有整数的起始位置。它\emph{也}是一个地址,C编译器将会用它替换
任何带有 ages 中第一个整数地址的 \ident{ages}。另一种思考 \ident{ages} 的方法是,
它是“ages 中第一个整数的地址”。诀窍在于 \ident{ages} 是\emph{整个电脑}里的内存地址。
不像 \ident{i} 只是 \ident{ages} 的内部地址。\ident{ages}的数组名就是电脑里的实际地址。

至此我们得到一种认识:C语言将整个电脑视为一个巨大的字节数组。显然这不是很管用,
但是接下来C在这个巨大的字节数组之上加上了\emph{类型}和这些类型\emph{大小}这类概念。
从前面的练习中你已经见过这是如何运行的,现在你可以开始了解C是如何对数组进行
以下操作的:

\begin{enumerate}
\item 在你的电脑上创建一块内存区域。
\item 将该内存块起始初“指向” \ident{ages} 名称。
\item 通过基础地址 \ident{ages} “检索”该内存块并获取 \ident{i} 个字节以外的元素。
\item 将 \ident{ages+i} 处的地址转化为大小合适的有效\ident{整型},以便索引正常工作
      并返回你所需求的:索引 \ident{i} 处的整型值。
\end{enumerate}

如果你能取得基础地址,比如 \ident{ages},再“加”上另一个地址比如 \ident{i}
以产生新的地址,那么你是否能得到一个始终指向此地址的某种东西呢?正式如此,
你所得到的这种东西,就叫做“指针”。这就是指针 \ident{cur\_age} 和 \ident{cur\_name}
所做的事情。它们是指向你电脑内存里 \ident{ages} 和 \ident{names} 地址的变量。
范例程序将它们移来移去,通过相关计算获得内存存储数值。在其中一例,它们只是
将 \ident{i} 加到 \ident{cur\_age},这就相当于 \verb|array[i]|。在最后一个 \ident{for 循环}
里,两个指针没有 \ident{i} 的帮忙也能自行移动。这个循环中,指针被看作是数组和整数
偏移量的整合。

指针仅仅是指向电脑内存的一个地址,明确类型之后你就可以得到大小正确的数据。
这有点像 \ident{ages} 与 \ident{i} 合二为一的数据类型。C语言知道指针指向何处,
知道所指数据类型,该类型数据大小以及如何为你获取它们。正如 \ident{i} 一样,
你可以让它们自增、自减,或者做加、减运算。也可以像 \ident{ages} 一样,你通过它们
得到数值,输入新值,进行所有数组操作。

指针的目的,在于当数组不能完全胜任的时候,手动检索一块内存区域。只要是不使用
数组的情况,你大可放心使用指针。然后又些时候你\emph{不得不}操作原始内存区域,
这才是指针的任务。指针为你操作内存提供了原始、直接的连接方式。

本节最后一件要知道的事是,你既可以使用数组操作语法,也可以使用指针操作语法
来编写程序。你可以用指针指向某个东西,然后用数组语法接入它,也可以利用指针算数
来操作数组。在以上的范例程序中,我演示了这一点,它也是C语言的基本特征:
指针和数组(基本上)是同一回事。

\section{指针使用练习}

在C语言中有四种主要的指针处理方法:

\begin{enumerate}
\item 向系统请求一块内存作为指针工作区域。这包括字符串和你还没见过的 \ident{structs}。 
\item 利用指针向函数传递一大块内存(比如字符串和数组),这样你就不必传递整个数据。
\item 掌握函数地址以便于动态回调。
\item 内存区域的复杂扫描,比如从网络套接字中将字节转换成数据结构或者解析文件。
\end{enumerate}

对于几乎其他所有你能看到的指针使用方法,应该都是作为数组。在C语言编程的初期,
由于编译器对数组的优化还很糟糕,人们便使用指针来加速程序的运行。而现今,不同的
数组与指针语法被翻译成相同的机器码,并被同样的优化,所以指针并不是必须的。
相反,在能使用数组的情况下应该尽量使用,而只在必须使用指针进行性能优化的时候
才使用指针。

\section{指针词典}

现在我要教给你一个小小的指针词典,用来读写指针。任何遇到复杂指针声明的时候
就来参考它,然后一点点的解决问题。(或者如果代码本身不太好就直接放弃使用):

\begin{description}
\item[type *ptr] “一个叫做 ptr 的 type 型指针”
\item[*ptr] “ptr 所指向地址对应的值”
\item[*(ptr + i)] “ptr 所指地址加 i 的位置的值”
\item[\&thing] “thing 的地址”
\item[type *ptr = \&thing] “将名为 ptr 的 type 型指针设置到 thing 的地址”
\item[ptr++] “增进 ptr 的指向位置”
\end{description}

我们将用这本简单的词典解决往后书中遇到的所有指针问题。

\section{如何让程序出错}

你只需将指针指向错误的地方就能让这个程序出错。

\begin{enumerate}
\item 尝试将 \ident{cur\_age} 指向 \ident{names}。 你需要用 C 语言中的指针映射(cast)来强制执行,
	研究一下具体需要怎样做。
\item 尝试让最后一个 \ident{for 循环}中,用各种方法让算术出错。
\item 尝试重写这些循环,从结尾向开头访问数组。这比看上去要困难。
\end{enumerate}

\section{加分习题}

\begin{enumerate}
\item 将程序中所有数组重写为指针。
\item 将程序中所有指针重写为数组。
\item 回顾以往使用数组的程序,尝试使用指针替代。
\item 只用指针处理命令行参数,方法与本例中处理 \ident{names} 相似。
\item 尝试指针方法获取各种东西的地址与值。
\item 在结尾加入另一个 \ident{for 循环},打印指针被调用时地址。使用 \func{printf}
	打印时将会用到 \verb|%p| 格式。
\item 重写程序,每种打印方式使用一个函数。尝试将指针传递给函数以便它们获得数据。
	记住你可以声明一个函数接受指针,但只能按数组的方式使用它。
\item 将\ident{for循环}改为 \ident{while 循环},看看对于不同种类的指针使用,哪种循环
	效果更好。
\end{enumerate}


