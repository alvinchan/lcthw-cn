\chapter{Exercise 12: If, Else-If, Else}

在每一种语言中都共有的就是 \ident{ if 语句},C 语言中同样也有。下面的代码使用的就是 
\ident{if 语句}以确认你只输入一个或两个参数:

\begin{code}{ex12.c}
<< d['code/ex12.c|pyg|l'] >>
\end{code}

\ident{if 语句}的格式是这样的:

\begin{Verbatim}
    if(TEST) {
        CODE;
    } else if(TEST) {
        CODE;
    } else {
        CODE;
    }
\end{Verbatim}

这和其他大多数语言里的类似,除了几处 C 语言特有的不同点:

\begin{enumerate}
\item 正如前面所提到的,如果 \ident{TEST} 的部分值为 0,结果就为假,其他情况则为真。
\item 你必须把 \ident{TEST} 元素放到圆括号中,但其他的一些语言不需要你这样做。
\item 你不需要用 \verb|{}| 花括号把代码包括起来,但这是一种\emph{非常}不好的格式。
花括号可以让你更清楚地看出某个代码分支开始和结束的位置。如果你不使用花括号,各种令人讨厌
的错误就会出现。
\end{enumerate}

除了这些,它和其他语言中的功能一样。\ident{else if} 和 \ident{else} 部分也都不是必须的。

\section{你应该看到的结果}

这个程序运行起来很简单:

\begin{code}{ex12 output}
\begin{lstlisting}
<< d['code/ex12.out|dexy'] >>
\end{lstlisting}
\end{code}

\section{让程序出错}
这个例子不容易出错,因为它太简单了,不过可以试着把 \ident{ if 语句} 里的测试
条件改乱掉试一下。

\begin{enumerate}
\item 移除结尾的 \ident{else},这样程序就永远不会找到边缘条件(edge case)了。
\item 把 \verb|&&| 换成 \verb,||, 你就用“或(or)”代替了“与(and)”测试,看看结果是怎样的。
\end{enumerate}

\section{加分习题}

\begin{enumerate}
\item 你已经简单学到了 \verb|&&| ,它的功能是执行“与”的比较,再上网研究一下各种不同的
“布尔操作符(boolean operator)”。
\item 多写写条件判断语句,看看你还能得出些什么来。
\item 回到习题 10 和 11,使用 \ident{if 语句}将循环提前结束。完成操作你需要 \ident{break} 
语句。自己查查怎么用。
\item 第一个条件判断是否真的是对的呢?对于你来说"第一个参数"和用户输入的第一个参数是
不一样的。把这里修改正确。
\end{enumerate}
