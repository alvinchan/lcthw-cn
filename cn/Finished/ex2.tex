\chapter{习题2: "Make"是你的Python了}

在 \href{http://learnpythonthehardway.org}{Python} 中你只用键入\verb|python| 和你希望运行的代码文件就能城启动脚本。Python解释器(interpreter)会直接运行他们,并实时载入任何其他你可能需要的库(libraries)。 C相对来说是只完全不同的怪兽,在C中你需要编译 (\emph{compile}) 源文件并将它们拼凑成一个可以自己运行的二进制程序。这个过程若要手动完成的话那就是个悲剧。在上个习题中,你是通过直接运行\file{make}来实现的。

在本章习题是一节 GUN make 的快速入门课,当然你也会在往后的C语言学习过程中继续学习如何使用它。 在这本书剩下的章节中,Make将担任你的"Python解释器"。它将帮助你 build 代码、测试代码、完成各种准备工作,以及为你做所有 Python 程序通常能做的事。

当然 make 和 python 还是有区别的,后面我还会向你展示更加智能化的 Makefile 魔法——生成程序时你将不需要指定每一条愚蠢的细节。这些内容不会出现在本章中,不过等你用了一段时间婴儿级别的 make 之后,大师级别的 make 就会闪亮登场了。


\section{使用Make}

有些程序 make 已经知道如何 build 了,使用 make 的第一阶段就是去 build 这样的程序。Make 已经拥有了数十年积累的丰富相关知识,用以将一些文件 build 出各种其他文件。在上一个练习里你已经使用了像这样的命令:

\begin{Terminal}{Building ex1 with -Wall}
\begin{lstlisting}
$ make ex1
# or this one too
$ CFLAGS="-Wall" make ex1
\end{lstlisting}
\end{Terminal}

在第一个命令里你在告诉 make,“我希望创建一个叫做 ex1 的文件”,然后 make 就执行了以下任务:

\begin{enumerate}
\item \file{ex1} 这个文件是不是已经存在了?
\item 否。好吧,那么是否有另外一个文件以 \file{ex1} 开头呢?
\item 是,它叫 \file{ex1.c}。 那么,我知道如何 build \file{.c} 的文件吗?
\item 是, 我将运行 \verb|cc ex1.c   -o ex1| 来 build 它们。
\item 我将通过使用 \file{cc} 从 \file{ex1.c} 中 build \file{ex1}.
\end{enumerate}

第二个命令是将修饰参数传递给 make 的一种方式。如果你不熟悉 Unix shell 是如何工作的,简单的解释就是你可以创建这些“环境变量(environment variable)”,而你的程序会在运行中获取这些变量。取决于你使用的shell,有时你可以通过运行 \verb|export CFLAGS="-Wall"| 这样的命令来创建环境变量。你也可以直接将它们放在你想要运行的命令之前,这样的话,变量将只在当前命令中生效。

在这个例子中我使用了 \verb|CFLAGS="-Wall" make ex1| ,结果就是 shell 将命令行选项 \verb|-Wall| 加到 \ident{make} 会运行的 \verb|cc| 这个命令中。该命令行选项将让编译器 (compiler) \ident{cc} 报告所有警告信息。(由于命运的纠结, 其实它不能报告出所有可能的警告信息。)

事实上单单使用 \ident{make} 你就已经可以做很多事情了,但还是让我们来看一看怎样写一个 \file{Makefile} ,这会帮你更好的理解它。请创建一个文件并输入以下内容:

\begin{code}{A simple Makefile}
\begin{lstlisting}
<< d['code/ex2.1.Makefile'] >>
\end{lstlisting}
\end{code}

把它命名为 \file{Makefile} 并保存到当前目录中。make 这个命令会自动假设当前目录存在一个叫做 \file{Makefile} 的文件,并且会直接运行它。同时, \emph{警告: 确保你只输入了制表符(TAB),而非制表符和空格的混合。}

这个 \file{Makefile} 将给你展现一些 make 的新玩意儿。首先我们在里边设定好 \ident{CFLAGS},这样我们就无需在别处再次设定了,同样,我们还添加了 \verb|-g| 这个实现调试功能的参数。然后我们使用了 \ident{clean} 这一小节,它的作用是为你的小项目清理文件。

先确保它在与文件 \file{ex1.c} 在相同路径下, 然后运行以下命令:

\begin{Terminal}{Running a simple Makefile}
\begin{lstlisting}
$ make clean
$ make ex1
\end{lstlisting}
\end{Terminal}


\section{你应该看到的结果}

如果一切正常你将可以看到:

\begin{Terminal}{Full build with Makefile}
\begin{lstlisting}
<< d['code/ex2.out'] >>
\end{lstlisting}
\end{Terminal}

这里你可以看到我通过运行 \verb|make clean| 来实现我们的目标 \ident{clean}。现在再来看看 Makefile, 你会发现我在文件中写入了想要 \ident{make} 去帮我运行的shell命令。在这里你可以想放多少命令就放多少命令,所以它还是个不错的自动化工具。

\begin{aside}{你修正 ex1.c 了吗?}
如果你为 \file{ex1.c} 添加了 \verb|#include <stdio.h>|,那么你的输出应该不会有关于puts的警告信息(虽然它显示为警告信息 warning,其实说它是个错误信息 error 更合适)。我得到了警告是因为我并没有修复它。
\end{aside}

同时还需要注意的是,尽管我们没有在 \file{Makefile} 中提到 \file{ex1} ,\ident{make} 依然知道如何去 build 它\emph{并}使用了我们的特殊设定。


\section{让程序出错}

以上这些应该已经够你起步了,但我们还是将这个 make 文件用某种方式打乱一下,这样你可以进一步了解到底发生了什么。 请选择 \verb|rm -f ex1| 这一行并取消缩进 (将整行左移),你将得到:

\begin{Terminal}{Bad make run}
\begin{lstlisting}
$ make clean
Makefile:4: *** missing separator.  Stop.
\end{lstlisting}
\end{Terminal}

一定要记得缩进,如果你看到了类似的奇怪错误,那就要检查一下你在是不是统一使用了制表符。要知道有些类型的 make 的是非常挑剔的。


\section{加分习题}

\begin{enumerate}
\item 创建一个 \verb|all: ex1| 的目标,然后仅通过命令 \verb|make| 来 build 它。
\item 阅读 \verb|man make| 来获取更多的使用说明。
\item 阅读 \verb|man cc|,找出更多关于 \verb|-Wall| 和 \verb|-g| 这两个 flag 功能的说明.
\item 上网研究一下 Makefile,看看还能不能改进你写的 makefile。
\item 在其他C语言项目中找一个 \file{Makefile} 并尝试了解一下它的功能。
\end{enumerate}

