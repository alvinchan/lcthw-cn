\chapter{Exercise 30: Automated Testing}

Automated testing is used frequently in other languages like Python and Ruby, but
rarely used in C.  Part of the reason comes from the difficulty of automatically
loading and testing pieces of C code.  In this chapter we'll create a very small
little testing "framework" and get your skeleton directory building an example
test case.

The frameworks I'm going to use, and which you'll include in your \file{c-skeleton}
skeleton is called "minunit" which started with code from a tiny snippet
of code by \href{http://www.jera.com/techinfo/jtns/jtn002.html}{Jera Design}.  I then
evolved it further, to be this:

\begin{code}{tests/minunit.h}
<< d['code/c-skeleton/tests/minunit.h|pyg|l'] >>
\end{code}

There's mostly nothing left of the original, as now I'm using the \file{dbg.h} 
macros and I've created a large macro at the end for the boilerplate test
runner.  Even with this tiny amount of code we'll create a fully functioning
unit test system you can use in your C code once it's combined with a 
shell script to run the tests.

\section{Wiring Up The Test Framework}

To continue this exercise, you should have your \file{src/libex29.c} working
and that you completed the Exercise 29 extra credit where you got the 
\file{ex29.c} loader program to properly run.  In Exercise 29 I had
an extra credit to make it work like a unit test, but I'm going to start
over and show you how to do that with \file{minunit.h}.

The first thing to do is create a simple empty unit test name \file{tests/libex29\_tests.c} with this in it:

\begin{code}{tests/libex29\_tests.c.h}
<< d['code/ex30.c|pyg|l'] >>
\end{code}

This code is demonstrating the \ident{RUN\_TESTS} macro in \file{tests/minunit.h}
and how to use the other test runner macros.  I have the actual test functions
stubbed out so that you can see how to structure a unit test.  I'll break
this file down first:

\begin{description}
\item[libex29\_tests.c:1] Include the \file{minunit.h} framework.
\item[libex29\_tests.c:3-7] A first test.  Tests are structured so they take no
    arguments and return a \ident{char *} which is \ident{NULL} on \emph{success}.
    This is important because the other macros will be used to return an error
    message to the test runner.
\item[libex29\_tests.c:9-25] More tests that are the same as the first one.
\item[libex29\_tests.c:27] The runner function that will control all the other
    tests.  It has the same form as any other test case, but it gets configured
    with some additional gear.
\item[libex29\_tests.c:28] Sets up some common stuff for a test with \ident{mu\_suite\_start}.
\item[libex29\_tests.c:30] This is how you say what test to run, using the \ident{mu\_run\_test} macro.
\item[libex29\_tests.c:35] After you say what tests to run, you then return \ident{NULL} just like a normal test function.
\item[libex29\_tests.c:38] Finally, you just use the big \ident{RUN\_TESTS} macro
    to wire up the \ident{main} method with all the goodies and tell it to run
    the \ident{all\_tests} starter.
\end{description}

That's all there is to running a test, now you should try getting just this to run
within the project skeleton.  Here's what it looks like when I do it:

\begin{code}{First run of libex29\_tests}
<< d['code/ex30.sh-session|pyg|l'] >>
\end{code}

I first did a \verb|make clean| and then I ran the build, which remade the template
\file{libYOUR\_LIBRARY.a} and \file{libYOUR\_LIBRARY.so} files.  Remember that you
had to do this in the extra credit for Exercise 29, but just in case you didn't figure
it out, here's the diff for the \file{Makefile} I'm using now:

\begin{code}{Makefile changes for .so builds}
<< d['code/ex30.Makefile.diff|pyg|l'] >>
\end{code}

With those changes you should be now building everything and you can finally
fill in the remaining unit test functions:

\begin{code}{Final version of tests/libex29\_tests.c}
<< d['code/c-skeleton/tests/libex29_tests.c|pyg|l'] >>
\end{code}

Hopefully by now you can figure out what's going on, since there's nothing
new in this except for the \ident{check\_function} function.  This is a common
pattern where I see that I'll be doing a chunk of code repeatedly, and then
simply automate it either by creating a function or a macro for it.  In this
case I'm going to run functions in the \file{.so} I load so I just made
a little function to do it.

\section{Extra Credit}

\begin{enumerate}
\item This works but it's probably a bit messy.  Clean the \file{c-skeleton} 
    directory up so that it has all these files, but remove any of the code
    related to Exercise 29.  You should be able to copy this directory
    over and kickstart new projects without much editing.
\item Study the \file{runtests.sh} and go read about \program{bash} syntax
    so you know what it does.  Think you could write a C version of this
    script?
\end{enumerate}


