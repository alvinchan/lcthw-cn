\chapter{Exercise 45: A Simple TCP/IP Client}

I'm going to use the \ident{RingBuffer} to create a very simplistic little
network testing tool called \program{netclient}.  To do this I have to
add some stuff to the \ident{Makefile} to handle little programs in the
\file{bin/} directory.

\section{Augment The Makefile}

First, add a variable for the programs just like the unit tests \ident{TESTS}
and \ident{TEST\_SRC} variables:

\begin{Verbatim}
PROGRAMS_SRC=$(wildcard bin/*.c)
PROGRAMS=$(patsubst %.c,%,$(PROGRAMS_SRC))
\end{Verbatim}

Then you want to add the \ident{PROGRAMS} to the all target:

\begin{Verbatim}
all: $(TARGET) $(SO_TARGET) tests $(PROGRAMS)
\end{Verbatim}

Then add \ident{PROGRAMS} to the \program{rm} line in the \ident{clean}
target:

\begin{Verbatim}
rm -rf build $(OBJECTS) $(TESTS) $(PROGRAMS)
\end{Verbatim}

Finally you just need a target at the end to build them all:

\begin{Verbatim}
$(PROGRAMS): CFLAGS += $(TARGET)
\end{Verbatim}

With these changes you can drop simple \file{.c} files into \file{bin}
and \program{make} will build them and link them to the library just
like the tests are done.

\section{The netclient Code}

The code for the little netclient looks like this:

\begin{code}{bin/netclient.c}
<< d['code/liblcthw/bin/netclient.c|pyg|l'] >>
\end{code}

This code uses \ident{select} to handle events from both
\ident{stdin} (file descriptor 0) and from the \ident{socket}
it uses to talk to a server.  It uses \ident{RingBuffers} to
store the data and copy it around, and you can consider the
functions \ident{read\_some} and \ident{write\_some} early
prototypes for similar functions in the \ident{RingBuffer}
library.

In this little bit of code are quite a few networking functions
you may not know.  As you hit a function you don't know, look it up
in the man pages and make sure you understand it.  This one little
file could actually get you to research all the APIs required to
write a little server in C.

\section{What You Should See}

If you have everything building then the quickest way to test it is 
see if you can get a special file off learncodethehardway.org:

\begin{code}{Testing netclient}
<< d['code/ex45.1.sh-session|pyg|l'] >>
\end{code}

What I did there is I type in the syntax needed to make the HTTP request
for the file \file{/ex45.txt}, then the \ident{Host:} header line, then
hit ENTER to get an empty line.  I then get the response, with headers
and the content.  After that I just hit CTRL-c to exit.

\section{How To Break It}

This code definitely could have bugs, and currently in the draft of the
book I'm going to have to keep working on this.  In the meantime, try
analyzing the code I have here and thrashing it against other servers.
There's a tool called \ident{netcat} that is great for setting up
these kinds of servers.  Another thing to do is use a language like
\ident{Python} or \ident{Ruby} to create a simple "junk server" that
spews out junk and bad data, closes connections randomly, and other
nasty things.

If you find bugs, report them in the comments and I'll fix them up.

\section{Extra Credit}

\begin{enumerate}
\item As I mentioned, there's quite a few functions you may not know, so
    look them up.  In fact, look them all up even if you think you know
    them.
\item Run this under \program{valgrind} and look for errors.
\item Go back through and add various defensive programming checks to
    the functions to improve them.
\item Use the \ident{getopt} function to allow the user to give this
    the option to \emph{not} translate \verb|\n| to \verb|\r\n|. This
    is only needed on protocols that require it for line endings, like HTTP.
    Sometimes you don't want the translation, so give the user an option.
\end{enumerate}
