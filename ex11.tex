\chapter{Exercise 11: While-Loop And Boolean Expressions}

You've had your first taste of how C does loops, but the boolean
expression \verb|i < argc| might have not been clear to you.  Let me
explain something about it before we see how a \ident{while-loop} 
works.

In C, there's not really a "boolean" type, and instead any integer
that's 0 is "false" and otherwise it's "true".  In the last exercise
the expression \verb|i < argc| actually resulted in 1 or 0, not 
an explicit \ident{True} or \ident{False} like in Python.  This is
another example of C being closer to how a computer works, because
to a computer truth values are just integers.

Now you'll take and implement the same program from the last exercise
but use a \ident{while-loop} instead.  This will let you compare the
two so you can see how one is related to another.

\begin{code}{ex11.c}
<< d['code/ex11.c|pyg|l'] >>
\end{code}

You can see from this that a \ident{while-loop} is simpler:

\begin{Verbatim}
    while(TEST) {
        CODE;
    }
\end{Verbatim}

It simply runs the \ident{CODE} as long as \ident{TEST} is true (1).
This means that to replicate how the \ident{for-loop} works we need to
do our own initializing and incrementing of \ident{i}.

\section{What You Should See}

The output is basically the same, so I just did it a little different
so you can see another way it runs.

\begin{code}{ex11 output}
\begin{lstlisting}
<< d['code/ex11.out'] >>
\end{lstlisting}
\end{code}

\section{How To Break It}

In your own code you should favor \ident{for-loop} constructs over
\ident{while-loop} because a \ident{for-loop} is harder to break.  Here's a few
common ways:

\begin{enumerate}
\item Forget to initialize the first \verb|int i;| so have it 
    loop wrong.
\item Forget to initialize the second loop's \ident{i} so that it
    retains the value from the end of the first loop.  Now your 
    second loop might or might not run.
\item Forget to do a \verb|i++| increment at the end of the loop
    and you get a "forever loop", one of the dreaded problems
    of the first decade or two of programming.
\end{enumerate}

\section{Extra Credit}

\begin{enumerate}
\item Make these loops count backward by using \verb|i--| to start
    at \verb|argc| and count down to 0.  You may have to do some
    math to make the array indexes work right.
\item Use a while loop to \emph{copy} the values from \ident{argv}
    into \ident{states}.
\item Make this copy loop never fail such that if there's too many
    \ident{argv} elements it won't put them all into \ident{states}.
\item Research if you've really copied these strings.  The answer may
    surprise and confuse you though.
\end{enumerate}


