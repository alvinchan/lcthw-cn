\chapter{Exercise 13: Switch Statement}

In other languages like Ruby you have a \ident{switch-statement} that can take
any expression.  Some languages like Python just don't have a
\ident{switch-statement} since an \ident{if-statement} with boolean expressions
is about the same thing.  For these languages, \ident{switch-statements} are
more alternatives to \ident{if-statements} and work the same internally.

The \ident{switch-statement} is actually entirely different and is really a "jump
table".  Instead of random boolean expressions, you can only put expressions
that result in integers, and these integers are used to calculate jumps from
the top of the \ident{switch} to the part that matches that value.  Here's some
code that we'll break down to understand this concept of "jump tables":

\begin{code}{ex13.c}
<< d['code/ex13.c|pyg|l'] >>
\end{code}

In this program we take a single command line argument and print out all
of the vowels in an incredibly tedious way to demonstrate
a \ident{switch-statement}.  Here's how the \ident{switch-statement}
works:

\begin{enumerate}
\item The compiler marks the place in the program where the
    \ident{switch-statement} starts, let's call this location Y.
\item It then evaluates the expression in \verb|switch(letter)| to
    come up with a number.  In this case the number will be the
    raw ASCII code of the letter in \verb|argv[1]|.
\item The compiler has also translated each of the \ident{case} 
    blocks like \verb|case 'A':| into a location in the program
    that is that far away.  So the code under \verb|case 'A'| is
    at Y+'A' in the program.
\item It then does the math to figure out where Y+letter is
    located in the \ident{switch-statement}, and if it's too
    far then it adjusts it to Y+default.
\item Once it knows the location, the program "jumps" to that spot
    in the code, and then continues running.  This is why you have
    \ident{break} on some of the \ident{case} blocks, but not others.
\item If \verb|'a'| is entered, then it jumps to \verb|case 'a'|, there's
    no break so it "falls through" to the one right under it \verb|case 'A'|
    which has code and a \ident{break}.
\item Finally it runs this code, hits the break then exits out of the
    \ident{switch-statement} entirely.
\end{enumerate}

This is a deep dive into how the \ident{switch-statement} works, but
in practice you just have to remember a few simple rules:

\begin{enumerate}
\item Always include a \ident{default:} branch so that you catch
    any missing inputs.
\item Don't allow "fall through" unless you really want it, and
    it's a good idea to add a comment \verb|//fallthrough| so 
    people know it's on purpose.
\item Always write the \ident{case} and the \ident{break} before
    you write the code that goes in it.
\item Try to just use \ident{if-statements} instead if you can.
\end{enumerate}

\section{What You Should See}

Here's an example of me playing with this, and also demonstrating
various ways to pass the argument in:

\begin{code}{ex13 output}
\begin{lstlisting}
<< d['code/ex13.out'] >>
\end{lstlisting}
\end{code}

Remember that there's that \ident{if-statement} at the top that
exits with a \verb|return 1;| when you don't give enough arguments.
Doing a return that's not 0 is how you indicate to the OS that
the program had an error.  Any value that's greater than 0 can be
tested for in scripts and other programs to figure out what
happened.

\section{How To Break It}

It is \emph{incredibly} easy to break a \ident{switch-statement}.
Here's just a few of the ways you can mess one of these up:

\begin{enumerate}
\item Forget a \ident{break} and it'll run two or more
    blocks of code you don't want it to run.
\item Forget a \ident{default} and have it silently
    ignore values you forgot.
\item Accidentally put in variable into the \ident{switch} that
    evaluates to something unexpected, like an \ident{int}
    that becomes weird values.
\item Use uninitialized values in the \ident{switch}.
\end{enumerate}

You can also break this program in a few other ways.  See if you
can bust it yourself.

\section{Extra Credit}

\begin{enumerate}
\item Write another program that uses math on the letter to
    convert it to lowercase, and then remove all the extraneous
    uppercase letters in the switch.
\item Use the \verb|','| (comma) to initialize \ident{letter} 
    in the \ident{for-loop}.
\item Make it handle all of the arguments you pass it with 
    yet another \ident{for-loop}.
\item Convert this \ident{switch-statement} to an \ident{if-statement}.
    Which do you like better?
\item In the case for 'Y' I have the break outside the \ident{if-statement}. What's the impact of this
    and what happens if you move it inside the \ident{if-statement}. Prove to yourself that you're right.
\end{enumerate}

