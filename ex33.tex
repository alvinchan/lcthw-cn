\chapter{Exercise 33: Linked List Algorithms}

I'm going to cover two algorithms you can do on a linked list that involve
sorting.  I'm going to warn you first that if you need to sort the data, then
don't use a linked list.  These are horrible for sorting things, and there's
much better data structures you can use if that's a requirement.  I'm covering
these two algorithms because they are slightly difficult to pull off with a
linked list and get you thinking about manipulating them efficiently.

In the interest of writing this book, I'm going to put the algorithms in two
different files \file{list\_algos.h} and \file{list\_algos.c} then write a test
in \file{list\_algos\_test.c}.  For now just follow my structure, as it does
keep things clean, but if you ever work on other libraries remember this isn't
a common structure.

In this exercise I'm going to also give you an extra challenge and I want you
to try not to cheat.  I'm going to give you the \emph{unit test} first, and I
want you to type it in.  Then I want you to try and implement the two
algorithms based on their descriptions in Wikipedia before seeing if your code
is like my code.

\subsection{Bubble And Merge Sort}

You know what's awesome about the Internet?  I can just link you to the
\href{http://en.wikipedia.org/wiki/Bubble\_sort}{Bubble Sort page} and
\href{http://en.wikipedia.org/wiki/Merge\_sort}{Merge Sort page} on Wikipedia
and tell you to read that.  Man, that saves me a boat load of typing.   Now I
can tell you how to actually implement each of these using the pseudo-code they
have there.  Here's how you can tackle an algorithm like this:

\begin{enumerate}
\item Read the description and look at any visualizations it has.
\item Either draw the algorithm on paper using boxes and lines, or actually take a deck of numbered cards (like Poker Cards)
    and try to do the algorithm manually.  This gives you a concrete demonstration
    of how the algorithm works.
\item Create the skeleton functions in your \file{list\_algos.c} file and make a working \file{list\_algos.h} file, then setup
    your test harness.
\item Write your first failing test and get everything to compile.
\item Go back to the Wikipedia page and copy-paste the pseudo-code (not the C code!) into the guts of the first function you're making.
\item Translate this pseudo-code into good C code like I've taught you, using your unit test to make sure it's working.
\item Fill out some more tests for edge cases like, empty lists, already sorted lists, etc.
\item Repeat for the next algorithm and test.
\end{enumerate}

I just gave you the secret to figuring out most of the algorithms out there,
that is until you get to some of the more insane ones.  In this case you're
just doing the Bubble and Merge Sorts from Wikipedia, but those will be good
starters.

\subsection{The Unit Test}

Here is the unit test you should try to get passing:

\begin{code}{tests/list\_algos\_tests.c}
<< d['code/liblcthw/tests/list_algos_tests.c|pyg|l'] >>
\end{code}

I suggest that you start with the bubble sort and get that working, then move on to the merge sort.  What I would do is lay
out the function prototypes and skeletons that get all three files compiling, but not passing the test.  Then just fill in
the implementation until it starts working.

\subsection{The Implementation}

Are you cheating?  In future exercises I will do exercises where I just give
you a unit test and tell you to implement it, so it'll be good practice for you
to not look at this code until you get your own working.  Here's the code for
the \file{list\_algos.c} and \file{list\_algos.h}:

\begin{code}{src/lcthw/list\_algos.h}
<< d['code/liblcthw/src/lcthw/list_algos.h|pyg|l'] >>
\end{code}

\begin{code}{src/lcthw/list\_algos.c}
<< d['code/liblcthw/src/lcthw/list_algos.c|pyg|l'] >>
\end{code}

The bubble sort isn't too bad to figure out, although it is really slow.  The
merge sort is much more complicated, and honestly I could probably spend a bit
more time optimizing this code if I wanted to sacrifice clarity.

\section{What You Should See}

If everything works then you should get something like this:

\begin{code}{Ex33 Session}
<< d['code/ex33.sh-session|pyg|l'] >>
\end{code}

After this exercise I'm not going to show you this output unless it's necessary
to show you how it works.  From now on you should know that I ran the tests and
they all passed and everything compiled.

\section{How To Improve It}

Going back to the description of the algorithms, there's several ways to
improve these implementations, and there's a few obvious ones:

\begin{enumerate}
\item The merge sort does a crazy amount of copying and creating lists, find ways to reduce this.
\item The bubble sort Wikipedia description mentions a few optimizations, implement them.
\item Can you use the \ident{List\_split} and \ident{List\_join} (if you implemented them) to improve merge sort?
\item Go through all the defensive programming checks and improve the robustness of
    this implementation, protecting against bad \ident{NULL} pointers, and create
    an optional debug level invariant that does what \ident{is\_sorted} does
    after a sort.
\end{enumerate}

\section{Extra Credit}

\begin{enumerate}
\item Create a unit test that compares the performance of the two algorithms.  You'll want to look at \verb|man 3 time| for a basic timer function,
    and you'll want to run enough iterations to at least have a few seconds of samples.
\item Play with the amount of data in the lists that need to be sorted and see if that changes your timing.
\item Find a way to simulate filling different sized random lists and measuring how long they take, then graph it and see how it compares to the
    description of the algorithm.
\item Try to explain why sorting linked lists is a really bad idea.
\item Implement a \ident{List\_insert\_sorted} that will take a given value, and using the \ident{List\_compare}, insert the element at the
    right position so that the list is always sorted.  How does using this method compare to sorting a list after you've built it?
\end{enumerate}

